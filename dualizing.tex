\input{preamble}

% OK, start here.
%
\begin{document}

\title{Dualizing Complexes}


\maketitle

\phantomsection
\label{section-phantom}

\tableofcontents

\section{Introduction}
\label{section-introduction}

\noindent
In this chapter we discuss dualizing complexes in commutative algebra.
A reference is \cite{RD}.

\medskip\noindent
We begin with a discussion of
essential surjections and essential injections,
projective covers,
injective hulls,
duality for Artinian rings, and
study injective hulls of residue fields,
leading quickly to a proof of Matlis duality.
See Sections \ref{section-essential},
\ref{section-injective-modules},
\ref{section-projective-cover},
\ref{section-injective-hull},
\ref{section-artinian}, and
\ref{section-hull-residue-field} and
Proposition \ref{proposition-matlis}.

\medskip\noindent
This is followed by three sections discussing local cohomology in
great generality, see Sections \ref{section-bad-local-cohomology},
\ref{section-local-cohomology}, and \ref{section-local-cohomology-noetherian}.
We apply some of this to a discussion of depth in
Section \ref{section-depth}. In another application we show how,
given a finitely generated ideal $I$ of a ring $A$, the
``$I$-complete'' and ``$I$-torsion'' objects
of the derived category of $A$ are equivalent, see
Section \ref{section-torsion-and-complete}.
To learn more about local cohomology, for example the finiteness
theorem (which relies on local duality -- see below) please visit
Local Cohomology, Section \ref{local-cohomology-section-introduction}.

\medskip\noindent
The bulk of this chapter is devoted to duality for a ring map and
dualizing complexes. See
Sections \ref{section-trivial},
\ref{section-base-change-trivial-duality},
\ref{section-dualizing},
\ref{section-dualizing-local},
\ref{section-dimension-function},
\ref{section-local-duality},
\ref{section-dualizing-module},
\ref{section-CM},
\ref{section-gorenstein},
\ref{section-ubiquity-dualizing}, and
\ref{section-formal-fibres}.
The key definition is that of a dualizing complex
$\omega_A^\bullet$ over a Noetherian ring $A$ as an object
$\omega_A^\bullet \in D^{+}(A)$ whose cohomology modules
$H^i(\omega_A^\bullet)$ are finite $A$-modules, which has
finite injective dimension, and is such that the map
$$
A \longrightarrow R\Hom_A(\omega_A^\bullet, \omega_A^\bullet)
$$
is a quasi-isomorphism. After establishing some elementary properties
of dualizing complexes, we show a dualizing complex gives rise to a
dimension function. Next, we prove Grothendieck's local duality theorem.
After briefly discussing dualizing modules and Cohen-Macaulay rings,
we introduce Gorenstein rings and we show many familiar Noetherian
rings have dualizing complexes. In a last section we apply the material
to show there is a good theory of Noetherian local rings whose formal fibres
are Gorenstein or local complete intersections.

\medskip\noindent
In the last few sections, we describe an algebraic construction of
the ``upper shriek functors'' used in algebraic geometry, for example
in the book \cite{RD}. This topic is continued in the chapter on
duality for schemes. See
Duality for Schemes, Section \ref{duality-section-introduction}.







\section{Essential surjections and injections}
\label{section-essential}

\noindent
We will mostly work in categories of modules, but we may as well make
the definition in general.

\begin{definition}
\label{definition-essential}
Let $\mathcal{A}$ be an abelian category.
\begin{enumerate}
\item An injection $A \subset B$ of $\mathcal{A}$ is {\it essential},
or we say that $B$ is an {\it essential extension of} $A$,
if every nonzero subobject $B' \subset B$ has nonzero intersection with $A$.
\item A surjection $f : A \to B$ of $\mathcal{A}$ is {\it essential}
if for every proper subobject $A' \subset A$ we have $f(A') \not = B$.
\end{enumerate}
\end{definition}

\noindent
Some lemmas about this notion.

\begin{lemma}
\label{lemma-essential}
Let $\mathcal{A}$ be an abelian category.
\begin{enumerate}
\item If $A \subset B$ and $B \subset C$ are essential extensions, then
$A \subset C$ is an essential extension.
\item If $A \subset B$ is an essential extension and $C \subset B$
is a subobject, then $A \cap C \subset C$ is an essential extension.
\item If $A \to B$ and $B \to C$ are essential surjections, then
$A \to C$ is an essential surjection.
\item Given an essential surjection $f : A \to B$ and a surjection
$A \to C$ with kernel $K$, the morphism $C \to B/f(K)$ is an essential
surjection.
\end{enumerate}
\end{lemma}

\begin{proof}
Omitted.
\end{proof}

\begin{lemma}
\label{lemma-union-essential-extensions}
Let $R$ be a ring. Let $M$ be an $R$-module. Let $E = \colim E_i$
be a filtered colimit of $R$-modules. Suppose given a compatible
system of essential injections $M \to E_i$ of $R$-modules.
Then $M \to E$ is an essential injection.
\end{lemma}

\begin{proof}
Immediate from the definitions and the fact that filtered
colimits are exact (Algebra, Lemma \ref{algebra-lemma-directed-colimit-exact}).
\end{proof}

\begin{lemma}
\label{lemma-essential-extension}
Let $R$ be a ring. Let $M \subset N$ be $R$-modules. The following
are equivalent
\begin{enumerate}
\item $M \subset N$ is an essential extension,
\item for all $x \in N$ nonzero there exists an $f \in R$ such that $fx \in M$
and $fx \not = 0$.
\end{enumerate}
\end{lemma}

\begin{proof}
Assume (1) and let $x \in N$ be a nonzero element. By (1) we have
$Rx \cap M \not = 0$. This implies (2).

\medskip\noindent
Assume (2). Let $N' \subset N$ be a nonzero submodule. Pick $x \in N'$
nonzero. By (2) we can find $f \in R$ with $fx \in M$ and $fx \not = 0$.
Thus $N' \cap M \not = 0$.
\end{proof}




\section{Injective modules}
\label{section-injective-modules}

\noindent
Some results about injective modules over rings.

\begin{lemma}
\label{lemma-product-injectives}
Let $R$ be a ring. Any product of injective $R$-modules is injective.
\end{lemma}

\begin{proof}
Special case of Homology, Lemma \ref{homology-lemma-product-injectives}.
\end{proof}

\begin{lemma}
\label{lemma-injective-flat}
Let $R \to S$ be a flat ring map. If $E$ is an injective $S$-module,
then $E$ is injective as an $R$-module.
\end{lemma}

\begin{proof}
This is true because $\Hom_R(M, E) = \Hom_S(M \otimes_R S, E)$
by Algebra, Lemma \ref{algebra-lemma-adjoint-tensor-restrict}
and the fact that tensoring with $S$ is exact.
\end{proof}

\begin{lemma}
\label{lemma-injective-epimorphism}
Let $R \to S$ be an epimorphism of rings. Let $E$ be an $S$-module.
If $E$ is injective as an $R$-module, then $E$ is an injective $S$-module.
\end{lemma}

\begin{proof}
This is true because $\Hom_R(N, E) = \Hom_S(N, E)$ for any $S$-module $N$,
see Algebra, Lemma \ref{algebra-lemma-epimorphism-modules}.
\end{proof}

\begin{lemma}
\label{lemma-hom-injective}
Let $R \to S$ be a ring map. If $E$ is an injective $R$-module,
then $\Hom_R(S, E)$ is an injective $S$-module.
\end{lemma}

\begin{proof}
This is true because $\Hom_S(N, \Hom_R(S, E)) = \Hom_R(N, E)$ by
Algebra, Lemma \ref{algebra-lemma-adjoint-hom-restrict}.
\end{proof}

\begin{lemma}
\label{lemma-essential-extensions-in-injective}
Let $R$ be a ring. Let $I$ be an injective $R$-module. Let $E \subset I$
be a submodule. The following are equivalent
\begin{enumerate}
\item $E$ is injective, and
\item for all $E \subset E' \subset I$ with $E \subset E'$ essential
we have $E = E'$.
\end{enumerate}
In particular, an $R$-module is injective if and only if every essential
extension is trivial.
\end{lemma}

\begin{proof}
The final assertion follows from the first and the fact that the
category of $R$-modules has enough injectives
(More on Algebra, Section \ref{more-algebra-section-injectives-modules}).

\medskip\noindent
Assume (1). Let $E \subset E' \subset I$ as in (2).
Then the map $\text{id}_E : E \to E$ can be extended
to a map $\alpha : E' \to E$. The kernel of $\alpha$ has to be
zero because it intersects $E$ trivially and $E'$ is an essential
extension. Hence $E = E'$.

\medskip\noindent
Assume (2). Let $M \subset N$ be $R$-modules and let $\varphi : M \to E$
be an $R$-module map. In order to prove (1) we have to show that
$\varphi$ extends to a morphism $N \to E$. Consider the set $\mathcal{S}$
of pairs
$(M', \varphi')$ where $M \subset M' \subset N$ and $\varphi' : M' \to E$
is an $R$-module map agreeing with $\varphi$ on $M$. We define an ordering
on $\mathcal{S}$ by the rule $(M', \varphi') \leq (M'', \varphi'')$
if and only if $M' \subset M''$ and $\varphi''|_{M'} = \varphi'$.
It is clear that we can take the maximum of a totally ordered subset
of $\mathcal{S}$. Hence by Zorn's lemma we may assume $(M, \varphi)$
is a maximal element.

\medskip\noindent
Choose an extension $\psi : N \to I$ of $\varphi$ composed
with the inclusion $E \to I$. This is possible as $I$ is injective.
If $\psi(N) \subset E$, then $\psi$ is the desired extension.
If $\psi(N)$ is not contained in $E$, then by (2) the inclusion
$E \subset E + \psi(N)$ is not essential. hence
we can find a nonzero submodule $K \subset E + \psi(N)$ meeting $E$ in $0$.
This means that $M' = \psi^{-1}(E + K)$ strictly contains $M$.
Thus we can extend $\varphi$ to $M'$ using
$$
M' \xrightarrow{\psi|_{M'}} E + K \to (E + K)/K = E
$$
This contradicts the maximality of $(M, \varphi)$.
\end{proof}

\begin{example}
\label{example-reduced-ring-injective}
Let $R$ be a reduced ring. Let $\mathfrak p \subset R$ be a minimal prime
so that $K = R_\mathfrak p$ is a field
(Algebra, Lemma \ref{algebra-lemma-minimal-prime-reduced-ring}).
Then $K$ is an injective $R$-module. Namely, we have
$\Hom_R(M, K) = \Hom_K(M_\mathfrak p, K)$ for any $R$-module
$M$. Since localization is an exact functor and taking duals is
an exact functor on $K$-vector spaces we conclude $\Hom_R(-, K)$
is an exact functor, i.e., $K$ is an injective $R$-module.
\end{example}

\begin{lemma}
\label{lemma-sum-injective-modules}
Let $R$ be a Noetherian ring. A direct sum of injective modules
is injective.
\end{lemma}

\begin{proof}
Let $E_i$ be a family of injective modules parametrized by a set $I$.
Set $E = \bigcup E_i$. To show that $E$ is injective we use
Injectives, Lemma \ref{injectives-lemma-criterion-baer}.
Thus let $\varphi : I \to E$ be a module map from an ideal of $R$
into $E$. As $I$ is a finite $R$-module (because $R$ is Noetherian)
we can find finitely many elements $i_1, \ldots, i_r \in I$
such that $\varphi$ maps into $\bigcup_{j = 1, \ldots, r} E_{i_j}$.
Then we can extend $\varphi$ into $\bigcup_{j = 1, \ldots, r} E_{i_j}$
using the injectivity of the modules $E_{i_j}$.
\end{proof}

\begin{lemma}
\label{lemma-localization-injective-modules}
Let $R$ be a Noetherian ring. Let $S \subset R$ be a multiplicative
subset. If $E$ is an injective $R$-module, then $S^{-1}E$ is an
injective $S^{-1}R$-module.
\end{lemma}

\begin{proof}
Since $R \to S^{-1}R$ is an epimorphism of rings, it suffices
to show that $S^{-1}E$ is injective as an $R$-module, see
Lemma \ref{lemma-injective-epimorphism}.
To show this we use Injectives, Lemma \ref{injectives-lemma-criterion-baer}.
Thus let $I \subset R$ be an ideal and let
$\varphi : I \to S^{-1} E$ be an $R$-module map.
As $I$ is a finitely presented $R$-module (because $R$ is Noetherian)
we can find an $f \in S$ and an $R$-module map $I \to E$
such that $f\varphi$ is the composition $I \to E \to S^{-1}E$
(Algebra, Lemma \ref{algebra-lemma-hom-from-finitely-presented}).
Then we can extend $I \to E$ to a homomorphism $R \to E$.
Then the composition
$$
R \to E \to S^{-1}E \xrightarrow{f^{-1}} S^{-1}E
$$
is the desired extension of $\varphi$ to $R$.
\end{proof}

\begin{lemma}
\label{lemma-injective-module-divide}
Let $R$ be a Noetherian ring. Let $I$ be an injective $R$-module.
\begin{enumerate}
\item Let $f \in R$. Then $E = \bigcup I[f^n] = I[f^\infty]$
is an injective submodule of $I$.
\item Let $J \subset R$ be an ideal. Then the $J$-power torsion
submodule $I[J^\infty]$ is an injective submodule of $I$.
\end{enumerate}
\end{lemma}

\begin{proof}
We will use Lemma \ref{lemma-essential-extensions-in-injective}
to prove (1).
Suppose that $E \subset E' \subset I$ and that $E'$ is an essential
extension of $E$. We will show that $E' = E$. If not, then we can
find $x \in E'$ and $x \not \in E$.
Let $J = \{ a \in R \mid ax \in E\}$. Since $R$ is Noetherian,
we may write $J = (g_1, \ldots, g_t)$ for some
$g_i \in R$. By definition $E$ is the set of elements of $I$ annihilated
by powers of $f$, so we may choose integers $n_i$ so that $f^{n_i}g_ix = 0$.
Set $n = \mathrm{max}\{ n_i \}$. Then $x' = f^n x$ is an element of $E'$
not in $E$ and is annihilated by $J$. Set $J' = \{ a \in R \mid ax' \in E \}$
so $J \subset J'$. Conversely, we have $a \in J'$ if and only if $ax' \in E$
if and only if $f^m a x' = 0$ for some $m \geq 0$. But then
$f^m a x' = f^{m + n} a x$ implies $ax \in E$, i.e., $a \in J$.
Hence $J = J'$. Thus $J = J' = \text{Ann}(x')$, so $Rx' \cap E  = 0$.
Hence $E'$ is not an essential extension of $E$, a contradiction.

\medskip\noindent
To prove (2) write $J = (f_1, \ldots, f_t)$. Then
$I[J^\infty]$ is equal to
$$
(\ldots((I[f_1^\infty])[f_2^\infty])\ldots)[f_t^\infty]
$$
and the result follows from (1) and induction.
\end{proof}

\begin{lemma}
\label{lemma-injective-dimension-over-polynomial-ring}
Let $A$ be a Noetherian ring. Let $E$ be an injective $A$-module.
Then $E \otimes_A A[x]$ has injective-amplitude $[0, 1]$
as an object of $D(A[x])$. In particular, $E \otimes_A A[x]$
has finite injective dimension as an $A[x]$-module.
\end{lemma}

\begin{proof}
Let us write $E[x] = E \otimes_A A[x]$. Consider the short exact
sequence of $A[x]$-modules
$$
0 \to E[x] \to \Hom_A(A[x], E[x]) \to \Hom_A(A[x], E[x]) \to 0
$$
where the first map sends $p \in E[x]$ to $f \mapsto fp$ and the
second map sends $\varphi$ to $f \mapsto \varphi(xf) - x\varphi(f)$.
The second map is surjective because
$\Hom_A(A[x], E[x]) = \prod_{n \geq 0} E[x]$ as an abelian group and
the map sends $(e_n)$ to $(e_{n + 1} - xe_n)$ which is surjective.
As an $A$-module we have $E[x] \cong \bigoplus_{n \geq 0} E$
which is injective by Lemma \ref{lemma-sum-injective-modules}.
Hence the $A[x]$-module $\Hom_A(A[x], E[x])$ is injective by
Lemma \ref{lemma-hom-injective} and the proof is complete.
\end{proof}



\section{Projective covers}
\label{section-projective-cover}

\noindent
In this section we briefly discuss projective covers.

\begin{definition}
\label{definition-projective-cover}
Let $R$ be a ring. A surjection $P \to M$ of $R$-modules is said
to be a {\it projective cover}, or sometimes a {\it projective envelope},
if $P$ is a projective $R$-module and $P \to M$ is an essential
surjection.
\end{definition}

\noindent
Projective covers do not always exist. For example, if $k$ is a field
and $R = k[x]$ is the polynomial ring over $k$, then the module $M = R/(x)$
does not have a projective cover. Namely, for any surjection $f : P \to M$
with $P$ projective over $R$, the proper submodule $(x - 1)P$ surjects
onto $M$. Hence $f$ is not essential.

\begin{lemma}
\label{lemma-projective-cover-unique}
Let $R$ be a ring and let $M$ be an $R$-module. If a projective cover
of $M$ exists, then it is unique up to isomorphism.
\end{lemma}

\begin{proof}
Let $P \to M$ and $P' \to M$ be projective covers. Because $P$ is a
projective $R$-module and $P' \to M$ is surjective, we can find an
$R$-module map $\alpha : P \to P'$ compatible with the maps to $M$.
Since $P' \to M$ is essential, we see that $\alpha$ is surjective.
As $P'$ is a projective $R$-module we can choose a direct sum decomposition
$P = \Ker(\alpha) \oplus P'$. Since $P' \to M$ is surjective
and since $P \to M$ is essential we conclude that $\Ker(\alpha)$
is zero as desired.
\end{proof}

\noindent
Here is an example where projective covers exist.

\begin{lemma}
\label{lemma-projective-covers-local}
Let $(R, \mathfrak m, \kappa)$ be a local ring. Any finite $R$-module has
a projective cover.
\end{lemma}

\begin{proof}
Let $M$ be a finite $R$-module. Let $r = \dim_\kappa(M/\mathfrak m M)$.
Choose $x_1, \ldots, x_r \in M$ mapping to a basis of $M/\mathfrak m M$.
Consider the map $f : R^{\oplus r} \to M$. By Nakayama's lemma this is
a surjection (Algebra, Lemma \ref{algebra-lemma-NAK}). If
$N \subset R^{\oplus r}$ is a proper submodule, then
$N/\mathfrak m N \to \kappa^{\oplus r}$ is not surjective (by
Nakayama's lemma again) hence $N/\mathfrak m N \to M/\mathfrak m M$
is not surjective. Thus $f$ is an essential surjection.
\end{proof}







\section{Injective hulls}
\label{section-injective-hull}

\noindent
In this section we briefly discuss injective hulls.

\begin{definition}
\label{definition-injective-hull}
Let $R$ be a ring. A injection $M \to I$ of $R$-modules is said
to be an {\it injective hull} if $I$ is a injective $R$-module and
$M \to I$ is an essential injection.
\end{definition}

\noindent
Injective hulls always exist.

\begin{lemma}
\label{lemma-injective-hull}
Let $R$ be a ring. Any $R$-module has an injective hull.
\end{lemma}

\begin{proof}
Let $M$ be an $R$-module. By
More on Algebra, Section \ref{more-algebra-section-injectives-modules}
the category of $R$-modules has enough injectives.
Choose an injection $M \to I$ with $I$ an injective $R$-module.
Consider the set $\mathcal{S}$ of submodules $M \subset E \subset I$
such that $E$ is an essential extension of $M$. We order $\mathcal{S}$
by inclusion. If $\{E_\alpha\}$ is a totally ordered subset
of $\mathcal{S}$, then $\bigcup E_\alpha$ is an essential extension of $M$
too (Lemma \ref{lemma-union-essential-extensions}).
Thus we can apply Zorn's lemma and find a maximal element
$E \in \mathcal{S}$. We claim $M \subset E$ is an injective hull, i.e.,
$E$ is an injective $R$-module. This follows from
Lemma \ref{lemma-essential-extensions-in-injective}.
\end{proof}

\begin{lemma}
\label{lemma-injective-hull-unique}
Let $R$ be a ring. Let $M$, $N$ be $R$-modules and let $M \to E$
and $N \to E'$ be injective hulls. Then
\begin{enumerate}
\item for any $R$-module map $\varphi : M \to N$ there exists an
$R$-module map $\psi : E \to E'$ such that
$$
\xymatrix{
M \ar[r] \ar[d]_\varphi & E \ar[d]^\psi \\
N \ar[r] & E'
}
$$
commutes,
\item if $\varphi$ is injective, then $\psi$ is injective,
\item if $\varphi$ is an essential injection, then $\psi$ is an isomorphism,
\item if $\varphi$ is an isomorphism, then $\psi$ is an isomorphism,
\item if $M \to I$ is an embedding of $M$ into an injective $R$-module,
then there is an isomorphism $I \cong E \oplus I'$ compatible with
the embeddings of $M$,
\end{enumerate}
In particular, the injective hull $E$ of $M$ is unique up to isomorphism.
\end{lemma}

\begin{proof}
Part (1) follows from the fact that $E'$ is an injective $R$-module.
Part (2) follows as $\Ker(\psi) \cap M = 0$
and $E$ is an essential extension of $M$.
Assume $\varphi$ is an essential injection. Then
$E \cong \psi(E) \subset E'$ by (2) which implies
$E' = \psi(E) \oplus E''$ because $E$ is injective.
Since $E'$ is an essential extension of
$M$ (Lemma \ref{lemma-essential}) we get $E'' = 0$.
Part (4) is a special case of (3).
Assume $M \to I$ as in (5).
Choose a map $\alpha : E \to I$ extending the map $M \to I$.
Arguing as before we see that $\alpha$ is injective.
Thus as before $\alpha(E)$ splits off from $I$.
This proves (5).
\end{proof}

\begin{example}
\label{example-injective-hull-domain}
Let $R$ be a domain with fraction field $K$. Then $R \subset K$ is an
injective hull of $R$. Namely, by
Example \ref{example-reduced-ring-injective} we see that $K$ is an injective
$R$-module and by Lemma \ref{lemma-essential-extension} we see that
$R \subset K$ is an essential extension.
\end{example}

\begin{definition}
\label{definition-indecomposable}
An object $X$ of an additive category is called {\it indecomposable}
if it is nonzero and if $X = Y \oplus Z$, then either $Y = 0$ or $Z = 0$.
\end{definition}

\begin{lemma}
\label{lemma-indecomposable-injective}
Let $R$ be a ring. Let $E$ be an indecomposable injective $R$-module.
Then
\begin{enumerate}
\item $E$ is the injective hull of any nonzero submodule of $E$,
\item the intersection of any two nonzero submodules of $E$ is nonzero,
\item $\text{End}_R(E, E)$ is a noncommutative local ring with maximal
ideal those $\varphi : E \to E$ whose kernel is nonzero, and
\item the set of zerodivisors on $E$ is a prime ideal $\mathfrak p$ of $R$
and $E$ is an injective $R_\mathfrak p$-module.
\end{enumerate}
\end{lemma}

\begin{proof}
Part (1) follows from Lemma \ref{lemma-injective-hull-unique}.
Part (2) follows from part (1) and the definition of injective hulls.

\medskip\noindent
Proof of (3). Set $A = \text{End}_R(E, E)$ and
$I = \{\varphi \in A \mid \Ker(f) \not = 0\}$.
The statement means that $I$ is a two sided ideal and
that any $\varphi \in A$, $\varphi \not \in I$ is invertible.
Suppose $\varphi$ and $\psi$ are not injective.
Then $\Ker(\varphi) \cap \Ker(\psi)$ is nonzero
by (2). Hence $\varphi + \psi \in I$. It follows that $I$
is a two sided ideal. If $\varphi \in A$, $\varphi \not \in I$,
then $E \cong \varphi(E) \subset E$ is an injective submodule,
hence $E = \varphi(E)$ because $E$ is indecomposable.

\medskip\noindent
Proof of (4). Consider the ring map $R \to A$ and let $\mathfrak p \subset R$
be the inverse image of the maximal ideal $I$. Then it is clear
that $\mathfrak p$ is a prime ideal and that $R \to A$ extends to
$R_\mathfrak p \to A$. Thus $E$ is an $R_\mathfrak p$-module.
It follows from Lemma \ref{lemma-injective-epimorphism} that $E$ is injective
as an $R_\mathfrak p$-module.
\end{proof}

\begin{lemma}
\label{lemma-injective-hull-indecomposable}
Let $\mathfrak p \subset R$ be a prime of a ring $R$.
Let $E$ be the injective hull of $R/\mathfrak p$. Then
\begin{enumerate}
\item $E$ is indecomposable,
\item $E$ is the injective hull of $\kappa(\mathfrak p)$,
\item $E$ is the injective hull of $\kappa(\mathfrak p)$
over the ring $R_\mathfrak p$.
\end{enumerate}
\end{lemma}

\begin{proof}
By Lemma \ref{lemma-essential-extension} the inclusion
$R/\mathfrak p \subset \kappa(\mathfrak p)$ is an essential
extension. Then Lemma \ref{lemma-injective-hull-unique}
shows (2) holds. For $f \in R$, $f \not \in \mathfrak p$
the map $f : \kappa(\mathfrak p) \to \kappa(\mathfrak p)$ is an isomorphism
hence the map $f : E \to E$ is an isomorphism,
see Lemma \ref{lemma-injective-hull-unique}.
Thus $E$ is an $R_\mathfrak p$-module. It is injective
as an $R_\mathfrak p$-module by Lemma \ref{lemma-injective-epimorphism}.
Finally, let $E' \subset E$ be a nonzero injective $R$-submodule.
Then $J = (R/\mathfrak p) \cap E'$ is nonzero. After shrinking $E'$
we may assume that $E'$ is the injective hull of $J$ (see
Lemma \ref{lemma-injective-hull-unique} for example).
Observe that $R/\mathfrak p$ is an essential extension of $J$ for example by
Lemma \ref{lemma-essential-extension}. Hence $E' \to E$
is an isomorphism by Lemma \ref{lemma-injective-hull-unique} part (3).
Hence $E$ is indecomposable.
\end{proof}

\begin{lemma}
\label{lemma-indecomposable-injective-noetherian}
Let $R$ be a Noetherian ring. Let $E$ be an indecomposable injective
$R$-module. Then there exists a prime ideal $\mathfrak p$ of $R$ such that
$E$ is the injective hull of $\kappa(\mathfrak p)$.
\end{lemma}

\begin{proof}
Let $\mathfrak p$ be the prime ideal found in
Lemma \ref{lemma-indecomposable-injective}.
Say $\mathfrak p = (f_1, \ldots, f_r)$.
Pick a nonzero element $x \in \bigcap \Ker(f_i : E \to E)$,
see Lemma \ref{lemma-indecomposable-injective}.
Then $(R_\mathfrak p)x$ is a module isomorphic to $\kappa(\mathfrak p)$
inside $E$. We conclude by Lemma \ref{lemma-indecomposable-injective}.
\end{proof}

\begin{proposition}[Structure of injective modules over Noetherian rings]
\label{proposition-structure-injectives-noetherian}
Let $R$ be a Noetherian ring.
Every injective module is a direct sum of indecomposable injective modules.
Every indecomposable injective module is the injective hull of
the residue field at a prime.
\end{proposition}

\begin{proof}
The second statement is Lemma \ref{lemma-indecomposable-injective-noetherian}.
For the first statement, let $I$ be an injective $R$-module.
We will use transfinite recursion to construct $I_\alpha \subset I$
for ordinals $\alpha$ which are direct sums of indecomposable injective
$R$-modules $E_{\beta + 1}$ for $\beta < \alpha$.
For $\alpha = 0$ we let $I_0 = 0$. Suppose given an ordinal $\alpha$
such that $I_\alpha$ has been constructed. Then $I_\alpha$ is an
injective $R$-module by Lemma \ref{lemma-sum-injective-modules}.
Hence $I \cong I_\alpha \oplus I'$. If $I' = 0$ we are done.
If not, then $I'$ has an associated prime by
Algebra, Lemma \ref{algebra-lemma-ass-zero}.
Thus $I'$ contains a copy of $R/\mathfrak p$ for some prime $\mathfrak p$.
Hence $I'$ contains an indecomposable submodule $E$ by
Lemmas \ref{lemma-injective-hull-unique} and
\ref{lemma-injective-hull-indecomposable}. Set
$I_{\alpha + 1} = I_\alpha \oplus E_\alpha$.
If $\alpha$ is a limit ordinal and $I_\beta$ has been constructed
for $\beta < \alpha$, then we set
$I_\alpha = \bigcup_{\beta < \alpha} I_\beta$.
Observe that $I_\alpha = \bigoplus_{\beta < \alpha} E_{\beta + 1}$.
This concludes the proof.
\end{proof}



\section{Duality over Artinian local rings}
\label{section-artinian}

\noindent
Let $(R, \mathfrak m, \kappa)$ be an artinian local ring.
Recall that this implies $R$ is Noetherian and that $R$ has finite
length as an $R$-module. Moreover an $R$-module is finite if and
only if it has finite length. We will use these facts without
further mention in this section. Please see
Algebra, Sections \ref{algebra-section-length} and
\ref{algebra-section-artinian}
and
Algebra, Proposition \ref{algebra-proposition-dimension-zero-ring}
for more details.

\begin{lemma}
\label{lemma-finite}
Let $(R, \mathfrak m, \kappa)$ be an artinian local ring.
Let $E$ be an injective hull of $\kappa$. For every finite
$R$-module $M$ we have
$$
\text{length}_R(M) = \text{length}_R(\Hom_R(M, E))
$$
In particular, the injective hull $E$ of $\kappa$ is a finite $R$-module.
\end{lemma}

\begin{proof}
Because $E$ is an essential extension of $\kappa$ we have
$\kappa = E[\mathfrak m]$ where $E[\mathfrak m]$ is the
$\mathfrak m$-torsion in $E$ (notation as in More on Algebra, Section
\ref{more-algebra-section-formal-glueing}).
Hence $\Hom_R(\kappa, E) \cong \kappa$ and the equality of lengths
holds for $M = \kappa$. We prove the displayed equality of the lemma
by induction on the length of $M$. If $M$ is nonzero there exists a surjection
$M \to \kappa$ with kernel $M'$. Since the functor $M \mapsto \Hom_R(M, E)$
is exact we obtain a short exact sequence
$$
0 \to \Hom_R(\kappa, E) \to \Hom_R(M, E) \to \Hom_R(M', E) \to 0.
$$
Additivity of length for this sequence and the sequence
$0 \to M' \to M \to \kappa \to 0$ and the equality for $M'$ (induction
hypothesis) and $\kappa$ implies the equality for $M$.
The final statement of the lemma follows as $E = \Hom_R(R, E)$.
\end{proof}

\begin{lemma}
\label{lemma-evaluate}
Let $(R, \mathfrak m, \kappa)$ be an artinian local ring.
Let $E$ be an injective hull of $\kappa$.
For any finite $R$-module $M$ the evaluation map
$$
M \longrightarrow \Hom_R(\Hom_R(M, E), E)
$$
is an isomorphism. In particular $R = \Hom_R(E, E)$.
\end{lemma}

\begin{proof}
Observe that the displayed arrow is injective. Namely, if $x \in M$ is
a nonzero element, then there is a nonzero map $Rx \to \kappa$ which
we can extend to a map $\varphi : M \to E$ that doesn't vanish on $x$.
Since the source and target of the arrow have the same length by
Lemma \ref{lemma-finite}
we conclude it is an isomorphism. The final statement follows
on taking $M = R$.
\end{proof}

\noindent
To state the next lemma, denote $\text{Mod}^{fg}_R$ the category of finite
$R$-modules over a ring $R$.

\begin{lemma}
\label{lemma-duality}
Let $(R, \mathfrak m, \kappa)$ be an artinian local ring.
Let $E$ be an injective hull of $\kappa$.
The functor $D(-) = \Hom_R(-, E)$ induces an exact anti-equivalence
$\text{Mod}^{fg}_R \to \text{Mod}^{fg}_R$ and
$D \circ D \cong \text{id}$.
\end{lemma}

\begin{proof}
We have seen that $D \circ D = \text{id}$ on $\text{Mod}^{fg}_R$
in Lemma \ref{lemma-evaluate}. It follows immediately that
$D$ is an anti-equivalence.
\end{proof}

\begin{lemma}
\label{lemma-duality-torsion-cotorsion}
Assumptions and notation as in Lemma \ref{lemma-duality}.
Let $I \subset R$ be an ideal and $M$ a finite $R$-module.
Then
$$
D(M[I]) = D(M)/ID(M) \quad\text{and}\quad D(M/IM) = D(M)[I]
$$
\end{lemma}

\begin{proof}
Say $I = (f_1, \ldots, f_t)$. Consider the map
$$
M^{\oplus t} \xrightarrow{f_1, \ldots, f_t} M
$$
with cokernel $M/IM$. Applying the exact functor $D$ we conclude that
$D(M/IM)$ is $D(M)[I]$. The other case is proved in the same way.
\end{proof}



\section{Injective hull of the residue field}
\label{section-hull-residue-field}

\noindent
Most of our results will be for Noetherian local rings in this section.

\begin{lemma}
\label{lemma-quotient}
Let $R \to S$ be a surjective map of local rings with kernel $I$.
Let $E$ be the injective hull of the residue field of $R$ over $R$.
Then $E[I]$ is the injective hull of the residue field of $S$ over $S$.
\end{lemma}

\begin{proof}
Observe that $E[I] = \Hom_R(S, E)$ as $S = R/I$. Hence $E[I]$ is an injective
$S$-module by Lemma \ref{lemma-hom-injective}. Since $E$ is an essential
extension of $\kappa = R/\mathfrak m_R$ it follows that $E[I]$ is an
essential extension of $\kappa$ as well. The result follows.
\end{proof}

\begin{lemma}
\label{lemma-torsion-submodule-sum-injective-hulls}
Let $(R, \mathfrak m, \kappa)$ be a local ring.
Let $E$ be the injective hull of $\kappa$.
Let $M$ be a $\mathfrak m$-power torsion $R$-module
with $n = \dim_\kappa(M[\mathfrak m]) < \infty$.
Then $M$ is isomorphic to a submodule of $E^{\oplus n}$.
\end{lemma}

\begin{proof}
Observe that $E^{\oplus n}$ is the injective hull of
$\kappa^{\oplus n} = M[\mathfrak m]$. Thus there is an $R$-module map
$M \to E^{\oplus n}$ which is injective on $M[\mathfrak m]$.
Since $M$ is $\mathfrak m$-power torsion the inclusion
$M[\mathfrak m] \subset M$ is an essential extension
(for example by Lemma \ref{lemma-essential-extension})
we conclude that the kernel of $M \to E^{\oplus n}$ is zero.
\end{proof}

\begin{lemma}
\label{lemma-union-artinian}
Let $(R, \mathfrak m, \kappa)$ be a Noetherian local ring.
Let $E$ be an injective hull of $\kappa$ over $R$.
Let $E_n$ be an injective hull of $\kappa$ over $R/\mathfrak m^n$.
Then $E = \bigcup E_n$ and $E_n = E[\mathfrak m^n]$.
\end{lemma}

\begin{proof}
We have $E_n = E[\mathfrak m^n]$ by Lemma \ref{lemma-quotient}.
We have $E = \bigcup E_n$ because $\bigcup E_n = E[\mathfrak m^\infty]$
is an injective $R$-submodule which contains $\kappa$, see
Lemma \ref{lemma-injective-module-divide}.
\end{proof}

\noindent
The following lemma tells us the injective hull of the residue
field of a Noetherian local ring only depends on the completion.

\begin{lemma}
\label{lemma-compare}
Let $R \to S$ be a flat local homomorphism of local Noetherian rings
such that $R/\mathfrak m_R \cong S/\mathfrak m_R S$.
Then the injective hull of the residue field
of $R$ is the injective hull of the residue field of $S$.
\end{lemma}

\begin{proof}
Set $\kappa = R/\mathfrak m_R = S/\mathfrak m_S$.
Let $E_R$ be the injective hull of $\kappa$ over $R$.
Let $E_S$ be the injective hull of $\kappa$ over $S$.
Observe that $E_S$ is an injective $R$-module by
Lemma \ref{lemma-injective-flat}.
Choose an extension $E_R \to E_S$ of the identification of
residue fields. This map is an isomorphism by
Lemma \ref{lemma-union-artinian}
because $R \to S$ induces an isomorphism
$R/\mathfrak m_R^n \to S/\mathfrak m_S^n$ for all $n$.
\end{proof}

\begin{lemma}
\label{lemma-endos}
Let $(R, \mathfrak m, \kappa)$ be a Noetherian local ring.
Let $E$ be an injective hull of $\kappa$ over $R$. Then
$\Hom_R(E, E)$ is canonically isomorphic to the completion of $R$.
\end{lemma}

\begin{proof}
Write $E = \bigcup E_n$ with $E_n = E[\mathfrak m^n]$ as in
Lemma \ref{lemma-union-artinian}. Any endomorphism of $E$
preserves this filtration. Hence
$$
\Hom_R(E, E) = \lim \Hom_R(E_n, E_n)
$$
The lemma follows as
$\Hom_R(E_n, E_n) = \Hom_{R/\mathfrak m^n}(E_n, E_n) = R/\mathfrak m^n$
by Lemma \ref{lemma-evaluate}.
\end{proof}

\begin{lemma}
\label{lemma-injective-hull-has-dcc}
Let $(R, \mathfrak m, \kappa)$ be a Noetherian local ring.
Let $E$ be an injective hull of $\kappa$ over $R$. Then
$E$ satisfies the descending chain condition.
\end{lemma}

\begin{proof}
If $E \supset M_1 \supset M_2 \supset \ldots$ is a sequence of submodules, then
$$
\Hom_R(E, E) \to \Hom_R(M_1, E) \to \Hom_R(M_2, E) \to \ldots
$$
is sequence of surjections. By Lemma \ref{lemma-endos} each of these is a
module over the completion $R^\wedge = \Hom_R(E, E)$.
Since $R^\wedge$ is Noetherian
(Algebra, Lemma \ref{algebra-lemma-completion-Noetherian-Noetherian})
the sequence stabilizes: $\Hom_R(M_n, E) = \Hom_R(M_{n + 1}, E) = \ldots$.
Since $E$ is injective, this can only happen if $\Hom_R(M_n/M_{n + 1}, E)$
is zero. However, if $M_n/M_{n + 1}$ is nonzero, then it contains a
nonzero element annihilated by $\mathfrak m$, because $E$ is
$\mathfrak m$-power torsion by Lemma \ref{lemma-union-artinian}.
In this case $M_n/M_{n + 1}$ has a nonzero map into $E$, contradicting
the assumed vanishing. This finishes the proof.
\end{proof}

\begin{lemma}
\label{lemma-describe-categories}
Let $(R, \mathfrak m, \kappa)$ be a Noetherian local ring.
Let $E$ be an injective hull of $\kappa$.
\begin{enumerate}
\item For an $R$-module $M$ the following are equivalent:
\begin{enumerate}
\item $M$ satisfies the ascending chain condition,
\item $M$ is a finite $R$-module, and
\item there exist $n, m$ and an exact sequence
$R^{\oplus m} \to R^{\oplus n} \to M \to 0$.
\end{enumerate}
\item For an $R$-module $M$ the following are equivalent:
\begin{enumerate}
\item $M$ satisfies the descending chain condition,
\item $M$ is $\mathfrak m$-power torsion and
$\dim_\kappa(M[\mathfrak m]) < \infty$, and
\item there exist $n, m$ and an exact sequence
$0 \to M \to E^{\oplus n} \to E^{\oplus m}$.
\end{enumerate}
\end{enumerate}
\end{lemma}

\begin{proof}
We omit the proof of (1).

\medskip\noindent
Let $M$ be an $R$-module with the descending chain condition. Let $x \in M$.
Then $\mathfrak m^n x$ is a descending chain of submodules, hence stabilizes.
Thus $\mathfrak m^nx = \mathfrak m^{n + 1}x$ for some $n$. By Nakayama's lemma
(Algebra, Lemma \ref{algebra-lemma-NAK}) this implies $\mathfrak m^n x = 0$,
i.e., $x$ is $\mathfrak m$-power torsion. Since $M[\mathfrak m]$ is a vector
space over $\kappa$ it has to be finite dimensional in order to have the
descending chain condition.

\medskip\noindent
Assume that $M$ is $\mathfrak m$-power torsion and has a finite dimensional
$\mathfrak m$-torsion submodule $M[\mathfrak m]$. By
Lemma \ref{lemma-torsion-submodule-sum-injective-hulls}
we see that $M$ is a submodule of $E^{\oplus n}$ for some $n$.
Consider the quotient $N = E^{\oplus n}/M$. By
Lemma \ref{lemma-injective-hull-has-dcc} the module $E$ has the
descending chain condition hence so do $E^{\oplus n}$ and $N$.
Therefore $N$ satisfies (2)(a) which implies $N$ satisfies
(2)(b) by the second paragraph of the proof. Thus by
Lemma \ref{lemma-torsion-submodule-sum-injective-hulls}
again we see that $N$ is a submodule of $E^{\oplus m}$ for some $m$.
Thus we have a short exact sequence
$0 \to M \to E^{\oplus n} \to E^{\oplus m}$.

\medskip\noindent
Assume we have a short exact sequence
$0 \to M \to E^{\oplus n} \to E^{\oplus m}$.
Since $E$ satisfies the descending chain condition by
Lemma \ref{lemma-injective-hull-has-dcc}
so does $M$.
\end{proof}

\begin{proposition}[Matlis duality]
\label{proposition-matlis}
Let $(R, \mathfrak m, \kappa)$ be a complete local Noetherian ring.
Let $E$ be an injective hull of $\kappa$ over $R$. The functor
$D(-) = \Hom_R(-, E)$ induces an anti-equivalence
$$
\left\{
\begin{matrix}
R\text{-modules with the} \\
\text{descending chain condition}
\end{matrix}
\right\}
\longleftrightarrow
\left\{
\begin{matrix}
R\text{-modules with the} \\
\text{ascending chain condition}
\end{matrix}
\right\}
$$
and we have $D \circ D = \text{id}$ on either side of the equivalence.
\end{proposition}

\begin{proof}
By Lemma \ref{lemma-endos} we have $R = \Hom_R(E, E) = D(E)$.
Of course we have $E = \Hom_R(R, E) = D(R)$. Since $E$ is injective
the functor $D$ is exact. The result now follows immediately from the
description of the categories in
Lemma \ref{lemma-describe-categories}.
\end{proof}

\begin{remark}
\label{remark-matlis}
Let $(R, \mathfrak m, \kappa)$ be a Noetherian local ring.
Let $E$ be an injective hull of $\kappa$ over $R$. Here is an
addendum to Matlis duality: If $N$ is an $\mathfrak m$-power torsion module
and $M = \Hom_R(N, E)$ is a finite module over the completion of $R$,
then $N$ satisfies the descending chain condition. Namely, for any
submodules $N'' \subset N' \subset N$ with $N'' \not = N'$, we can
find an embedding $\kappa \subset N''/N'$ and hence a nonzero
map $N' \to E$ annihilating $N''$ which we can extend to a map $N \to E$
annihilating $N''$. Thus $N \supset N' \mapsto M' = \Hom_R(N/N', E) \subset M$
is an inclusion preserving map from submodules of $N$ to submodules
of $M$, whence the conclusion.
\end{remark}




















\section{Deriving torsion}
\label{section-bad-local-cohomology}

\noindent
Let $A$ be a ring and let $I \subset A$ be a finitely generated ideal
(if $I$ is not finitely generated perhaps a different definition
should be used). Let $Z = V(I) \subset \Spec(A)$. Recall that the
category $I^\infty\text{-torsion}$ of $I$-power torsion modules
only depends on the closed subset $Z$ and not on the choice of the
finitely generated ideal $I$ such that $Z = V(I)$, see
More on Algebra, Lemma \ref{more-algebra-lemma-local-cohomology-closed}.
In this section we will consider the functor
$$
H^0_{I} : \text{Mod}_A \longrightarrow I^\infty\text{-torsion},\quad
M \longmapsto M[I^\infty] = \bigcup M[I^n]
$$
which sends $M$ to the submodule of $I$-power torsion.

\medskip\noindent
Let $A$ be a ring and let $I$ be a finitely generated ideal.
Note that $I^\infty\text{-torsion}$ is a Grothendieck
abelian category (direct sums exist, filtered colimits are
exact, and $\bigoplus A/I^n$ is a generator by
More on Algebra, Lemma \ref{more-algebra-lemma-I-power-torsion-presentation}).
Hence the derived category $D(I^\infty\text{-torsion})$ exists, see
Injectives, Remark \ref{injectives-remark-existence-D}.
Our functor $H^0_I$ is left exact and has a derived extension
which we will denote
$$
R\Gamma_I : D(A) \longrightarrow D(I^\infty\text{-torsion}).
$$
{\bf Warning:} this functor does not deserve the name
local cohomology unless the ring $A$ is Noetherian.
The functors $H^0_I$, $R\Gamma_I$, and the satellites $H^p_I$
only depend on the closed subset $Z \subset \Spec(A)$ and not
on the choice of the finitely generated ideal $I$ such that
$V(I) = Z$. However, we insist on using the subscript $I$ for
the functors above as the notation $R\Gamma_Z$ is going
to be used for a different functor, see
(\ref{equation-local-cohomology}), which
agrees with the functor $R\Gamma_I$ only (as far as we know)
in case $A$ is Noetherian
(see Lemma \ref{lemma-local-cohomology-noetherian}).

\begin{lemma}
\label{lemma-adjoint}
Let $A$ be a ring and let $I \subset A$ be a finitely generated ideal.
The functor $R\Gamma_I$ is right adjoint to the functor
$D(I^\infty\text{-torsion}) \to D(A)$.
\end{lemma}

\begin{proof}
This follows from the fact that taking $I$-power torsion submodules
is the right adjoint to the inclusion functor
$I^\infty\text{-torsion} \to \text{Mod}_A$. See
Derived Categories, Lemma \ref{derived-lemma-derived-adjoint-functors}.
\end{proof}

\begin{lemma}
\label{lemma-local-cohomology-ext}
Let $A$ be a ring and let $I \subset A$ be a finitely generated ideal.
For any object $K$ of $D(A)$ we have
$$
R\Gamma_I(K) = \text{hocolim}\ R\Hom_A(A/I^n, K)
$$
in $D(A)$ and
$$
R^q\Gamma_I(K) = \colim_n \Ext_A^q(A/I^n, K)
$$
as modules for all $q \in \mathbf{Z}$.
\end{lemma}

\begin{proof}
Let $J^\bullet$ be a K-injective complex representing $K$. Then
$$
R\Gamma_I(K) = J^\bullet[I^\infty] = \colim J^\bullet[I^n] =
\colim \Hom_A(A/I^n, J^\bullet)
$$
where the first equality is the definition of $R\Gamma_I(K)$.
By Derived Categories, Lemma \ref{derived-lemma-colim-hocolim}
we obtain the first displayed equality in the statement of the lemma.
The second displayed equality in the statement of the lemma then
follows because $H^q(\Hom_A(A/I^n, J^\bullet)) = \Ext^q_A(A/I^n, K)$
and because filtered colimits are exact in the category of abelian
groups.
\end{proof}

\begin{lemma}
\label{lemma-bad-local-cohomology-vanishes}
Let $A$ be a ring and let $I \subset A$ be a finitely generated ideal.
Let $K^\bullet$ be a complex of $A$-modules such that
$f : K^\bullet \to K^\bullet$ is an isomorphism for some
$f \in I$, i.e., $K^\bullet$ is a complex of $A_f$-modules. Then
$R\Gamma_I(K^\bullet) = 0$.
\end{lemma}

\begin{proof}
Namely, in this case the cohomology modules of $R\Gamma_I(K^\bullet)$
are both $f$-power torsion and $f$ acts by automorphisms. Hence the
cohomology modules are zero and hence the object is zero.
\end{proof}

\noindent
Let $A$ be a ring and $I \subset A$ a finitely generated ideal.
By More on Algebra, Lemma \ref{more-algebra-lemma-I-power-torsion}
the category of $I$-power torsion modules is a Serre subcategory
of the category of all $A$-modules, hence there is a functor
\begin{equation}
\label{equation-compare-torsion}
D(I^\infty\text{-torsion}) \to D_{I^\infty\text{-torsion}}(A)
\end{equation}
see Derived Categories, Section \ref{derived-section-triangulated-sub}.

\begin{lemma}
\label{lemma-not-equal}
Let $A$ be a ring and let $I$ be a finitely generated ideal.
Let $M$ and $N$ be $I$-power torsion modules.
\begin{enumerate}
\item $\Hom_{D(A)}(M, N) = \Hom_{D({I^\infty\text{-torsion}})}(M, N)$,
\item $\Ext^1_{D(A)}(M, N) =
\Ext^1_{D({I^\infty\text{-torsion}})}(M, N)$,
\item $\Ext^2_{D({I^\infty\text{-torsion}})}(M, N) \to
\Ext^2_{D(A)}(M, N)$ is not surjective in general,
\item (\ref{equation-compare-torsion}) is not an equivalence in general.
\end{enumerate}
\end{lemma}

\begin{proof}
Parts (1) and (2) follow immediately from the fact that $I$-power torsion
forms a Serre subcategory of $\text{Mod}_A$. Part (4) follows from
part (3).

\medskip\noindent
For part (3) let $A$ be a ring with an element $f \in A$ such that
$A[f]$ contains a nonzero element $x$ annihilated by $f$ and
$A$ contains elements $x_n$ with $f^nx_n = x$. Such a ring $A$
exists because we can take
$$
A = \mathbf{Z}[f, x, x_n]/(fx, f^nx_n - x)
$$
Given $A$ set $I = (f)$. Then the exact sequence
$$
0 \to A[f] \to A \xrightarrow{f} A \to A/fA \to 0
$$
defines an element in $\Ext^2_A(A/fA, A[f])$. We claim this
element does not come from an element of
$\Ext^2_{D(f^\infty\text{-torsion})}(A/fA, A[f])$.
Namely, if it did, then there would be an exact sequence
$$
0 \to A[f] \to M \to N \to A/fA \to 0
$$
where $M$ and $N$ are $f$-power torsion modules defining the same
$2$ extension class. Since $A \to A$ is a complex of free modules
and since the $2$ extension classes are the same
we would be able to find a map
$$
\xymatrix{
0 \ar[r] &
A[f] \ar[r] \ar[d] &
A \ar[r] \ar[d]_\varphi &
A \ar[r] \ar[d]_\psi &
A/fA \ar[r] \ar[d] & 0 \\
0 \ar[r] &
A[f] \ar[r] &
M \ar[r] &
N \ar[r] &
A/fA \ar[r] & 0
}
$$
(some details omitted). Then we could replace $M$ by the image of
$\varphi$ and $N$ by the image of $\psi$. Then $M$ would be a cyclic
module, hence $f^n M = 0$ for some $n$. Considering $\varphi(x_{n + 1})$
we get a contradiction with the fact that $f^{n + 1}x_n = x$ is
nonzero in $A[f]$.
\end{proof}









\section{Local cohomology}
\label{section-local-cohomology}

\noindent
Let $A$ be a ring and let $I \subset A$ be a finitely generated ideal.
Set $Z = V(I) \subset \Spec(A)$. We will construct a functor
\begin{equation}
\label{equation-local-cohomology}
R\Gamma_Z : D(A) \longrightarrow D_{I^\infty\text{-torsion}}(A).
\end{equation}
which is right adjoint to the inclusion functor. For notation
see Section \ref{section-bad-local-cohomology}. The cohomology
modules of $R\Gamma_Z(K)$ are the {\it local cohomology groups
of $K$ with respect to $Z$}.
By Lemma \ref{lemma-not-equal} this functor will in general {\bf not} be
equal to $R\Gamma_I( - )$ even viewed as functors into $D(A)$.
In Section \ref{section-local-cohomology-noetherian}
we will show that if $A$ is Noetherian, then the two agree.

\medskip\noindent
We will continue the discussion of local cohomology in
the chapter on local cohomology, see
Local Cohomology, Section \ref{local-cohomology-section-introduction}.
For example, there we will show that $R\Gamma_Z$ computes cohomology
with support in $Z$ for the associated complex of quasi-coherent sheaves
on $\Spec(A)$. See Local Cohomology, Lemma
\ref{local-cohomology-lemma-local-cohomology-is-local-cohomology}.


\begin{lemma}
\label{lemma-local-cohomology-adjoint}
Let $A$ be a ring and let $I \subset A$ be a finitely generated ideal.
There exists a right adjoint $R\Gamma_Z$ (\ref{equation-local-cohomology})
to the inclusion functor $D_{I^\infty\text{-torsion}}(A) \to D(A)$.
In fact, if $I$ is generated by $f_1, \ldots, f_r \in A$, then we have
$$
R\Gamma_Z(K) =
(A \to \prod\nolimits_{i_0} A_{f_{i_0}} \to
\prod\nolimits_{i_0 < i_1} A_{f_{i_0}f_{i_1}}
\to \ldots \to A_{f_1\ldots f_r}) \otimes_A^\mathbf{L} K
$$
functorially in $K \in D(A)$.
\end{lemma}

\begin{proof}
Say $I = (f_1, \ldots, f_r)$ is an ideal.
Let $K^\bullet$ be a complex of $A$-modules.
There is a canonical map of complexes
$$
(A \to \prod\nolimits_{i_0} A_{f_{i_0}} \to
\prod\nolimits_{i_0 < i_1} A_{f_{i_0}f_{i_1}} \to
\ldots \to A_{f_1\ldots f_r}) \longrightarrow A.
$$
from the extended {\v C}ech complex to $A$.
Tensoring with $K^\bullet$, taking associated total complex,
we get a map
$$
\text{Tot}\left(
K^\bullet \otimes_A
(A \to \prod\nolimits_{i_0} A_{f_{i_0}} \to
\prod\nolimits_{i_0 < i_1} A_{f_{i_0}f_{i_1}} \to
\ldots \to A_{f_1\ldots f_r})\right)
\longrightarrow
K^\bullet
$$
in $D(A)$. We claim the cohomology modules of the complex on the left are
$I$-power torsion, i.e., the LHS is an object of
$D_{I^\infty\text{-torsion}}(A)$. Namely, we have
$$
(A \to \prod\nolimits_{i_0} A_{f_{i_0}} \to
\prod\nolimits_{i_0 < i_1} A_{f_{i_0}f_{i_1}} \to
\ldots \to A_{f_1\ldots f_r}) = \colim K(A, f_1^n, \ldots, f_r^n)
$$
by More on Algebra, Lemma
\ref{more-algebra-lemma-extended-alternating-Cech-is-colimit-koszul}.
Moreover, multiplication by $f_i^n$ on the complex
$K(A, f_1^n, \ldots, f_r^n)$ is homotopic to zero by
More on Algebra, Lemma \ref{more-algebra-lemma-homotopy-koszul}.
Since
$$
H^q\left( LHS \right) =
\colim H^q(\text{Tot}(K^\bullet \otimes_A K(A, f_1^n, \ldots, f_r^n)))
$$
we obtain our claim. On the other hand, if $K^\bullet$ is an
object of $D_{I^\infty\text{-torsion}}(A)$, then the complexes
$K^\bullet \otimes_A A_{f_{i_0} \ldots f_{i_p}}$ have vanishing
cohomology. Hence in this case the map $LHS \to K^\bullet$
is an isomorphism in $D(A)$. The construction
$$
R\Gamma_Z(K^\bullet) =
\text{Tot}\left(
K^\bullet \otimes_A
(A \to \prod\nolimits_{i_0} A_{f_{i_0}} \to
\prod\nolimits_{i_0 < i_1} A_{f_{i_0}f_{i_1}} \to
\ldots \to A_{f_1\ldots f_r})\right)
$$
is functorial in $K^\bullet$ and defines an exact functor
$D(A) \to D_{I^\infty\text{-torsion}}(A)$ between
triangulated categories. It follows formally from the
existence of the natural transformation $R\Gamma_Z \to \text{id}$
given above and the fact that this evaluates to an isomorphism
on $K^\bullet$ in the subcategory, that $R\Gamma_Z$ is the desired
right adjoint.
\end{proof}

\begin{lemma}
\label{lemma-local-cohomology-and-restriction}
Let $A \to B$ be a ring homomorphism and let $I \subset A$
be a finitely generated ideal. Set $J = IB$. Set $Z = V(I)$
and $Y = V(J)$. Then
$$
R\Gamma_Z(M_A) = R\Gamma_Y(M)_A
$$
functorially in $M \in D(B)$. Here $(-)_A$ denotes the restriction
functors $D(B) \to D(A)$ and
$D_{J^\infty\text{-torsion}}(B) \to D_{I^\infty\text{-torsion}}(A)$.
\end{lemma}

\begin{proof}
This follows from uniqueness of adjoint functors as both
$R\Gamma_Z((-)_A)$ and $R\Gamma_Y(-)_A$
are right adjoint to the functor $D_{I^\infty\text{-torsion}}(A) \to D(B)$,
$K \mapsto K \otimes_A^\mathbf{L} B$.
Alternatively, one can use the description of $R\Gamma_Z$ and $R\Gamma_Y$
in terms of alternating {\v C}ech complexes
(Lemma \ref{lemma-local-cohomology-adjoint}).
Namely, if $I = (f_1, \ldots, f_r)$ then $J$ is generated by the images
$g_1, \ldots, g_r \in B$ of $f_1, \ldots, f_r$.
Then the statement of the lemma follows from the existence of
a canonical isomorphism
\begin{align*}
& M_A \otimes_A (A \to \prod\nolimits_{i_0} A_{f_{i_0}} \to
\prod\nolimits_{i_0 < i_1} A_{f_{i_0}f_{i_1}}
\to \ldots \to A_{f_1\ldots f_r}) \\
& = 
M \otimes_B (B \to \prod\nolimits_{i_0} B_{g_{i_0}} \to
\prod\nolimits_{i_0 < i_1} B_{g_{i_0}g_{i_1}}
\to \ldots \to B_{g_1\ldots g_r})
\end{align*}
for any $B$-module $M$.
\end{proof}

\begin{lemma}
\label{lemma-torsion-change-rings}
Let $A \to B$ be a ring homomorphism and let $I \subset A$
be a finitely generated ideal. Set $J = IB$. Let $Z = V(I)$ and $Y = V(J)$.
Then
$$
R\Gamma_Z(K) \otimes_A^\mathbf{L} B = R\Gamma_Y(K \otimes_A^\mathbf{L} B)
$$
functorially in $K \in D(A)$.
\end{lemma}

\begin{proof}
Write $I = (f_1, \ldots, f_r)$. Then $J$ is generated by the images
$g_1, \ldots, g_r \in B$ of $f_1, \ldots, f_r$. Then we have
$$
(A \to \prod A_{f_{i_0}} \to \ldots \to A_{f_1\ldots f_r}) \otimes_A B =
(B \to \prod B_{g_{i_0}} \to \ldots \to B_{g_1\ldots g_r})
$$
as complexes of $B$-modules. Represent $K$ by a K-flat complex $K^\bullet$
of $A$-modules. Since the total complexes associated to
$$
K^\bullet \otimes_A
(A \to \prod A_{f_{i_0}} \to \ldots \to A_{f_1\ldots f_r}) \otimes_A B
$$
and
$$
K^\bullet \otimes_A B \otimes_B
(B \to \prod B_{g_{i_0}} \to \ldots \to B_{g_1\ldots g_r})
$$
represent the left and right hand side of the displayed formula of the
lemma (see Lemma \ref{lemma-local-cohomology-adjoint}) we conclude.
\end{proof}

\begin{lemma}
\label{lemma-local-cohomology-vanishes}
Let $A$ be a ring and let $I \subset A$ be a finitely generated ideal.
Let $K^\bullet$ be a complex of $A$-modules such that
$f : K^\bullet \to K^\bullet$ is an isomorphism for some
$f \in I$, i.e., $K^\bullet$ is a complex of $A_f$-modules. Then
$R\Gamma_Z(K^\bullet) = 0$.
\end{lemma}

\begin{proof}
Namely, in this case the cohomology modules of $R\Gamma_Z(K^\bullet)$
are both $f$-power torsion and $f$ acts by automorphisms. Hence the
cohomology modules are zero and hence the object is zero.
\end{proof}

\begin{lemma}
\label{lemma-torsion-tensor-product}
Let $A$ be a ring and let $I \subset A$ be a finitely generated ideal.
For $K, L \in D(A)$ we have
$$
R\Gamma_Z(K \otimes_A^\mathbf{L} L) =
K \otimes_A^\mathbf{L} R\Gamma_Z(L) =
R\Gamma_Z(K) \otimes_A^\mathbf{L} L =
R\Gamma_Z(K) \otimes_A^\mathbf{L} R\Gamma_Z(L)
$$
If $K$ or $L$ is in $D_{I^\infty\text{-torsion}}(A)$ then so is
$K \otimes_A^\mathbf{L} L$.
\end{lemma}

\begin{proof}
By Lemma \ref{lemma-local-cohomology-adjoint} we know that
$R\Gamma_Z$ is given by $C \otimes^\mathbf{L} -$ for some $C \in D(A)$.
Hence, for $K, L \in D(A)$ general we have
$$
R\Gamma_Z(K \otimes_A^\mathbf{L} L) =
K \otimes^\mathbf{L} L \otimes_A^\mathbf{L} C =
K \otimes_A^\mathbf{L} R\Gamma_Z(L)
$$
The other equalities follow formally from this one. This also implies
the last statement of the lemma.
\end{proof}

\begin{lemma}
\label{lemma-local-cohomology-ss}
Let $A$ be a ring and let $I, J \subset A$ be finitely generated
ideals. Set $Z = V(I)$ and $Y = V(J)$. Then $Z \cap Y = V(I + J)$
and $R\Gamma_Y \circ R\Gamma_Z = R\Gamma_{Y \cap Z}$ as functors
$D(A) \to D_{(I + J)^\infty\text{-torsion}}(A)$. For $K \in D^+(A)$
there is a spectral sequence
$$
E_2^{p, q} = H^p_Y(H^q_Z(K)) \Rightarrow H^{p + q}_{Y \cap Z}(K)
$$
as in Derived Categories, Lemma
\ref{derived-lemma-grothendieck-spectral-sequence}.
\end{lemma}

\begin{proof}
There is a bit of abuse of notation in the lemma as strictly
speaking we cannot compose $R\Gamma_Y$ and $R\Gamma_Z$. The
meaning of the statement is simply that we are composing
$R\Gamma_Z$ with the inclusion $D_{I^\infty\text{-torsion}}(A) \to D(A)$
and then with $R\Gamma_Y$. Then the equality
$R\Gamma_Y \circ R\Gamma_Z = R\Gamma_{Y \cap Z}$
follows from the fact that
$$
D_{I^\infty\text{-torsion}}(A) \to D(A) \xrightarrow{R\Gamma_Y}
D_{(I + J)^\infty\text{-torsion}}(A)
$$
is right adjoint to the inclusion
$D_{(I + J)^\infty\text{-torsion}}(A) \to D_{I^\infty\text{-torsion}}(A)$.
Alternatively one can prove the formula using
Lemma \ref{lemma-local-cohomology-adjoint}
and the fact that the tensor product of
extended {\v C}ech complexes on $f_1, \ldots, f_r$ and
$g_1, \ldots, g_m$ is the extended {\v C}ech complex on
$f_1, \ldots, f_n. g_1, \ldots, g_m$.
The final assertion follows from this and the cited lemma.
\end{proof}

\noindent
The following lemma is the analogue of
More on Algebra, Lemma
\ref{more-algebra-lemma-restriction-derived-complete-equivalence}
for complexes with torsion cohomologies.

\begin{lemma}
\label{lemma-torsion-flat-change-rings}
Let $A \to B$ be a flat ring map and let $I \subset A$ be a finitely
generated ideal such that $A/I = B/IB$. Then base change and
restriction induce quasi-inverse equivalences
$D_{I^\infty\text{-torsion}}(A) = D_{(IB)^\infty\text{-torsion}}(B)$.
\end{lemma}

\begin{proof}
More precisely the functors are $K \mapsto K \otimes_A^\mathbf{L} B$
for $K$ in $D_{I^\infty\text{-torsion}}(A)$ and $M \mapsto M_A$
for $M$ in $D_{(IB)^\infty\text{-torsion}}(B)$. The reason this works
is that $H^i(K \otimes_A^\mathbf{L} B) = H^i(K) \otimes_A B = H^i(K)$.
The first equality holds as $A \to B$ is flat and the second by
More on Algebra, Lemma \ref{more-algebra-lemma-neighbourhood-isomorphism}.
\end{proof}

\noindent
The following lemma was shown for $\Hom$ and $\Ext^1$ of modules in
More on Algebra, Lemmas \ref{more-algebra-lemma-neighbourhood-equivalence} and
\ref{more-algebra-lemma-neighbourhood-extensions}.

\begin{lemma}
\label{lemma-neighbourhood-extensions}
Let $A \to B$ be a flat ring map and let $I \subset A$ be a
finitely generated ideal such that $A/I \to B/IB$ is an isomorphism.
For $K \in D_{I^\infty\text{-torsion}}(A)$ and $L \in D(A)$
the map
$$
R\Hom_A(K, L) \longrightarrow R\Hom_B(K \otimes_A B, L \otimes_A B)
$$
is a quasi-isomorphism. In particular, if $M$, $N$ are $A$-modules and
$M$ is $I$-power torsion, then the canonical map
$$
\Ext^i_A(M, N)
\longrightarrow
\Ext^i_B(M \otimes_A B, N \otimes_A B)
$$
is an isomorphism for all $i$. 
\end{lemma}

\begin{proof}
Let $Z = V(I) \subset \Spec(A)$ and $Y = V(IB) \subset \Spec(B)$.
Since the cohomology modules of $K$ are $I$ power torsion, the
canonical map $R\Gamma_Z(L) \to L$ induces an isomorphism
$$
R\Hom_A(K, R\Gamma_Z(L)) \to R\Hom_A(K, L)
$$
in $D(A)$. Similarly, the cohomology modules of $K \otimes_A B$ are
$IB$ power torsion and we have an isomorphism
$$
R\Hom_B(K \otimes_A B, R\Gamma_Y(L \otimes_A B)) \to 
R\Hom_B(K \otimes_A B, L \otimes_A B)
$$
in $D(B)$.
By Lemma \ref{lemma-torsion-change-rings} we have
$R\Gamma_Z(L) \otimes_A B = R\Gamma_Y(L \otimes_A B)$.
Hence it suffices to show that the map
$$
R\Hom_A(K, R\Gamma_Z(L)) \to R\Hom_B(K \otimes_A B, R\Gamma_Z(L) \otimes_A B)
$$
is a quasi-isomorphism. This follows from
Lemma \ref{lemma-torsion-flat-change-rings}.
\end{proof}




\section{Local cohomology for Noetherian rings}
\label{section-local-cohomology-noetherian}

\noindent
Let $A$ be a ring and let $I \subset A$ be a finitely generated ideal.
Set $Z = V(I) \subset \Spec(A)$. Recall that (\ref{equation-compare-torsion})
is the functor
$$
D(I^\infty\text{-torsion}) \to D_{I^\infty\text{-torsion}}(A)
$$
In fact, there is a natural transformation of functors
\begin{equation}
\label{equation-compare-torsion-functors}
(\ref{equation-compare-torsion}) \circ R\Gamma_I(-)
\longrightarrow
R\Gamma_Z(-)
\end{equation}
Namely, given a complex of $A$-modules $K^\bullet$ the canonical map
$R\Gamma_I(K^\bullet) \to K^\bullet$ in $D(A)$ factors (uniquely)
through $R\Gamma_Z(K^\bullet)$ as $R\Gamma_I(K^\bullet)$ has
$I$-power torsion cohomology modules (see Lemma \ref{lemma-adjoint}).
In general this map is not an isomorphism (we've seen this in
Lemma \ref{lemma-not-equal}).

\begin{lemma}
\label{lemma-local-cohomology-noetherian}
Let $A$ be a Noetherian ring and let $I \subset A$ be an ideal.
\begin{enumerate}
\item the adjunction $R\Gamma_I(K) \to K$ is an isomorphism
for $K \in D_{I^\infty\text{-torsion}}(A)$,
\item the functor
(\ref{equation-compare-torsion})
$D(I^\infty\text{-torsion}) \to D_{I^\infty\text{-torsion}}(A)$
is an equivalence,
\item the transformation of functors
(\ref{equation-compare-torsion-functors}) is an isomorphism,
in other words $R\Gamma_I(K) = R\Gamma_Z(K)$ for $K \in D(A)$.
\end{enumerate}
\end{lemma}

\begin{proof}
A formal argument, which we omit, shows that it suffices to prove (1).

\medskip\noindent
Let $M$ be an $I$-power torsion $A$-module. Choose an embedding
$M \to J$ into an injective $A$-module. Then $J[I^\infty]$ is
an injective $A$-module, see Lemma \ref{lemma-injective-module-divide},
and we obtain an embedding $M \to J[I^\infty]$.
Thus every $I$-power torsion module has an injective resolution
$M \to J^\bullet$ with $J^n$ also $I$-power torsion. It follows
that $R\Gamma_I(M) = M$ (this is not a triviality and this is not
true in general if $A$ is not Noetherian). Next, suppose that
$K \in D_{I^\infty\text{-torsion}}^+(A)$. Then the spectral sequence
$$
R^q\Gamma_I(H^p(K)) \Rightarrow R^{p + q}\Gamma_I(K)
$$
(Derived Categories, Lemma \ref{derived-lemma-two-ss-complex-functor})
converges and above we have seen that only the terms with $q = 0$
are nonzero. Thus we see that $R\Gamma_I(K) \to K$ is an isomorphism.

\medskip\noindent
Suppose $K$ is an arbitrary object of $D_{I^\infty\text{-torsion}}(A)$.
We have
$$
R^q\Gamma_I(K) = \colim \Ext^q_A(A/I^n, K)
$$
by Lemma \ref{lemma-local-cohomology-ext}. Choose $f_1, \ldots, f_r \in A$
generating $I$. Let $K_n^\bullet = K(A, f_1^n, \ldots, f_r^n)$ be the
Koszul complex with terms in degrees $-r, \ldots, 0$. Since the
pro-objects $\{A/I^n\}$ and $\{K_n^\bullet\}$ in $D(A)$ are the same by
More on Algebra, Lemma \ref{more-algebra-lemma-sequence-Koszul-complexes},
we see that
$$
R^q\Gamma_I(K) = \colim \Ext^q_A(K_n^\bullet, K)
$$
Pick any complex $K^\bullet$ of $A$-modules representing $K$.
Since $K_n^\bullet$ is a finite complex of finite free modules we see
that
$$
\Ext^q_A(K_n, K) =
H^q(\text{Tot}((K_n^\bullet)^\vee \otimes_A K^\bullet))
$$
where $(K_n^\bullet)^\vee$ is the dual of the complex $K_n^\bullet$.
See More on Algebra, Lemma \ref{more-algebra-lemma-RHom-out-of-projective}.
As $(K_n^\bullet)^\vee$ is a complex of finite free $A$-modules sitting
in degrees $0, \ldots, r$ we see that the terms of the complex
$\text{Tot}((K_n^\bullet)^\vee \otimes_A K^\bullet)$ are the
same as the terms of the complex
$\text{Tot}((K_n^\bullet)^\vee \otimes_A \tau_{\geq q - r - 2} K^\bullet)$
in degrees $q - 1$ and higher. Hence we see that
$$
\Ext^q_A(K_n, K) = \text{Ext}^q_A(K_n, \tau_{\geq q - r - 2}K)
$$
for all $n$. It follows that
$$
R^q\Gamma_I(K) = R^q\Gamma_I(\tau_{\geq q - r - 2}K) =
H^q(\tau_{\geq q - r - 2}K) = H^q(K)
$$
Thus we see that the map $R\Gamma_I(K) \to K$ is an isomorphism.
\end{proof}

\begin{lemma}
\label{lemma-compute-local-cohomology-noetherian}
Let $A$ be a Noetherian ring and let $I = (f_1, \ldots, f_r)$ be an ideal
of $A$. Set $Z = V(I) \subset \Spec(A)$. There are canonical isomorphisms
$$
R\Gamma_I(A) \to
(A \to \prod\nolimits_{i_0} A_{f_{i_0}} \to
\prod\nolimits_{i_0 < i_1} A_{f_{i_0}f_{i_1}} \to
\ldots \to A_{f_1\ldots f_r}) \to R\Gamma_Z(A)
$$
in $D(A)$. If $M$ is an $A$-module, then we have similarly
$$
R\Gamma_I(M) \cong
(M \to \prod\nolimits_{i_0} M_{f_{i_0}} \to
\prod\nolimits_{i_0 < i_1} M_{f_{i_0}f_{i_1}} \to
\ldots \to M_{f_1\ldots f_r}) \cong R\Gamma_Z(M)
$$
in $D(A)$.
\end{lemma}

\begin{proof}
This follows from Lemma \ref{lemma-local-cohomology-noetherian}
and the computation of the functor $R\Gamma_Z$ in
Lemma \ref{lemma-local-cohomology-adjoint}.
\end{proof}

\begin{lemma}
\label{lemma-local-cohomology-change-rings}
If $A \to B$ is a homomorphism of Noetherian rings and $I \subset A$
is an ideal, then in $D(B)$ we have
$$
R\Gamma_I(A) \otimes_A^\mathbf{L} B =
R\Gamma_Z(A) \otimes_A^\mathbf{L} B =
R\Gamma_Y(B) = R\Gamma_{IB}(B)
$$
where $Y = V(IB) \subset \Spec(B)$.
\end{lemma}

\begin{proof}
Combine Lemmas \ref{lemma-compute-local-cohomology-noetherian} and
\ref{lemma-torsion-change-rings}.
\end{proof}






\section{Depth}
\label{section-depth}

\noindent
In this section we revisit the notion of depth introduced in
Algebra, Section \ref{algebra-section-depth}.

\begin{lemma}
\label{lemma-depth}
Let $A$ be a Noetherian ring, let $I \subset A$ be an ideal, and
let $M$ be a finite $A$-module such that $IM \not = M$. Then
the following integers are equal:
\begin{enumerate}
\item $\text{depth}_I(M)$,
\item the smallest integer $i$ such that $\Ext_A^i(A/I, M)$
is nonzero, and
\item the smallest integer $i$ such that $H^i_I(M)$ is nonzero.
\end{enumerate}
Moreover, we have $\Ext^i_A(N, M) = 0$ for $i < \text{depth}_I(M)$
for any finite $A$-module $N$ annihilated by a power of $I$.
\end{lemma}

\begin{proof}
We prove the equality of (1) and (2) by induction on $\text{depth}_I(M)$
which is allowed by
Algebra, Lemma \ref{algebra-lemma-depth-finite-noetherian}.

\medskip\noindent
Base case. If $\text{depth}_I(M) = 0$, then $I$ is contained in the union
of the associated primes of $M$
(Algebra, Lemma \ref{algebra-lemma-ass-zero-divisors}).
By prime avoidance (Algebra, Lemma \ref{algebra-lemma-silly})
we see that $I \subset \mathfrak p$ for some associated prime $\mathfrak p$.
Hence $\Hom_A(A/I, M)$
is nonzero. Thus equality holds in this case.

\medskip\noindent
Assume that $\text{depth}_I(M) > 0$. Let $f \in I$ be a nonzerodivisor
on $M$ such that $\text{depth}_I(M/fM) = \text{depth}_I(M) - 1$.
Consider the short exact sequence
$$
0 \to M \to M \to M/fM \to 0
$$
and the associated long exact sequence for $\Ext^*_A(A/I, -)$.
Note that $\Ext^i_A(A/I, M)$ is a finite $A/I$-module
(Algebra, Lemmas \ref{algebra-lemma-ext-noetherian} and
\ref{algebra-lemma-annihilate-ext}). Hence we obtain
$$
\Hom_A(A/I, M/fM) = \Ext^1_A(A/I, M)
$$
and short exact sequences
$$
0 \to \Ext^i_A(A/I, M) \to \text{Ext}^i_A(A/I, M/fM) \to
\Ext^{i + 1}_A(A/I, M) \to 0
$$
Thus the equality of (1) and (2) by induction.

\medskip\noindent
Observe that $\text{depth}_I(M) = \text{depth}_{I^n}(M)$ for all $n \geq 1$
for example by Algebra, Lemma \ref{algebra-lemma-regular-sequence-powers}.
Hence by the equality of (1) and (2) we see that
$\Ext^i_A(A/I^n, M) = 0$ for all $n$ and $i < \text{depth}_I(M)$.
Let $N$ be a finite $A$-module annihilated by a power of $I$.
Then we can choose a short exact sequence
$$
0 \to N' \to (A/I^n)^{\oplus m} \to N \to 0
$$
for some $n, m \geq 0$. Then
$\Hom_A(N, M) \subset \Hom_A((A/I^n)^{\oplus m}, M)$
and
$\Ext^i_A(N, M) \subset \text{Ext}^{i - 1}_A(N', M)$
for $i < \text{depth}_I(M)$. Thus a simply induction argument
shows that the final statement of the lemma holds.

\medskip\noindent
Finally, we prove that (3) is equal to (1) and (2).
We have $H^p_I(M) = \colim \Ext^p_A(A/I^n, M)$ by
Lemma \ref{lemma-local-cohomology-ext}.
Thus we see that $H^i_I(M) = 0$ for $i < \text{depth}_I(M)$.
For $i = \text{depth}_I(M)$, using the vanishing of
$\Ext_A^{i - 1}(I/I^n, M)$ we see that the map
$\Ext_A^i(A/I, M) \to H_I^i(M)$ is injective which
proves nonvanishing in the correct degree.
\end{proof}

\begin{lemma}
\label{lemma-depth-in-ses}
Let $A$ be a Noetherian ring. Let $0 \to N' \to N \to N'' \to 0$
be a short exact sequence of finite $A$-modules.
Let $I \subset A$ be an ideal.
\begin{enumerate}
\item
$\text{depth}_I(N) \geq \min\{\text{depth}_I(N'), \text{depth}_I(N'')\}$
\item
$\text{depth}_I(N'') \geq \min\{\text{depth}_I(N), \text{depth}_I(N') - 1\}$
\item
$\text{depth}_I(N') \geq \min\{\text{depth}_I(N), \text{depth}_I(N'') + 1\}$
\end{enumerate}
\end{lemma}

\begin{proof}
Assume $IN \not = N$, $IN' \not = N'$, and $IN'' \not = N''$. Then we
can use the characterization of depth using the Ext groups
$\Ext^i(A/I, N)$, see Lemma \ref{lemma-depth},
and use the long exact cohomology sequence
$$
\begin{matrix}
0
\to \Hom_A(A/I, N')
\to \Hom_A(A/I, N)
\to \Hom_A(A/I, N'')
\\
\phantom{0\ }
\to \Ext^1_A(A/I, N')
\to \Ext^1_A(A/I, N)
\to \Ext^1_A(A/I, N'')
\to \ldots
\end{matrix}
$$
from Algebra, Lemma \ref{algebra-lemma-long-exact-seq-ext}.
This argument also works if $IN = N$
because in this case $\Ext^i_A(A/I, N) = 0$ for all $i$.
Similarly in case $IN' \not = N'$ and/or $IN'' \not = N''$.
\end{proof}

\begin{lemma}
\label{lemma-depth-drops-by-one}
Let $A$ be a Noetherian ring, let $I \subset A$ be an ideal, and
let $M$ a finite $A$-module with $IM \not = M$.
\begin{enumerate}
\item If $x \in I$ is a nonzerodivisor on $M$, then
$\text{depth}_I(M/xM) = \text{depth}_I(M) - 1$.
\item Any $M$-regular sequence $x_1, \ldots, x_r$ in $I$ can be extended to an
$M$-regular sequence in $I$ of length $\text{depth}_I(M)$.
\end{enumerate}
\end{lemma}

\begin{proof}
Part (2) is a formal consequence of part (1). Let $x \in I$ be as in (1).
By the short exact sequence $0 \to M \to M \to M/xM \to 0$ and
Lemma \ref{lemma-depth-in-ses} we see that
$\text{depth}_I(M/xM) \geq \text{depth}_I(M) - 1$.
On the other hand, if $x_1, \ldots, x_r \in I$
is a regular sequence for $M/xM$, then $x, x_1, \ldots, x_r$
is a regular sequence for $M$. Hence (1) holds.
\end{proof}

\begin{lemma}
\label{lemma-depth-CM}
Let $R$ be a Noetherian local ring. If $M$ is a finite Cohen-Macaulay
$R$-module and $I \subset R$ a nontrivial ideal. Then
$$
\text{depth}_I(M) = \dim(\text{Supp}(M)) - \dim(\text{Supp}(M/IM)).
$$
\end{lemma}

\begin{proof}
We will prove this by induction on $\text{depth}_I(M)$.

\medskip\noindent
If $\text{depth}_I(M) = 0$, then $I$ is contained in one
of the associated primes $\mathfrak p$ of $M$
(Algebra, Lemma \ref{algebra-lemma-ideal-nonzerodivisor}).
Then $\mathfrak p \in \text{Supp}(M/IM)$, hence
$\dim(\text{Supp}(M/IM)) \geq \dim(R/\mathfrak p) = \dim(\text{Supp}(M))$
where equality holds by
Algebra, Lemma \ref{algebra-lemma-CM-ass-minimal-support}.
Thus the lemma holds in this case.

\medskip\noindent
If $\text{depth}_I(M) > 0$, we pick $x \in I$ which is a
nonzerodivisor on $M$. Note that $(M/xM)/I(M/xM) = M/IM$.
On the other hand we have
$\text{depth}_I(M/xM) = \text{depth}_I(M) - 1$
by Lemma \ref{lemma-depth-drops-by-one}
and $\dim(\text{Supp}(M/xM)) = \dim(\text{Supp}(M)) -  1$
by Algebra, Lemma \ref{algebra-lemma-one-equation-module}.
Thus the result by induction hypothesis.
\end{proof}

\begin{lemma}
\label{lemma-depth-flat-CM}
Let $R \to S$ be a flat local ring homomorphism of Noetherian local
rings. Denote $\mathfrak m \subset R$ the maximal ideal.
Let $I \subset S$ be an ideal.
If $S/\mathfrak mS$ is Cohen-Macaulay, then
$$
\text{depth}_I(S) \geq \dim(S/\mathfrak mS) - \dim(S/\mathfrak mS + I)
$$
\end{lemma}

\begin{proof}
By Algebra, Lemma \ref{algebra-lemma-grothendieck-regular-sequence}
any sequence in $S$ which maps to a regular sequence in $S/\mathfrak mS$
is a regular sequence in $S$. Thus it suffices to prove the lemma
in case $R$ is a field. This is a special case of Lemma \ref{lemma-depth-CM}.
\end{proof}

\begin{lemma}
\label{lemma-divide-by-torsion}
Let $A$ be a ring and let $I \subset A$ be a finitely generated ideal.
Let $M$ be an $A$-module. Let $Z = V(I)$.
Then $H^0_I(M) = H^0_Z(M)$. Let $N$ be the common value and
set $M' = M/N$. Then
\begin{enumerate}
\item $H^0_I(M') = 0$ and $H^p_I(M) = H^p_I(M')$ and $H^p_I(N) = 0$
for all $p > 0$,
\item $H^0_Z(M') = 0$ and $H^p_Z(M) = H^p_Z(M')$ and $H^p_Z(N) = 0$
for all $p > 0$.
\end{enumerate}
\end{lemma}

\begin{proof}
By definition $H^0_I(M) = M[I^\infty]$ is $I$-power torsion.
By Lemma \ref{lemma-local-cohomology-adjoint} we see that
$$
H^0_Z(M) = \Ker(M \longrightarrow M_{f_1} \times \ldots \times M_{f_r})
$$
if $I = (f_1, \ldots, f_r)$. Thus $H^0_I(M) \subset H^0_Z(M)$ and
conversely, if $x \in H^0_Z(M)$, then it is annihilated by a $f_i^{e_i}$
for some $e_i \geq 1$ hence annihilated by some power of $I$.
This proves the first equality and moreover $N$ is $I$-power torsion.
By Lemma \ref{lemma-adjoint} we see that $R\Gamma_I(N) = N$.
By Lemma \ref{lemma-local-cohomology-adjoint} we see that $R\Gamma_Z(N) = N$.
This proves the higher vanishing of $H^p_I(N)$ and $H^p_Z(N)$ in (1) and (2).
The vanishing of $H^0_I(M')$ and $H^0_Z(M')$ follow from the preceding
remarks and the fact that $M'$ is $I$-power torsion free by
More on Algebra, Lemma \ref{more-algebra-lemma-divide-by-torsion}.
The equality of higher cohomologies for $M$ and $M'$ follow
immediately from the long exact cohomology sequence.
\end{proof}









\section{Torsion versus complete modules}
\label{section-torsion-and-complete}

\noindent
Let $A$ be a ring and let $I$ be a finitely generated ideal.
In this case we can consider the derived category
$D_{I^\infty\text{-torsion}}(A)$ of complexes
with $I$-power torsion cohomology modules
(Section \ref{section-local-cohomology})
and the derived category
$D_{comp}(A, I)$ of derived complete complexes
(More on Algebra, Section \ref{more-algebra-section-derived-completion}).
In this section we show these categories are equivalent.
A more general statement can be found in
\cite{Dwyer-Greenlees}.

\begin{lemma}
\label{lemma-complete-and-local}
\begin{slogan}
Results of this nature are sometimes referred to as Greenlees-May duality.
\end{slogan}
Let $A$ be a ring and let $I$ be a finitely generated ideal.
Let $R\Gamma_Z$ be as in Lemma \ref{lemma-local-cohomology-adjoint}.
Let ${\ }^\wedge$ denote derived completion as in
More on Algebra, Lemma \ref{more-algebra-lemma-derived-completion}.
For an object $K$ in $D(A)$ we have
$$
R\Gamma_Z(K^\wedge) = R\Gamma_Z(K)
\quad\text{and}\quad
(R\Gamma_Z(K))^\wedge = K^\wedge
$$
in $D(A)$.
\end{lemma}

\begin{proof}
Choose $f_1, \ldots, f_r \in A$ generating $I$. Recall that
$$
K^\wedge = R\Hom_A\left((A \to \prod A_{f_{i_0}}
\to \prod A_{f_{i_0i_1}} \to \ldots \to A_{f_1 \ldots f_r}), K\right)
$$
by More on Algebra, Lemma \ref{more-algebra-lemma-derived-completion}.
Hence the cone $C = \text{Cone}(K \to K^\wedge)$
is given by
$$
R\Hom_A\left((\prod A_{f_{i_0}}
\to \prod A_{f_{i_0i_1}} \to \ldots \to A_{f_1 \ldots f_r}), K\right)
$$
which can be represented by a complex endowed with a finite filtration
whose successive quotients are isomorphic to
$$
R\Hom_A(A_{f_{i_0} \ldots f_{i_p}}, K), \quad p > 0
$$
These complexes vanish on applying $R\Gamma_Z$, see
Lemma \ref{lemma-local-cohomology-vanishes}. Applying $R\Gamma_Z$
to the distinguished triangle $K \to K^\wedge \to C \to K[1]$
we see that the first formula of the lemma is correct.

\medskip\noindent
Recall that
$$
R\Gamma_Z(K) =
K \otimes^\mathbf{L} (A \to \prod A_{f_{i_0}}
\to \prod A_{f_{i_0i_1}} \to \ldots \to A_{f_1 \ldots f_r})
$$
by Lemma \ref{lemma-local-cohomology-adjoint}.
Hence the cone $C = \text{Cone}(R\Gamma_Z(K) \to K)$
can be represented by a complex endowed with a finite filtration
whose successive quotients are isomorphic to
$$
K \otimes_A A_{f_{i_0} \ldots f_{i_p}}, \quad p > 0
$$
These complexes vanish on applying ${\ }^\wedge$, see
More on Algebra, Lemma \ref{more-algebra-lemma-derived-completion-vanishes}.
Applying derived completion to the distinguished triangle
$R\Gamma_Z(K) \to K \to C \to R\Gamma_Z(K)[1]$
we see that the second formula of the lemma is correct.
\end{proof}

\noindent
The following result is a special case of a very general phenomenon
concerning admissible subcategories of a triangulated category.

\begin{proposition}
\label{proposition-torsion-complete}
\begin{reference}
This is a special case of \cite[Theorem 1.1]{Porta-Liran-Yekutieli}.
\end{reference}
Let $A$ be a ring and let $I \subset A$ be a finitely generated ideal.
The functors $R\Gamma_Z$ and ${\ }^\wedge$
define quasi-inverse equivalences of categories
$$
D_{I^\infty\text{-torsion}}(A) \leftrightarrow D_{comp}(A, I)
$$
\end{proposition}

\begin{proof}
Follows immediately from Lemma \ref{lemma-complete-and-local}.
\end{proof}

\noindent
The following addendum of the proposition above makes the
correspondence on morphisms more precise.

\begin{lemma}
\label{lemma-compare-RHom}
With notation as in Lemma \ref{lemma-complete-and-local}.
For objects $K, L$ in $D(A)$ there is a canonical isomorphism
$$
R\Hom_A(K^\wedge, L^\wedge) \longrightarrow R\Hom_A(R\Gamma_Z(K), R\Gamma_Z(L))
$$
in $D(A)$.
\end{lemma}

\begin{proof}
Say $I = (f_1, \ldots, f_r)$. Denote
$C = (A \to \prod A_{f_i} \to \ldots \to A_{f_1 \ldots f_r})$ the
alternating {\v C}ech complex. Then derived completion is given by
$R\Hom_A(C, -)$ (More on Algebra, Lemma
\ref{more-algebra-lemma-derived-completion}) and local cohomology by
$C \otimes^\mathbf{L} -$ (Lemma \ref{lemma-local-cohomology-adjoint}).
Combining the isomorphism
$$
R\Hom_A(K \otimes^\mathbf{L} C, L \otimes^\mathbf{L} C) =
R\Hom_A(K, R\Hom_A(C,  L \otimes^\mathbf{L} C))
$$
(More on Algebra, Lemma \ref{more-algebra-lemma-internal-hom})
and the map
$$
L \to R\Hom_A(C,  L \otimes^\mathbf{L} C)
$$
(More on Algebra, Lemma \ref{more-algebra-lemma-internal-hom-diagonal})
we obtain a map
$$
\gamma :
R\Hom_A(K, L)
\longrightarrow
R\Hom_A(K \otimes^\mathbf{L} C, L \otimes^\mathbf{L} C)
$$
On the other hand, the right hand side is derived complete as it is
equal to
$$
R\Hom_A(C, R\Hom_A(K, L \otimes^\mathbf{L} C)).
$$
Thus $\gamma$ factors through the derived completion of
$R\Hom_A(K, L)$ by the universal property of derived completion.
However, the derived completion goes inside the $R\Hom_A$ by
More on Algebra, Lemma \ref{more-algebra-lemma-completion-RHom}
and we obtain the desired map.

\medskip\noindent
To show that the map of the lemma is an isomorphism
we may assume that $K$ and $L$ are derived complete, i.e.,
$K = K^\wedge$ and $L = L^\wedge$. In this case we are
looking at the map
$$
\gamma : R\Hom_A(K, L) \longrightarrow R\Hom_A(R\Gamma_Z(K), R\Gamma_Z(L))
$$
By Proposition \ref{proposition-torsion-complete} we know that
the cohomology groups
of the left and the right hand side coincide. In other words,
we have to check that the map $\gamma$ sends a morphism
$\alpha : K \to L$ in $D(A)$ to the morphism
$R\Gamma_Z(\alpha) : R\Gamma_Z(K) \to R\Gamma_Z(L)$.
We omit the verification (hint: note that $R\Gamma_Z(\alpha)$
is just the map
$\alpha \otimes \text{id}_C :
K \otimes^\mathbf{L} C
\to
L \otimes^\mathbf{L} C$ which is almost the same as the
construction of the map in
More on Algebra, Lemma \ref{more-algebra-lemma-internal-hom-diagonal}).
\end{proof}

\begin{lemma}
\label{lemma-completion-local}
Let $I$ and $J$ be ideals in a Noetherian ring $A$. Let $M$ be a finite
$A$-module. Set $Z =V(J)$. Consider the derived $I$-adic completion
$R\Gamma_Z(M)^\wedge$ of local cohomology. Then
\begin{enumerate}
\item we have $R\Gamma_Z(M)^\wedge = R\lim R\Gamma_Z(M/I^nM)$, and
\item there are short exact sequences
$$
0 \to R^1\lim H^{i - 1}_Z(M/I^nM) \to H^i(R\Gamma_Z(M)^\wedge) \to
\lim H^i_Z(M/I^nM) \to 0
$$
\end{enumerate}
In particular $R\Gamma_Z(M)^\wedge$ has vanishing cohomology
in negative degrees.
\end{lemma}

\begin{proof}
Suppose that $J = (g_1, \ldots, g_m)$.
Then $R\Gamma_Z(M)$ is computed by the complex
$$
M \to \prod M_{g_{j_0}} \to \prod M_{g_{j_0}g_{j_1}} \to
\ldots \to M_{g_1g_2\ldots g_m}
$$
by Lemma \ref{lemma-local-cohomology-adjoint}.
By More on Algebra, Lemma
\ref{more-algebra-lemma-when-derived-completion-is-completion}
the derived $I$-adic completion of
this complex is given by the complex
$$
\lim M/I^nM \to \prod \lim (M/I^nM)_{g_{j_0}} \to
\ldots \to \lim (M/I^nM)_{g_1g_2\ldots g_m}
$$
of usual completions. Since $R\Gamma_Z(M/I^nM)$ is computed by
the complex $ M/I^nM \to \prod (M/I^nM)_{g_{j_0}} \to
\ldots \to (M/I^nM)_{g_1g_2\ldots g_m}$ and since the
transition maps between these complexes are surjective,
we conclude that (1) holds by
More on Algebra, Lemma \ref{more-algebra-lemma-compute-Rlim-modules}.
Part (2) then follows from More on Algebra, Lemma
\ref{more-algebra-lemma-break-long-exact-sequence-modules}.
\end{proof}

\begin{lemma}
\label{lemma-completion-local-H0}
With notation and hypotheses as in Lemma \ref{lemma-completion-local}
assume $A$ is $I$-adically complete. Then
$$
H^0(R\Gamma_Z(M)^\wedge) = \colim H^0_{V(J')}(M)
$$
where the filtered colimit is over $J' \subset J$ such that
$V(J') \cap V(I) = V(J) \cap V(I)$.
\end{lemma}

\begin{proof}
Since $M$ is a finite $A$-module, we have that $M$ is $I$-adically complete.
The proof of Lemma \ref{lemma-completion-local} shows that
$$
H^0(R\Gamma_Z(M)^\wedge) =
\Ker(M^\wedge \to \prod M_{g_j}^\wedge) =
\Ker(M \to \prod M_{g_j}^\wedge)
$$
where on the right hand side we have usual $I$-adic completion.
The kernel $K_j$ of $M_{g_j} \to M_{g_j}^\wedge$ is $\bigcap I^n M_{g_j}$.
By Algebra, Lemma \ref{algebra-lemma-intersection-powers-ideal-module}
for every $\mathfrak p \in V(IA_{g_j})$ we find an
$f \in A_{g_j}$, $f \not \in \mathfrak p$ such that $(K_j)_f = 0$.

\medskip\noindent
Let $s \in H^0(R\Gamma_Z(M)^\wedge)$.
By the above we may think of $s$ as an element of $M$.
The support $Z'$ of $s$ intersected with $D(g_j)$ is disjoint from
$D(g_j) \cap V(I)$ by the arguments above.
Thus $Z'$ is a closed subset of $\Spec(A)$ with $Z' \cap V(I) \subset V(J)$.
Then $Z' \cup V(J) = V(J')$ for some ideal $J' \subset J$ with
$V(J') \cap V(I) \subset V(J)$ and we have $s \in H^0_{V(J')}(M)$.
Conversely, any $s \in H^0_{V(J')}(M)$ with $J' \subset J$ and
$V(J') \cap V(I) \subset V(J)$ maps to zero in $M_{g_j}^\wedge$ for all $j$.
This proves the lemma.
\end{proof}









\section{Trivial duality for a ring map}
\label{section-trivial}

\noindent
Let $A \to B$ be a ring homomorphism. Consider the functor
$$
\Hom_A(B, -) : \text{Mod}_A \longrightarrow \text{Mod}_B,\quad
M \longmapsto \Hom_A(B, M)
$$
This functor is left exact and has a derived extension
$R\Hom(B, -) : D(A) \to D(B)$.

\begin{lemma}
\label{lemma-right-adjoint}
Let $A \to B$ be a ring homomorphism. The functor $R\Hom(B, -)$
constructed above is right adjoint to the restriction functor
$D(B) \to D(A)$.
\end{lemma}

\begin{proof}
This is a consequence of the fact that restriction and $\Hom_A(B, -)$ are
adjoint functors by Algebra, Lemma \ref{algebra-lemma-adjoint-hom-restrict}.
See Derived Categories, Lemma \ref{derived-lemma-derived-adjoint-functors}.
\end{proof}

\begin{lemma}
\label{lemma-composition-right-adjoints}
Let $A \to B \to C$ be ring maps. Then
$R\Hom(C, -) \circ R\Hom(B, -) : D(A) \to D(C)$
is the functor $R\Hom(C, -) : D(A) \to D(C)$.
\end{lemma}

\begin{proof}
Follows from uniqueness of right adjoints and Lemma \ref{lemma-right-adjoint}.
\end{proof}

\begin{lemma}
\label{lemma-RHom-ext}
Let $\varphi : A \to B$ be a ring homomorphism. For $K$ in $D(A)$ we have
$$
\varphi_*R\Hom(B, K) = R\Hom_A(B, K)
$$
where $\varphi_* : D(B) \to D(A)$ is restriction. In particular
$R^q\Hom(B, K) = \Ext_A^q(B, K)$.
\end{lemma}

\begin{proof}
Choose a K-injective complex $I^\bullet$ representing $K$.
Then $R\Hom(B, K)$ is represented by the complex $\Hom_A(B, I^\bullet)$
of $B$-modules. Since this complex, as a complex of $A$-modules,
represents $R\Hom_A(B, K)$ we see that the lemma is true.
\end{proof}

\noindent
Let $A$ be a Noetherian ring. We will denote
$$
D_{\textit{Coh}}(A) \subset D(A)
$$
the full subcategory consisting of those objects $K$ of $D(A)$
whose cohomology modules are all finite $A$-modules. This makes sense
by Derived Categories, Section \ref{derived-section-triangulated-sub}
because as $A$ is Noetherian, the subcategory of finite $A$-modules
is a Serre subcategory of $\text{Mod}_A$.

\begin{lemma}
\label{lemma-exact-support-coherent}
With notation as above, assume $A \to B$ is a finite ring map of
Noetherian rings. Then $R\Hom(B, -)$ maps
$D^+_{\textit{Coh}}(A)$ into $D^+_{\textit{Coh}}(B)$.
\end{lemma}

\begin{proof}
We have to show: if $K \in D^+(A)$ has finite cohomology modules, then the
complex $R\Hom(B, K)$ has finite cohomology modules too.
This follows for example from Lemma \ref{lemma-RHom-ext}
if we can show the ext modules $\Ext^i_A(B, K)$
are finite $A$-modules. Since $K$ is bounded below there is a
convergent spectral sequence
$$
\Ext^p_A(B, H^q(K)) \Rightarrow \text{Ext}^{p + q}_A(B, K)
$$
This finishes the proof as the modules $\Ext^p_A(B, H^q(K))$
are finite by
Algebra, Lemma \ref{algebra-lemma-ext-noetherian}.
\end{proof}

\begin{remark}
\label{remark-exact-support}
Let $A$ be a ring and let $I \subset A$ be an ideal. Set $B = A/I$.
In this case the functor $\Hom_A(B, -)$ is equal to the functor
$$
\text{Mod}_A \longrightarrow \text{Mod}_B,\quad M \longmapsto M[I]
$$
which sends $M$ to the submodule of $I$-torsion.
\end{remark}

\begin{situation}
\label{situation-resolution}
Let $R \to A$ be a ring map.
We will give an alternative construction of $R\Hom(A, -)$
which will stand us in good stead later in this chapter.
Namely, suppose we have a differential graded algebra $(E, d)$
over $R$ and a quasi-isomorphism $E \to A$ where we view $A$
as a differential graded algebra over $R$ with zero differential.
Then we have commutative diagrams
$$
\vcenter{
\xymatrix{
D(E, \text{d}) \ar[rd] & &  D(A) \ar[ll] \ar[ld] \\
& D(R)
}
}
\quad\text{and}\quad
\vcenter{
\xymatrix{
D(E, \text{d}) \ar[rr]_{- \otimes_E^\mathbf{L} A} & &  D(A) \\
& D(R) \ar[lu]^{- \otimes_R^\mathbf{L} E} \ar[ru]_{- \otimes_R^\mathbf{L} A}
}
}
$$
where the horizontal arrows are equivalences of categories
(Differential Graded Algebra, Lemma \ref{dga-lemma-qis-equivalence}).
It is clear that the first diagram commutes.
The second diagram commutes because the first one does
and our functors are their left adjoints
(Differential Graded Algebra, Example \ref{dga-example-map-hom-tensor})
or because we have $E \otimes^\mathbf{L}_E A = E \otimes_E A$
and we can use 
Differential Graded Algebra, Lemma
\ref{dga-lemma-compose-tensor-functors-general}.
\end{situation}

\begin{lemma}
\label{lemma-RHom-dga}
In Situation \ref{situation-resolution} the functor $R\Hom(A, -)$
is equal to the composition of
$R\Hom(E, -) : D(R) \to D(E, \text{d})$
and the equivalence $- \otimes^\mathbf{L}_E A : D(E, \text{d}) \to D(A)$.
\end{lemma}

\begin{proof}
This is true because $R\Hom(E, -)$ is the right adjoint
to $- \otimes^\mathbf{L}_R E$, see
Differential Graded Algebra, Lemma \ref{dga-lemma-tensor-hom-adjoint}.
Hence this functor plays the same role as the functor
$R\Hom(A, -)$ for the map $R \to A$ (Lemma \ref{lemma-right-adjoint}),
whence these functors must correspond via the equivalence
$- \otimes^\mathbf{L}_E A : D(E, \text{d}) \to D(A)$.
\end{proof}

\begin{lemma}
\label{lemma-RHom-is-tensor}
In Situation \ref{situation-resolution} assume that
\begin{enumerate}
\item $E$ viewed as an object of $D(R)$ is compact, and
\item $N = \Hom^\bullet_R(E^\bullet, R)$ computes $R\Hom(E, R)$.
\end{enumerate}
Then $R\Hom(E, -) : D(R) \to D(E)$ is isomorphic to
$K \mapsto K \otimes_R^\mathbf{L} N$.
\end{lemma}

\begin{proof}
Special case of Differential Graded Algebra, Lemma
\ref{dga-lemma-RHom-is-tensor}.
\end{proof}

\begin{lemma}
\label{lemma-RHom-is-tensor-special}
In Situation \ref{situation-resolution} assume $A$ is a perfect $R$-module.
Then
$$
R\Hom(A, -) : D(R) \to D(A)
$$
is given by $K \mapsto K \otimes_R^\mathbf{L} M$
where $M = R\Hom(A, R) \in D(A)$.
\end{lemma}

\begin{proof}
We apply Divided Power Algebra, Lemma
\ref{dpa-lemma-tate-resoluton-pseudo-coherent-ring-map}
to choose a Tate resolution $(E, \text{d})$ of $A$ over $R$.
Note that $E^i = 0$ for $i > 0$, $E^0 = R[x_1, \ldots, x_n]$
is a polynomial algebra, and $E^i$ is a finite free $E^0$-module
for $i < 0$. It follows that $E$ viewed as a complex of $R$-modules
is a bounded above complex of free $R$-modules.
We check the assumptions of Lemma \ref{lemma-RHom-is-tensor}.
The first holds because $A$ is perfect
(hence compact by More on Algebra, Proposition
\ref{more-algebra-proposition-perfect-is-compact})
and the second by
More on Algebra, Lemma \ref{more-algebra-lemma-RHom-out-of-projective}.
From the lemma conclude that $K \mapsto R\Hom(E, K)$ is
isomorphic to $K \mapsto K \otimes_R^\mathbf{L} N$ for
some differential graded $E$-module $N$. Observe that
$$
(R \otimes_R E) \otimes_E^\mathbf{L} A = R \otimes_E E \otimes_E A
$$
in $D(A)$. Hence by Differential Graded Algebra, Lemma
\ref{dga-lemma-compose-tensor-functors-general-algebra}
we conclude that the composition of
$- \otimes_R^\mathbf{L} N$ and $- \otimes_R^\mathbf{L} A$
is of the form $- \otimes_R M$ for some $M \in D(A)$.
To finish the proof we apply Lemma \ref{lemma-RHom-dga}.
\end{proof}

\begin{lemma}
\label{lemma-compute-for-effective-Cartier-algebraic}
Let $R \to A$ be a surjective ring map whose kernel $I$
is an invertible $R$-module. The functor
$R\Hom(A, -) : D(R) \to D(A)$
is isomorphic to $K \mapsto K \otimes_R^\mathbf{L} N[-1]$
where $N$ is inverse of the invertible $A$-module $I \otimes_R A$.
\end{lemma}

\begin{proof}
Since $A$ has the finite projective resolution
$$
0 \to I \to R \to A \to 0
$$
we see that $A$ is a perfect $R$-module. By
Lemma \ref{lemma-RHom-is-tensor-special} it suffices
to prove that $R\Hom(A, R)$ is represented by $N[-1]$ in $D(A)$.
This means $R\Hom(A, R)$ has a unique nonzero
cohomology module, namely $N$ in degree $1$. As
$\text{Mod}_A \to \text{Mod}_R$ is fully faithful it suffice to prove
this after applying the restriction functor $i_* : D(A) \to D(R)$.
By Lemma \ref{lemma-RHom-ext} we have
$$
i_*R\Hom(A, R) = R\Hom_R(A, R)
$$
Using the finite projective resolution above we find that the latter
is represented by the complex $R \to I^{\otimes -1}$ with $R$
in degree $0$. The map $R \to I^{\otimes -1}$ is injective
and the cokernel is $N$.
\end{proof}





\section{Base change for trivial duality}
\label{section-base-change-trivial-duality}

\noindent
In this section we consider a cocartesian square of rings
$$
\xymatrix{
A \ar[r]_\alpha & A' \\
R \ar[u]^\varphi \ar[r]^\rho & R' \ar[u]_{\varphi'}
}
$$
In other words, we have $A' = A \otimes_R R'$. If $A$ and $R'$
are {\bf tor independent over} $R$ then there is a canonical base change map
\begin{equation}
\label{equation-base-change}
R\Hom(A, K) \otimes_A^\mathbf{L} A'
\longrightarrow
R\Hom(A', K \otimes_R^\mathbf{L} R')
\end{equation}
in $D(A')$ functorial for $K$ in $D(R)$. Namely, by the adjointness
of Lemma \ref{lemma-right-adjoint} such an arrow is the same thing as a map
$$
\varphi'_*\left(R\Hom(A, K) \otimes_A^\mathbf{L} A'\right)
\longrightarrow
K \otimes_R^\mathbf{L} R'
$$
in $D(R')$ where $\varphi'_* : D(A') \to D(R')$ is the restriction functor.
We may apply
More on Algebra, Lemma \ref{more-algebra-lemma-base-change-comparison}
to the left hand side to get that this is the same thing as a map
$$
\varphi_*(R\Hom(A, K)) \otimes_R^\mathbf{L} R'
\longrightarrow
K \otimes_R^\mathbf{L} R'
$$
in $D(R')$ where $\varphi_* : D(A) \to D(R)$ is the restriction functor.
For this we can choose $can \otimes^\mathbf{L} \text{id}_{R'}$
where $can : \varphi_*(R\Hom(A, K)) \to K$ is the
counit of the adjunction between $R\Hom(A, -)$ and $\varphi_*$.

\begin{lemma}
\label{lemma-check-base-change-is-iso}
In the situation above, the map (\ref{equation-base-change})
is an isomorphism if and only if the map
$$
R\Hom_R(A, K) \otimes_R^\mathbf{L} R'
\longrightarrow
R\Hom_R(A, K \otimes_R^\mathbf{L} R')
$$
of More on Algebra, Lemma
\ref{more-algebra-lemma-internal-hom-diagonal-better} is an isomorphism.
\end{lemma}

\begin{proof}
To see that the map is an isomorphism, it suffices to prove it
is an isomorphism after applying $\varphi'_*$.
Applying the functor $\varphi'_*$ to (\ref{equation-base-change})
and using that $A' = A \otimes_R^\mathbf{L} R'$
we obtain the base change map
$R\Hom_R(A, K) \otimes_R^\mathbf{L} R' \to
R\Hom_{R'}(A \otimes_R^\mathbf{L} R', K \otimes_R^\mathbf{L} R')$
for derived hom of
More on Algebra, Equation (\ref{more-algebra-equation-base-change-RHom}).
Unwinding the left and right hand side exactly as in the proof of
More on Algebra, Lemma \ref{more-algebra-lemma-base-change-RHom}
and in particular using
More on Algebra, Lemma \ref{more-algebra-lemma-upgrade-adjoint-tensor-RHom}
gives the desired result.
\end{proof}

\begin{lemma}
\label{lemma-flat-bc-surjection}
Let $R \to A$ and $R \to R'$ be ring maps and $A' = A \otimes_R R'$.
Assume
\begin{enumerate}
\item $A$ is pseudo-coherent as an $R$-module,
\item $R'$ has finite tor dimension as an $R$-module (for example
$R \to R'$ is flat),
\item $A$ and $R'$ are tor independent over $R$.
\end{enumerate}
Then (\ref{equation-base-change}) is an isomorphism for $K \in D^+(R)$.
\end{lemma}

\begin{proof}
Follows from Lemma \ref{lemma-check-base-change-is-iso} and
More on Algebra, Lemma
\ref{more-algebra-lemma-internal-hom-evaluate-tensor-isomorphism} part (4).
\end{proof}

\begin{lemma}
\label{lemma-bc-surjection}
Let $R \to A$ and $R \to R'$ be ring maps and $A' = A \otimes_R R'$.
Assume
\begin{enumerate}
\item $A$ is perfect as an $R$-module,
\item $A$ and $R'$ are tor independent over $R$.
\end{enumerate}
Then (\ref{equation-base-change}) is an isomorphism for all $K \in D(R)$.
\end{lemma}

\begin{proof}
Follows from Lemma \ref{lemma-check-base-change-is-iso} and
More on Algebra, Lemma
\ref{more-algebra-lemma-internal-hom-evaluate-tensor-isomorphism} part (1).
\end{proof}







\section{Dualizing complexes}
\label{section-dualizing}

\noindent
In this section we define dualizing complexes for Noetherian rings.

\begin{definition}
\label{definition-dualizing}
Let $A$ be a Noetherian ring. A {\it dualizing complex} is a
complex of $A$-modules $\omega_A^\bullet$ such that
\begin{enumerate}
\item $\omega_A^\bullet$ has finite injective dimension,
\item $H^i(\omega_A^\bullet)$ is a finite $A$-module for all $i$, and
\item $A \to R\Hom_A(\omega_A^\bullet, \omega_A^\bullet)$
is a quasi-isomorphism.
\end{enumerate}
\end{definition}

\noindent
This definition takes some time getting used to. It is perhaps a good
idea to prove some of the following lemmas yourself without reading
the proofs.

\begin{lemma}
\label{lemma-finite-ext-into-bounded-injective}
Let $A$ be a Noetherian ring. Let $K, L \in D_{\textit{Coh}}(A)$
and assume $L$ has finite injective dimension. Then
$R\Hom_A(K, L)$ is in $D_{\textit{Coh}}(A)$.
\end{lemma}

\begin{proof}
Pick an integer $n$ and consider the distinguished triangle
$$
\tau_{\leq n}K \to K \to \tau_{\geq n + 1}K \to \tau_{\leq n}K[1]
$$
see Derived Categories, Remark
\ref{derived-remark-truncation-distinguished-triangle}.
Since $L$ has finite injective dimension we see
that $R\Hom_A(\tau_{\geq n + 1}K, L)$ has vanishing
cohomology in degrees $\geq c - n$ for some constant $c$.
Hence, given $i$, we see that
$\Ext^i_A(K, L) \to \Ext^i_A(\tau_{\leq n}K, L)$
is an isomorphism for some $n \gg - i$. By
Derived Categories of Schemes, Lemma \ref{perfect-lemma-coherent-internal-hom}
applied to $\tau_{\leq n}K$ and $L$
we see conclude that $\Ext^i_A(K, L)$ is
a finite $A$-module for all $i$. Hence $R\Hom_A(K, L)$
is indeed an object of $D_{\textit{Coh}}(A)$.
\end{proof}

\begin{lemma}
\label{lemma-dualizing}
Let $A$ be a Noetherian ring. If $\omega_A^\bullet$ is a dualizing
complex, then the functor
$$
D : K \longmapsto R\Hom_A(K, \omega_A^\bullet)
$$
is an anti-equivalence $D_{\textit{Coh}}(A) \to D_{\textit{Coh}}(A)$
which exchanges $D^+_{\textit{Coh}}(A)$ and $D^-_{\textit{Coh}}(A)$
and induces an anti-equivalence
$D^b_{\textit{Coh}}(A) \to D^b_{\textit{Coh}}(A)$.
Moreover $D \circ D$ is isomorphic to the identity functor.
\end{lemma}

\begin{proof}
Let $K$ be an object of $D_{\textit{Coh}}(A)$. From
Lemma \ref{lemma-finite-ext-into-bounded-injective}
we see $R\Hom_A(K, \omega_A^\bullet)$ is an object of $D_{\textit{Coh}}(A)$.
By More on Algebra, Lemma
\ref{more-algebra-lemma-internal-hom-evaluate-isomorphism-technical}
and the assumptions on the dualizing complex
we obtain a canonical isomorphism
$$
K = R\Hom_A(\omega_A^\bullet, \omega_A^\bullet) \otimes_A^\mathbf{L} K
\longrightarrow
R\Hom_A(R\Hom_A(K, \omega_A^\bullet), \omega_A^\bullet)
$$
Thus our functor has a quasi-inverse and the proof is complete.
\end{proof}

\noindent
Let $R$ be a ring. Recall that an object $L$ of $D(R)$ is
{\it invertible} if it is an invertible object for the
symmetric monoidal structure on $D(R)$ given by derived
tensor product. In
More on Algebra, Lemma \ref{more-algebra-lemma-invertible-derived}
we have seen this means $L$ is perfect, $L = \bigoplus H^n(L)[-n]$,
this is a finite sum, each $H^n(L)$ is finite projective,
and there is an open covering $\Spec(R) = \bigcup D(f_i)$ such that
$L \otimes_R R_{f_i} \cong R_{f_i}[-n_i]$ for some integers $n_i$.

\begin{lemma}
\label{lemma-equivalence-comes-from-invertible}
Let $A$ be a Noetherian ring. Let
$F : D^b_{\textit{Coh}}(A) \to D^b_{\textit{Coh}}(A)$ be an $A$-linear
equivalence of categories. Then $F(A)$ is an invertible object of $D(A)$.
\end{lemma}

\begin{proof}
Let $\mathfrak m \subset A$ be a maximal ideal with residue field $\kappa$.
Consider the object $F(\kappa)$. Since
$\kappa = \Hom_{D(A)}(\kappa, \kappa)$ we find that all
cohomology groups of $F(\kappa)$ are annihilated by $\mathfrak m$.
We also see that
$$
\Ext^i_A(\kappa, \kappa) = \text{Ext}^i_A(F(\kappa), F(\kappa))
= \Hom_{D(A)}(F(\kappa), F(\kappa)[i])
$$
is zero for $i < 0$. Say $H^a(F(\kappa)) \not = 0$ and
$H^b(F(\kappa)) \not = 0$ with $a$ minimal and $b$ maximal
(so in particular $a \leq b$). Then there is a nonzero map
$$
F(\kappa) \to H^b(F(\kappa))[-b] \to H^a(F(\kappa))[-b]
\to F(\kappa)[a - b]
$$
in $D(A)$ (nonzero because it induces a nonzero map on cohomology).
This proves that $b = a$. We conclude that $F(\kappa) = \kappa[-a]$.

\medskip\noindent
Let $G$ be a quasi-inverse to our functor $F$. Arguing as above
we find an integer $b$ such that $G(\kappa) = \kappa[-b]$.
On composing we find $a + b = 0$. Let $E$ be a finite $A$-module
wich is annihilated by a power of $\mathfrak m$. Arguing by
induction on the length of $E$ we find that $G(E) = E'[-b]$
for some finite $A$-module $E'$ annihilated by a power of
$\mathfrak m$. Then $E[-a] = F(E')$.
Next, we consider the groups
$$
\Ext^i_A(A, E') = \text{Ext}^i_A(F(A), F(E')) =
\Hom_{D(A)}(F(A), E[-a + i])
$$
The left hand side is nonzero if and only if $i = 0$ and then
we get $E'$. Applying this with $E = E' = \kappa$ and using Nakayama's
lemma this implies that $H^j(F(A))_\mathfrak m$ is zero for $j > a$ and
generated by $1$ element for $j = a$. On the other hand, if
$H^j(F(A))_\mathfrak m$ is not zero for some $j < a$, then
there is a map $F(A) \to E[-a + i]$ for some $i < 0$ and some
$E$ (More on Algebra, Lemma \ref{more-algebra-lemma-detect-cohomology})
which is a contradiction.
Thus we see that $F(A)_\mathfrak m = M[-a]$
for some $A_\mathfrak m$-module $M$ generated by $1$ element.
However, since
$$
A_\mathfrak m = \Hom_{D(A)}(A, A)_\mathfrak m =
\Hom_{D(A)}(F(A), F(A))_\mathfrak m = \Hom_{A_\mathfrak m}(M, M)
$$
we see that $M \cong A_\mathfrak m$. We conclude that there exists
an element $f \in A$, $f \not \in \mathfrak m$ such that
$F(A)_f$ is isomorphic to $A_f[-a]$. This finishes the proof.
\end{proof}

\begin{lemma}
\label{lemma-dualizing-unique}
Let $A$ be a Noetherian ring. If $\omega_A^\bullet$ and
$(\omega'_A)^\bullet$ are dualizing complexes, then
$(\omega'_A)^\bullet$ is quasi-isomorphic to
$\omega_A^\bullet \otimes_A^\mathbf{L} L$
for some invertible object $L$ of $D(A)$.
\end{lemma}

\begin{proof}
By Lemmas \ref{lemma-dualizing} and
\ref{lemma-equivalence-comes-from-invertible} the functor
$K \mapsto R\Hom_A(R\Hom_A(K, \omega_A^\bullet), (\omega_A')^\bullet)$
maps $A$ to an invertible object $L$. In other words, there is
an isomorphism
$$
L \longrightarrow R\Hom_A(\omega_A^\bullet, (\omega_A')^\bullet)
$$
Since $L$ has finite tor dimension, this means that we can apply
More on Algebra, Lemma
\ref{more-algebra-lemma-internal-hom-evaluate-isomorphism-technical}
to see that
$$
R\Hom_A(\omega_A^\bullet, (\omega'_A)^\bullet) \otimes_A^\mathbf{L} K
\longrightarrow
R\Hom_A(R\Hom_A(K, \omega_A^\bullet), (\omega_A')^\bullet)
$$
is an isomorphism for $K$ in $D^b_{\textit{Coh}}(A)$.
In particular, setting $K = \omega_A^\bullet$ finishes the proof.
\end{proof}

\begin{lemma}
\label{lemma-dualizing-localize}
Let $A$ be a Noetherian ring. Let $B = S^{-1}A$ be a localization.
If $\omega_A^\bullet$ is a dualizing
complex, then $\omega_A^\bullet \otimes_A B$ is a dualizing
complex for $B$.
\end{lemma}

\begin{proof}
Let $\omega_A^\bullet \to I^\bullet$ be a quasi-isomorphism
with $I^\bullet$ a bounded complex of injectives.
Then $S^{-1}I^\bullet$ is a bounded complex of injective
$B = S^{-1}A$-modules (Lemma \ref{lemma-localization-injective-modules})
representing $\omega_A^\bullet \otimes_A B$.
Thus $\omega_A^\bullet \otimes_A B$ has finite injective dimension.
Since $H^i(\omega_A^\bullet \otimes_A B) = H^i(\omega_A^\bullet) \otimes_A B$
by flatness of $A \to B$ we see that $\omega_A^\bullet \otimes_A B$
has finite cohomology modules. Finally, the map
$$
B \longrightarrow
R\Hom_A(\omega_A^\bullet \otimes_A B, \omega_A^\bullet \otimes_A B)
$$
is a quasi-isomorphism as formation of internal hom commutes with
flat base change in this case, see
More on Algebra, Lemma \ref{more-algebra-lemma-base-change-RHom}.
\end{proof}

\begin{lemma}
\label{lemma-dualizing-glue}
Let $A$ be a Noetherian ring. Let $f_1, \ldots, f_n \in A$
generate the unit ideal. If $\omega_A^\bullet$ is a complex
of $A$-modules such that $(\omega_A^\bullet)_{f_i}$ is a dualizing
complex for $A_{f_i}$ for all $i$, then $\omega_A^\bullet$ is a dualizing
complex for $A$.
\end{lemma}

\begin{proof}
Consider the double complex
$$
\prod\nolimits_{i_0} (\omega_A^\bullet)_{f_{i_0}}
\to
\prod\nolimits_{i_0 < i_1} (\omega_A^\bullet)_{f_{i_0}f_{i_1}}
\to \ldots
$$
The associated total complex is quasi-isomorphic to $\omega_A^\bullet$
for example by Descent, Remark \ref{descent-remark-standard-covering}
or by
Derived Categories of Schemes, Lemma
\ref{perfect-lemma-alternating-cech-complex-complex-computes-cohomology}.
By assumption the complexes $(\omega_A^\bullet)_{f_i}$ have
finite injective dimension as complexes of $A_{f_i}$-modules.
This implies that each of the complexes
$(\omega_A^\bullet)_{f_{i_0} \ldots f_{i_p}}$, $p > 0$ has
finite injective dimension over $A_{f_{i_0} \ldots f_{i_p}}$,
see Lemma \ref{lemma-localization-injective-modules}.
This in turn implies that each of the complexes
$(\omega_A^\bullet)_{f_{i_0} \ldots f_{i_p}}$, $p > 0$ has
finite injective dimension over $A$, see
Lemma \ref{lemma-injective-flat}. Hence $\omega_A^\bullet$
has finite injective dimension as a complex of $A$-modules
(as it can be represented by a complex endowed with
a finite filtration whose graded parts have finite injective
dimension). Since $H^n(\omega_A^\bullet)_{f_i}$ is a finite
$A_{f_i}$ module for each $i$ we see that $H^i(\omega_A^\bullet)$
is a finite $A$-module, see Algebra, Lemma \ref{algebra-lemma-cover}.
Finally, the (derived) base change of the map
$A \to R\Hom_A(\omega_A^\bullet, \omega_A^\bullet)$ to $A_{f_i}$
is the map
$A_{f_i} \to R\Hom_A((\omega_A^\bullet)_{f_i}, (\omega_A^\bullet)_{f_i})$ by
More on Algebra, Lemma \ref{more-algebra-lemma-base-change-RHom}.
Hence we deduce that
$A \to R\Hom_A(\omega_A^\bullet, \omega_A^\bullet)$
is an isomorphism and the proof is complete.
\end{proof}

\begin{lemma}
\label{lemma-dualizing-finite}
Let $A \to B$ be a finite ring map of Noetherian rings.
Let $\omega_A^\bullet$ be a dualizing complex.
Then $R\Hom(B, \omega_A^\bullet)$ is a dualizing complex for $B$.
\end{lemma}

\begin{proof}
Let $\omega_A^\bullet \to I^\bullet$ be a quasi-isomorphism
with $I^\bullet$ a bounded complex of injectives.
Then $\Hom_A(B, I^\bullet)$ is a bounded complex of injective
$B$-modules (Lemma \ref{lemma-hom-injective}) representing
$R\Hom(B, \omega_A^\bullet)$.
Thus $R\Hom(B, \omega_A^\bullet)$ has finite injective dimension.
By Lemma \ref{lemma-exact-support-coherent} it is an object of
$D_{\textit{Coh}}(B)$. Finally, we compute
$$
\Hom_{D(B)}(R\Hom(B, \omega_A^\bullet), R\Hom(B, \omega_A^\bullet)) =
\Hom_{D(A)}(R\Hom(B, \omega_A^\bullet), \omega_A^\bullet) = B
$$
and for $n \not = 0$ we compute
$$
\Hom_{D(B)}(R\Hom(B, \omega_A^\bullet), R\Hom(B, \omega_A^\bullet)[n]) =
\Hom_{D(A)}(R\Hom(B, \omega_A^\bullet), \omega_A^\bullet[n]) = 0
$$
which proves the last property of a dualizing complex.
In the displayed equations, the first
equality holds by Lemma \ref{lemma-right-adjoint}
and the second equality holds by Lemma \ref{lemma-dualizing}.
\end{proof}

\begin{lemma}
\label{lemma-dualizing-quotient}
Let $A \to B$ be a surjective homomorphism of Noetherian rings.
Let $\omega_A^\bullet$ be a dualizing complex.
Then $R\Hom(B, \omega_A^\bullet)$ is a dualizing complex for $B$.
\end{lemma}

\begin{proof}
Special case of Lemma \ref{lemma-dualizing-finite}.
\end{proof}

\begin{lemma}
\label{lemma-dualizing-polynomial-ring}
Let $A$ be a Noetherian ring. If $\omega_A^\bullet$ is a dualizing
complex, then $\omega_A^\bullet \otimes_A A[x]$ is a dualizing
complex for $A[x]$.
\end{lemma}

\begin{proof}
Set $B = A[x]$ and $\omega_B^\bullet = \omega_A^\bullet \otimes_A B$.
It follows from Lemma \ref{lemma-injective-dimension-over-polynomial-ring}
and More on Algebra, Lemma \ref{more-algebra-lemma-finite-injective-dimension}
that $\omega_B^\bullet$ has finite injective dimension.
Since $H^i(\omega_B^\bullet) = H^i(\omega_A^\bullet) \otimes_A B$
by flatness of $A \to B$ we see that $\omega_A^\bullet \otimes_A B$
has finite cohomology modules. Finally, the map
$$
B \longrightarrow R\Hom_B(\omega_B^\bullet, \omega_B^\bullet)
$$
is a quasi-isomorphism as formation of internal hom commutes with
flat base change in this case, see
More on Algebra, Lemma \ref{more-algebra-lemma-base-change-RHom}.
\end{proof}

\begin{proposition}
\label{proposition-dualizing-essentially-finite-type}
Let $A$ be a Noetherian ring which has a dualizing complex.
Then any $A$-algebra essentially of finite type over $A$
has a dualizing complex.
\end{proposition}

\begin{proof}
This follows from a combination of
Lemmas \ref{lemma-dualizing-localize},
\ref{lemma-dualizing-quotient}, and \ref{lemma-dualizing-polynomial-ring}.
\end{proof}

\begin{lemma}
\label{lemma-find-function}
Let $A$ be a Noetherian ring. Let $\omega_A^\bullet$ be a dualizing
complex. Let $\mathfrak m \subset A$ be a maximal ideal and set
$\kappa = A/\mathfrak m$. Then
$R\Hom_A(\kappa, \omega_A^\bullet) \cong \kappa[n]$ for some
$n \in \mathbf{Z}$.
\end{lemma}

\begin{proof}
This is true because $R\Hom_A(\kappa, \omega_A^\bullet)$ is a dualizing
complex over $\kappa$ (Lemma \ref{lemma-dualizing-quotient}),
because dualizing complexes over $\kappa$ are unique up to shifts
(Lemma \ref{lemma-dualizing-unique}), and because $\kappa$ is a
dualizing complex over $\kappa$.
\end{proof}




\section{Dualizing complexes over local rings}
\label{section-dualizing-local}

\noindent
In this section $(A, \mathfrak m, \kappa)$ will be a Noetherian local
ring endowed with a dualizing complex $\omega_A^\bullet$ such that
the integer $n$ of Lemma \ref{lemma-find-function} is zero.
More precisely, we assume that $R\Hom_A(\kappa, \omega_A^\bullet) = \kappa[0]$.
In this case we will say that the dualizing complex is {\it normalized}.
Observe that a normalized dualizing complex is unique up to
isomorphism and that any other dualizing complex for $A$ is isomorphic
to a shift of a normalized one (Lemma \ref{lemma-dualizing-unique}).

\begin{lemma}
\label{lemma-normalized-finite}
Let $(A, \mathfrak m, \kappa) \to (B, \mathfrak m', \kappa')$
be a finite local map of Noetherian local rings. Let $\omega_A^\bullet$
be a normalized dualizing complex. Then
$\omega_B^\bullet = R\Hom(B, \omega_A^\bullet)$ is a
normalized dualizing complex for $B$.
\end{lemma}

\begin{proof}
By Lemma \ref{lemma-dualizing-finite} the complex
$\omega_B^\bullet$ is dualizing for $B$. We have
$$
R\Hom_B(\kappa', \omega_B^\bullet) =
R\Hom_B(\kappa', R\Hom(B, \omega_A^\bullet)) =
R\Hom_A(\kappa', \omega_A^\bullet)
$$
by Lemma \ref{lemma-right-adjoint}. Since $\kappa'$ is isomorphic
to a finite direct sum of copies of $\kappa$ as an $A$-module
and since $\omega_A^\bullet$ is normalized, we
see that this complex only has cohomology placed in degree $0$.
Thus $\omega_B^\bullet$ is a normalized dualizing complex as well.
\end{proof}

\begin{lemma}
\label{lemma-normalized-quotient}
Let $(A, \mathfrak m, \kappa)$ be a Noetherian local
ring with normalized dualizing complex $\omega_A^\bullet$.
Let $A \to B$ be surjective. Then
$\omega_B^\bullet = R\Hom_A(B, \omega_A^\bullet)$ is a
normalized dualizing complex for $B$.
\end{lemma}

\begin{proof}
Special case of Lemma \ref{lemma-normalized-finite}.
\end{proof}

\begin{lemma}
\label{lemma-equivalence-finite-length}
Let $(A, \mathfrak m, \kappa)$ be a Noetherian local
ring. Let $F$ be an $A$-linear self-equivalence of the category of
finite length $A$-modules. Then $F$ is isomorphic to the identity functor.
\end{lemma}

\begin{proof}
Since $\kappa$ is the unique simple object of the category we have
$F(\kappa) \cong \kappa$. Since our category is abelian, we find that
$F$ is exact. Hence $F(E)$ has the same length as $E$ for all finite
length modules $E$.
Since $\Hom(E, \kappa) = \Hom(F(E), F(\kappa)) \cong \Hom(F(E), \kappa)$
we conclude from Nakayama's lemma that $E$ and $F(E)$ have the same
number of generators. Hence $F(A/\mathfrak m^n)$ is a cyclic $A$-module.
Pick a generator $e \in F(A/\mathfrak m^n)$.
Since $F$ is $A$-linear we conclude that $\mathfrak m^n e = 0$.
The map $A/\mathfrak m^n \to F(A/\mathfrak m^n)$ has to be
an isomorphism as the lengths are equal. Pick an element
$$
e \in \lim F(A/\mathfrak m^n)
$$
which maps to a generator for all $n$ (small argument omitted).
Then we obtain a system of isomorphisms
$A/\mathfrak m^n \to F(A/\mathfrak m^n)$ compatible with all
$A$-module maps $A/\mathfrak m^n \to A/\mathfrak m^{n'}$ (by $A$-linearity
of $F$ again). Since any finite length module is a cokernel
of a map between direct sums of cyclic modules, we obtain the isomorphism
of the lemma.
\end{proof}

\begin{lemma}
\label{lemma-dualizing-finite-length}
Let $(A, \mathfrak m, \kappa)$ be a Noetherian local
ring with normalized dualizing complex $\omega_A^\bullet$.
Let $E$ be an injective hull of $\kappa$. Then there exists
a functorial isomorphism
$$
R\Hom_A(N, \omega_A^\bullet) = \Hom_A(N, E)[0]
$$
for $N$ running through the finite length $A$-modules.
\end{lemma}

\begin{proof}
By induction on the length of $N$ we see that $R\Hom_A(N, \omega_A^\bullet)$
is a module of finite length sitting in degree $0$. Thus
$R\Hom_A(-, \omega_A^\bullet)$ induces an anti-equivalence
on the category of finite length modules. Since the same is true
for $\Hom_A(-, E)$ by Proposition \ref{proposition-matlis} we see that
$$
N \longmapsto \Hom_A(R\Hom_A(N, \omega_A^\bullet), E)
$$
is an equivalence as in Lemma \ref{lemma-equivalence-finite-length}.
Hence it is isomorphic to the identity functor.
Since $\Hom_A(-, E)$ applied twice is the identity
(Proposition \ref{proposition-matlis}) we obtain
the statement of the lemma.
\end{proof}

\begin{lemma}
\label{lemma-sitting-in-degrees}
Let $(A, \mathfrak m, \kappa)$ be a Noetherian local ring with
normalized dualizing complex $\omega_A^\bullet$. Let $M$ be a finite
$A$-module and let $d = \dim(\text{Supp}(M))$. Then
\begin{enumerate}
\item if $\Ext^i_A(M, \omega_A^\bullet)$ is nonzero, then
$i \in \{-d, \ldots, 0\}$,
\item the dimension of the support of $\Ext^i_A(M, \omega_A^\bullet)$
is at most $-i$,
\item $\text{depth}(M)$ is the smallest integer $\delta \geq 0$ such that
$\Ext^{-\delta}_A(M, \omega_A^\bullet) \not = 0$.
\end{enumerate}
\end{lemma}

\begin{proof}
We prove this by induction on $d$. If $d = 0$, this follows from
Lemma \ref{lemma-dualizing-finite-length} and Matlis duality
(Proposition \ref{proposition-matlis}) which guarantees that
$\Hom_A(M, E)$ is nonzero if $M$ is nonzero.

\medskip\noindent
Assume the result holds for modules with support of dimension $< d$ and that
$M$ has depth $> 0$. Choose an $f \in \mathfrak m$ which is a nonzerodivisor
on $M$ and consider the short exact sequence
$$
0 \to M \to M \to M/fM \to 0
$$
Since $\dim(\text{Supp}(M/fM)) = d - 1$
(Algebra, Lemma \ref{algebra-lemma-one-equation-module}) we
may apply the induction hypothesis.
Writing
$E^i = \Ext^i_A(M, \omega_A^\bullet)$ and
$F^i = \Ext^i_A(M/fM, \omega_A^\bullet)$
we obtain a long exact sequence
$$
\ldots \to F^i \to E^i \xrightarrow{f} E^i \to F^{i + 1} \to \ldots
$$
By induction $E^i/fE^i = 0$ for
$i + 1 \not \in \{-\dim(\text{Supp}(M/fM)), \ldots, -\text{depth}(M/fM)\}$.
By Nakayama's lemma (Algebra, Lemma \ref{algebra-lemma-NAK})
and Algebra, Lemma \ref{algebra-lemma-depth-drops-by-one}
we conclude $E^i = 0$ for
$i \not \in \{-\dim(\text{Supp}(M)), \ldots, -\text{depth}(M)\}$.
Moreover, in the boundary case $i = - \text{depth}(M)$ we deduce that $E^i$
is nonzero as $F^{i + 1}$ is nonzero by induction.
Since $E^i/fE^i \subset F^{i + 1}$ we get
$$
\dim(\text{Supp}(F^{i + 1})) \geq \dim(\text{Supp}(E^i/fE^i))
\geq \dim(\text{Supp}(E^i)) - 1
$$
(see lemma used above) we also obtain the dimension estimate (2).

\medskip\noindent
If $M$ has depth $0$ and $d > 0$ we let $N = M[\mathfrak m^\infty]$ and set
$M' = M/N$ (compare with Lemma \ref{lemma-divide-by-torsion}).
Then $M'$ has depth $> 0$ and $\dim(\text{Supp}(M')) = d$.
Thus we know the result for $M'$ and since
$R\Hom_A(N, \omega_A^\bullet) = \Hom_A(N, E)$
(Lemma \ref{lemma-dualizing-finite-length})
the long exact cohomology sequence of $\Ext$'s implies the
result for $M$.
\end{proof}

\begin{remark}
\label{remark-vanishing-for-arbitrary-modules}
Let $(A, \mathfrak m)$ and $\omega_A^\bullet$ be as in
Lemma \ref{lemma-sitting-in-degrees}.
By More on Algebra, Lemma \ref{more-algebra-lemma-injective-amplitude}
we see that $\omega_A^\bullet$ has injective-amplitude in $[-d, 0]$
because part (3) of that lemma applies.
In particular, for any $A$-module $M$ (not necessarily finite) we have
$\Ext^i_A(M, \omega_A^\bullet) = 0$ for $i \not \in \{-d, \ldots, 0\}$.
\end{remark}

\begin{lemma}
\label{lemma-local-CM}
Let $(A, \mathfrak m, \kappa)$ be a Noetherian local ring
with normalized dualizing complex $\omega_A^\bullet$. Let $M$
be a finite $A$-module. The following are equivalent
\begin{enumerate}
\item $M$ is Cohen-Macaulay,
\item $\Ext^i_A(M, \omega_A^\bullet)$ is nonzero for a single $i$,
\item $\Ext^{-i}_A(M, \omega_A^\bullet)$ is zero for
$i \not = \dim(\text{Supp}(M))$.
\end{enumerate}
Denote $CM_d$ the category of finite Cohen-Macaulay $A$-modules
of depth $d$. Then $M \mapsto \Ext^{-d}_A(M, \omega_A^\bullet)$
defines an anti-auto-equivalence of $CM_d$.
\end{lemma}

\begin{proof}
We will use the results of Lemma \ref{lemma-sitting-in-degrees}
without further mention. Fix a finite module $M$.
If $M$ is Cohen-Macaulay, then only
$\Ext^{-d}_A(M, \omega_A^\bullet)$ can be nonzero,
hence (1) $\Rightarrow$ (3).
The implication (3) $\Rightarrow$ (2) is immediate.
Assume (2) and let $N = \Ext^{-\delta}_A(M, \omega_A^\bullet)$
be the nonzero $\Ext$ where $\delta = \text{depth}(M)$. Then, since
$$
M[0] = R\Hom_A(R\Hom_A(M, \omega_A^\bullet), \omega_A^\bullet) =
R\Hom_A(N[\delta], \omega_A^\bullet)
$$
(Lemma \ref{lemma-dualizing})
we conclude that $M = \Ext_A^{-\delta}(N, \omega_A^\bullet)$.
Thus $\delta \geq \dim(\text{Supp}(M))$. However,
since we also know that $\delta \leq \dim(\text{Supp}(M))$
(Algebra, Lemma \ref{algebra-lemma-bound-depth}) we conclude that $M$ is
Cohen-Macaulay.

\medskip\noindent
To prove the final statement, it suffices to show that
$N = \Ext^{-d}_A(M, \omega_A^\bullet)$ is in $CM_d$
for $M$ in $CM_d$. Above we have seen that
$M[0] = R\Hom_A(N[d], \omega_A^\bullet)$ and this proves the
desired result by the equivalence of (1) and (3).
\end{proof}

\begin{lemma}
\label{lemma-dualizing-artinian}
Let $(A, \mathfrak m, \kappa)$ be a Noetherian local
ring with normalized dualizing complex $\omega_A^\bullet$.
If $\dim(A) = 0$, then $\omega_A^\bullet \cong E[0]$
where $E$ is an injective hull of the residue field.
\end{lemma}

\begin{proof}
Immediate from Lemma \ref{lemma-dualizing-finite-length}.
\end{proof}

\begin{lemma}
\label{lemma-divide-by-finite-length-ideal}
Let $(A, \mathfrak m, \kappa)$ be a Noetherian local
ring with normalized dualizing complex. Let $I \subset \mathfrak m$ be an
ideal of finite length. Set $B = A/I$. Then there is a distinguished
triangle
$$
\omega_B^\bullet \to \omega_A^\bullet \to \Hom_A(I, E)[0] \to
\omega_B^\bullet[1]
$$
in $D(A)$ where $E$ is an injective hull of $\kappa$ and
$\omega_B^\bullet$ is a normalized dualizing complex for $B$.
\end{lemma}

\begin{proof}
Use the short exact sequence $0 \to I \to A \to B \to 0$
and Lemmas \ref{lemma-dualizing-finite-length} and
\ref{lemma-normalized-quotient}.
\end{proof}

\begin{lemma}
\label{lemma-divide-by-nonzerodivisor}
Let $(A, \mathfrak m, \kappa)$ be a Noetherian local
ring with normalized dualizing complex $\omega_A^\bullet$.
Let $f \in \mathfrak m$ be a
nonzerodivisor. Set $B = A/(f)$. Then there is a distinguished
triangle
$$
\omega_B^\bullet \to \omega_A^\bullet \to \omega_A^\bullet \to
\omega_B^\bullet[1]
$$
in $D(A)$ where $\omega_B^\bullet$ is a normalized dualizing complex
for $B$.
\end{lemma}

\begin{proof}
Use the short exact sequence $0 \to A \to A \to B \to 0$
and Lemma \ref{lemma-normalized-quotient}.
\end{proof}

\begin{lemma}
\label{lemma-nonvanishing-generically-local}
Let $(A, \mathfrak m, \kappa)$ be a Noetherian local ring with
normalized dualizing complex $\omega_A^\bullet$.
Let $\mathfrak p$ be a minimal prime of $A$ with
$\dim(A/\mathfrak p) = e$. Then
$H^i(\omega_A^\bullet)_\mathfrak p$ is nonzero
if and only if $i = -e$.
\end{lemma}

\begin{proof}
Since $A_\mathfrak p$ has dimension zero, there exists an integer
$n > 0$ such that $\mathfrak p^nA_\mathfrak p$ is zero.
Set $B = A/\mathfrak p^n$ and
$\omega_B^\bullet = R\Hom_A(B, \omega_A^\bullet)$.
Since $B_\mathfrak p = A_\mathfrak p$ we see that
$$
(\omega_B^\bullet)_\mathfrak p =
R\Hom_A(B, \omega_A^\bullet) \otimes_A^\mathbf{L} A_\mathfrak p =
R\Hom_{A_\mathfrak p}(B_\mathfrak p, (\omega_A^\bullet)_\mathfrak p) =
(\omega_A^\bullet)_\mathfrak p
$$
The second equality holds by
More on Algebra, Lemma \ref{more-algebra-lemma-base-change-RHom}.
By Lemma \ref{lemma-normalized-quotient} we may replace $A$ by $B$.
After doing so, we see that $\dim(A) = e$. Then we see that
$H^i(\omega_A^\bullet)_\mathfrak p$ can only be nonzero if $i = -e$
by Lemma \ref{lemma-sitting-in-degrees} parts (1) and (2).
On the other hand, since $(\omega_A^\bullet)_\mathfrak p$
is a dualizing complex for the nonzero ring $A_\mathfrak p$
(Lemma \ref{lemma-dualizing-localize})
we see that the remaining module has to be nonzero.
\end{proof}





\section{Dualizing complexes and dimension functions}
\label{section-dimension-function}

\noindent
Our results in the local setting have the following consequence:
a Noetherian ring which has a dualizing complex is a
universally catenary ring of finite dimension.

\begin{lemma}
\label{lemma-nonvanishing-generically}
Let $A$ be a Noetherian ring. Let $\mathfrak p$ be a minimal prime
of $A$. Then $H^i(\omega_A^\bullet)_\mathfrak p$ is nonzero
for exactly one $i$.
\end{lemma}

\begin{proof}
The complex $\omega_A^\bullet \otimes_A A_\mathfrak p$
is a dualizing complex for $A_\mathfrak p$
(Lemma \ref{lemma-dualizing-localize}).
The dimension of $A_\mathfrak p$ is zero as $\mathfrak p$
is minimal. Hence the result follows from
Lemma \ref{lemma-dualizing-artinian}.
\end{proof}

\noindent
Let $A$ be a Noetherian ring and let $\omega_A^\bullet$ be a dualizing
complex. Lemma \ref{lemma-find-function} allows us to define a function
$$
\delta = \delta_{\omega_A^\bullet} : \Spec(A) \longrightarrow \mathbf{Z}
$$
by mapping $\mathfrak p$ to the integer of Lemma \ref{lemma-find-function}
for the dualizing complex $(\omega_A^\bullet)_\mathfrak p$
over $A_\mathfrak p$ (Lemma \ref{lemma-dualizing-localize})
and the residue field $\kappa(\mathfrak p)$. To be precise, we define
$\delta(\mathfrak p)$ to be the unique integer such that
$$
(\omega_A^\bullet)_\mathfrak p[-\delta(\mathfrak p)]
$$
is a normalized dualizing complex over the Noetherian local ring
$A_\mathfrak p$.

\begin{lemma}
\label{lemma-quotient-function}
Let $A$ be a Noetherian ring and let $\omega_A^\bullet$ be a dualizing
complex. Let $A \to B$ be a surjective ring map and let
$\omega_B^\bullet = R\Hom(B, \omega_A^\bullet)$ be the dualizing
complex for $B$ of Lemma \ref{lemma-dualizing-quotient}. Then we have
$$
\delta_{\omega_B^\bullet} = \delta_{\omega_A^\bullet}|_{\Spec(B)}
$$
\end{lemma}

\begin{proof}
This follows from the definition of the functions and
Lemma \ref{lemma-normalized-quotient}.
\end{proof}

\begin{lemma}
\label{lemma-dimension-function}
Let $A$ be a Noetherian ring and let $\omega_A^\bullet$ be a dualizing
complex. The function $\delta = \delta_{\omega_A^\bullet}$
defined above is a dimension function
(Topology, Definition \ref{topology-definition-dimension-function}).
\end{lemma}

\begin{proof}
Let $\mathfrak p \subset \mathfrak q$ be an immediate specialization.
We have to show that $\delta(\mathfrak p) = \delta(\mathfrak q) + 1$.
We may replace $A$ by $A/\mathfrak p$, the complex $\omega_A^\bullet$ by
$\omega_{A/\mathfrak p}^\bullet = R\Hom(A/\mathfrak p, \omega_A^\bullet)$,
the prime $\mathfrak p$ by $(0)$, and the prime $\mathfrak q$
by $\mathfrak q/\mathfrak p$,
see Lemma \ref{lemma-quotient-function}. Thus we may assume that
$A$ is a domain, $\mathfrak p = (0)$, and $\mathfrak q$ is a prime
ideal of height $1$.

\medskip\noindent
Then $H^i(\omega_A^\bullet)_{(0)}$ is nonzero
for exactly one $i$, say $i_0$, by Lemma \ref{lemma-nonvanishing-generically}.
In fact $i_0 = -\delta((0))$ because
$(\omega_A^\bullet)_{(0)}[-\delta((0))]$
is a normalized dualizing complex over the field $A_{(0)}$.

\medskip\noindent
On the other hand $(\omega_A^\bullet)_\mathfrak q[-\delta(\mathfrak q)]$
is a normalized dualizing complex for $A_\mathfrak q$. By
Lemma \ref{lemma-nonvanishing-generically-local}
we see that
$$
H^e((\omega_A^\bullet)_\mathfrak q[-\delta(\mathfrak q)])_{(0)} =
H^{e - \delta(\mathfrak q)}(\omega_A^\bullet)_{(0)}
$$
is nonzero only for $e = -\dim(A_\mathfrak q) = -1$.
We conclude
$$
-\delta((0)) = -1 - \delta(\mathfrak q)
$$
as desired.
\end{proof}

\begin{lemma}
\label{lemma-universally-catenary}
Let $A$ be a Noetherian ring which has a dualizing
complex. Then $A$ is universally catenary of finite dimension.
\end{lemma}

\begin{proof}
Because $\Spec(A)$ has a dimension function by
Lemma \ref{lemma-dimension-function}
it is catenary, see
Topology, Lemma \ref{topology-lemma-dimension-function-catenary}.
Hence $A$ is catenary, see
Algebra, Lemma \ref{algebra-lemma-catenary}.
It follows from
Proposition \ref{proposition-dualizing-essentially-finite-type}
that $A$ is universally catenary.

\medskip\noindent
Because any dualizing complex $\omega_A^\bullet$ is
in $D^b_{\textit{Coh}}(A)$ the values of the function
$\delta_{\omega_A^\bullet}$ in minimal primes are bounded by
Lemma \ref{lemma-nonvanishing-generically}.
On the other hand, for a maximal ideal $\mathfrak m$ with
residue field $\kappa$ the integer $i = -\delta(\mathfrak m)$
is the unique integer such that
$\Ext_A^i(\kappa, \omega_A^\bullet)$ is nonzero
(Lemma \ref{lemma-find-function}).
Since $\omega_A^\bullet$ has finite injective dimension
these values are bounded too. Since the dimension of
$A$ is the maximal value of $\delta(\mathfrak p) - \delta(\mathfrak m)$
where $\mathfrak p \subset \mathfrak m$ are a pair
consisting of a minimal prime and a maximal prime we find that the
dimension of $\Spec(A)$ is bounded.
\end{proof}

\begin{lemma}
\label{lemma-depth-dualizing-module}
Let $(A, \mathfrak m, \kappa)$ be a Noetherian local ring with
normalized dualizing complex $\omega_A^\bullet$. Let $d = \dim(A)$
and $\omega_A = H^{-d}(\omega_A^\bullet)$. Then
\begin{enumerate}
\item the support of $\omega_A$ is the union of the irreducible components
of $\Spec(A)$ of dimension $d$,
\item $\omega_A$ satisfies $(S_2)$, see
Algebra, Definition \ref{algebra-definition-conditions}.
\end{enumerate}
\end{lemma}

\begin{proof}
We will use Lemma \ref{lemma-sitting-in-degrees} without further mention.
By Lemma \ref{lemma-nonvanishing-generically-local} the support
of $\omega_A$ contains the irreducible components of dimension $d$.
Let $\mathfrak p \subset A$ be a prime. By Lemma \ref{lemma-dimension-function}
the complex $(\omega_A^\bullet)_{\mathfrak p}[-\dim(A/\mathfrak p)]$
is a normalized dualizing complex for $A_\mathfrak p$. Hence if
$\dim(A/\mathfrak p) + \dim(A_\mathfrak p) < d$, then
$(\omega_A)_\mathfrak p = 0$.
This proves the support of $\omega_A$ is the union of the irreducible
components of dimension $d$, because the complement of this union
is exactly the primes $\mathfrak p$ of $A$ for which
$\dim(A/\mathfrak p) + \dim(A_\mathfrak p) < d$ as $A$ is catenary
(Lemma \ref{lemma-universally-catenary}).
On the other hand, if $\dim(A/\mathfrak p) + \dim(A_\mathfrak p) = d$, then
$$
(\omega_A)_\mathfrak p =
H^{-\dim(A_\mathfrak p)}\left(
(\omega_A^\bullet)_{\mathfrak p}[-\dim(A/\mathfrak p)] \right)
$$
Hence in order to prove $\omega_A$ has $(S_2)$ it suffices to show that
the depth of $\omega_A$ is at least $\min(\dim(A), 2)$.
We prove this by induction on $\dim(A)$. The case $\dim(A) = 0$ is
trivial.

\medskip\noindent
Assume $\text{depth}(A) > 0$. Choose a nonzerodivisor $f \in \mathfrak m$
and set $B = A/fA$. Then $\dim(B) = \dim(A) - 1$ and we may apply the
induction hypothesis to $B$. By Lemma \ref{lemma-divide-by-nonzerodivisor}
we see that multiplication by $f$ is injective on $\omega_A$ and we get
$\omega_A/f\omega_A \subset \omega_B$. This proves the depth of $\omega_A$
is at least $1$. If $\dim(A) > 1$, then $\dim(B) > 0$ and $\omega_B$
has depth $ > 0$. Hence $\omega_A$ has depth $> 1$ and we conclude in
this case.

\medskip\noindent
Assume $\dim(A) > 0$ and $\text{depth}(A) = 0$. Let
$I = A[\mathfrak m^\infty]$ and set $B = A/I$. Then $B$ has
depth $\geq 1$ and $\omega_A = \omega_B$ by
Lemma \ref{lemma-divide-by-finite-length-ideal}.
Since we proved the result for $\omega_B$ above the proof is done.
\end{proof}





\section{The local duality theorem}
\label{section-local-duality}

\noindent
The main result in this section is due to Grothendieck.

\begin{lemma}
\label{lemma-local-cohomology-of-dualizing}
Let $(A, \mathfrak m, \kappa)$ be a Noetherian local ring.
Let $\omega_A^\bullet$ be a normalized dualizing complex.
Let $Z = V(\mathfrak m) \subset \Spec(A)$.
Then $E = R^0\Gamma_Z(\omega_A^\bullet)$ is an injective hull of
$\kappa$ and $R\Gamma_Z(\omega_A^\bullet) = E[0]$.
\end{lemma}

\begin{proof}
By Lemma \ref{lemma-local-cohomology-noetherian} we have
$R\Gamma_{\mathfrak m} = R\Gamma_Z$. Thus
$$
R\Gamma_Z(\omega_A^\bullet) =
R\Gamma_{\mathfrak m}(\omega_A^\bullet) =
\text{hocolim}\ R\Hom_A(A/\mathfrak m^n, \omega_A^\bullet)
$$
by Lemma \ref{lemma-local-cohomology-ext}. Let $E'$ be an injective
hull of the residue field.
By Lemma \ref{lemma-dualizing-finite-length}
we can find isomorphisms
$$
R\Hom_A(A/\mathfrak m^n, \omega_A^\bullet) \cong \Hom_A(A/\mathfrak m^n, E')[0]
$$
compatible with transition maps. Since
$E' = \bigcup E'[\mathfrak m^n] = \colim \Hom_A(A/\mathfrak m^n, E')$
by Lemma \ref{lemma-union-artinian}
we conclude that $E \cong E'$ and that all other cohomology
groups of the complex $R\Gamma_Z(\omega_A^\bullet)$ are zero.
\end{proof}

\begin{remark}
\label{remark-specific-injective-hull}
Let $(A, \mathfrak m, \kappa)$ be a Noetherian local ring
with a normalized dualizing complex $\omega_A^\bullet$.
By Lemma \ref{lemma-local-cohomology-of-dualizing}
above we see that $R\Gamma_Z(\omega_A^\bullet)$
is an injective hull of the residue field placed in degree $0$.
In fact, this gives a ``construction'' or ``realization''
of the injective hull which is slightly more canonical than
just picking any old injective hull. Namely, a normalized
dualizing complex is unique up to isomorphism, with group
of automorphisms the group of units of $A$, whereas an
injective hull of $\kappa$ is unique up to isomorphism, with
group of automorphisms the group of units of the completion
$A^\wedge$ of $A$ with respect to $\mathfrak m$.
\end{remark}

\noindent
Here is the main result of this section.

\begin{theorem}
\label{theorem-local-duality}
Let $(A, \mathfrak m, \kappa)$ be a Noetherian local ring.
Let $\omega_A^\bullet$ be a normalized dualizing complex.
Let $E$ be an injective hull of the residue field.
Let $Z = V(\mathfrak m) \subset \Spec(A)$.
Denote ${}^\wedge$ derived completion with respect to $\mathfrak m$.
Then
$$
R\Hom_A(K, \omega_A^\bullet)^\wedge \cong R\Hom_A(R\Gamma_Z(K), E[0])
$$
for $K$ in $D(A)$.
\end{theorem}

\begin{proof}
Observe that $E[0] \cong R\Gamma_Z(\omega_A^\bullet)$ by
Lemma \ref{lemma-local-cohomology-of-dualizing}.
By More on Algebra, Lemma \ref{more-algebra-lemma-completion-RHom}
completion on the left hand side goes inside.
Thus we have to prove
$$
R\Hom_A(K^\wedge, (\omega_A^\bullet)^\wedge)
=
R\Hom_A(R\Gamma_Z(K), R\Gamma_Z(\omega_A^\bullet))
$$
This follows from the equivalence between
$D_{comp}(A, \mathfrak m)$ and $D_{\mathfrak m^\infty\text{-torsion}}(A)$
given in Proposition \ref{proposition-torsion-complete}.
More precisely, it is a special case of Lemma \ref{lemma-compare-RHom}.
\end{proof}

\noindent
Here is a special case of the theorem above.

\begin{lemma}
\label{lemma-special-case-local-duality}
Let $(A, \mathfrak m, \kappa)$ be a Noetherian local ring.
Let $\omega_A^\bullet$ be a normalized dualizing complex.
Let $E$ be an injective hull of the residue field.
Let $K \in D_{\textit{Coh}}(A)$. Then
$$
\Ext^{-i}_A(K, \omega_A^\bullet)^\wedge =
\Hom_A(H^i_{\mathfrak m}(K), E)
$$
where ${}^\wedge$ denotes $\mathfrak m$-adic completion.
\end{lemma}

\begin{proof}
By Lemma \ref{lemma-dualizing} we see that $R\Hom_A(K, \omega_A^\bullet)$
is an object of $D_{\textit{Coh}}(A)$.
It follows that the cohomology modules of the derived completion
of $R\Hom_A(K, \omega_A^\bullet)$ are equal to the usual completions
$\Ext^i_A(K, \omega_A^\bullet)^\wedge$ by
More on Algebra, Lemma
\ref{more-algebra-lemma-derived-completion-pseudo-coherent}.
On the other hand, we have $R\Gamma_{\mathfrak m} = R\Gamma_Z$
for $Z = V(\mathfrak m)$ by Lemma \ref{lemma-local-cohomology-noetherian}.
Moreover, the functor $\Hom_A(-, E)$ is exact hence
factors through cohomology.
Hence the lemma is consequence of
Theorem \ref{theorem-local-duality}.
\end{proof}





\section{Dualizing modules}
\label{section-dualizing-module}

\noindent
If $(A, \mathfrak m, \kappa)$ is a Noetherian local ring and
$\omega_A^\bullet$ is a normalized dualizing complex, then
we say the module $\omega_A = H^{-\dim(A)}(\omega_A^\bullet)$, described
in Lemma \ref{lemma-depth-dualizing-module},
is a {\it dualizing module}
for $A$. This module is a canonical module of $A$.
It seems generally agreed upon to define a {\it canonical module}
for a Noetherian local ring $(A, \mathfrak m, \kappa)$ to be
a finite $A$-module $K$ such that
$$
\Hom_A(K, E) \cong H^{\dim(A)}_\mathfrak m(A)
$$
where $E$ is an injective hull of the residue field. A dualizing
module is canonical because
$$
\Hom_A(H^{\dim(A)}_\mathfrak m(A), E) = (\omega_A)^\wedge
$$
by Lemma \ref{lemma-special-case-local-duality}
and hence applying
$\Hom_A(-, E)$ we get
\begin{align*}
\Hom_A(\omega_A, E)
& =
\Hom_A((\omega_A)^\wedge, E) \\
& =
\Hom_A(\Hom_A(H^{\dim(A)}_\mathfrak m(A), E), E) \\
& = H^{\dim(A)}_\mathfrak m(A)
\end{align*}
the first equality because $E$ is $\mathfrak m$-power torsion, the
second by the above, and the third by Matlis duality
(Proposition \ref{proposition-matlis}).
The utility of the definition
of a canonical module given above lies in the fact that it makes sense
even if $A$ does not have a dualizing complex.




\section{Cohen-Macaulay rings}
\label{section-CM}

\noindent
Cohen-Macaulay modules and rings were studied in
Algebra, Sections \ref{algebra-section-CM} and \ref{algebra-section-CM-ring}.

\begin{lemma}
\label{lemma-depth-in-terms-dualizing-complex}
Let $(A, \mathfrak m, \kappa)$ be a Noetherian local ring with
normalized dualizing complex $\omega_A^\bullet$.
Then $\text{depth}(A)$ is equal to the smallest integer $\delta \geq 0$
such that $H^{-\delta}(\omega_A^\bullet) \not = 0$.
\end{lemma}

\begin{proof}
This follows immediately from
Lemma \ref{lemma-sitting-in-degrees}.
Here are two other ways to see that it is true.

\medskip\noindent
First alternative. By Nakayama's lemma we see that
$\delta$ is the smallest integer such that
$\Hom_A(H^{-\delta}(\omega_A^\bullet), \kappa) \not = 0$.
In other words, it is the smallest integer such that
$\Ext_A^{-\delta}(\omega_A^\bullet, \kappa)$
is nonzero. Using Lemma \ref{lemma-dualizing} and the fact that
$\omega_A^\bullet$ is normalized this is equal to the
smallest integer such that $\Ext_A^\delta(\kappa, A)$ is
nonzero. This is equal to the depth of $A$ by
Algebra, Lemma \ref{algebra-lemma-depth-ext}.

\medskip\noindent
Second alternative. By the local duality theorem
(in the form of Lemma \ref{lemma-special-case-local-duality})
$\delta$ is the smallest integer such that $H^\delta_\mathfrak m(A)$
is nonzero. This is equal to the depth of $A$ by
Lemma \ref{lemma-depth}.
\end{proof}

\begin{lemma}
\label{lemma-apply-CM}
Let $(A, \mathfrak m, \kappa)$ be a Noetherian local ring
with normalized dualizing complex $\omega_A^\bullet$
and dualizing module $\omega_A = H^{-\dim(A)}(\omega_A^\bullet)$.
The following are equivalent
\begin{enumerate}
\item $A$ is Cohen-Macaulay,
\item $\omega_A^\bullet$ is concentrated in a single degree, and
\item $\omega_A^\bullet = \omega_A[\dim(A)]$.
\end{enumerate}
In this case $\omega_A$ is a maximal Cohen-Macaulay module.
\end{lemma}

\begin{proof}
Follows immediately from Lemma \ref{lemma-local-CM}.
\end{proof}

\begin{lemma}
\label{lemma-has-dualizing-module-CM}
Let $A$ be a Noetherian ring. If there exists a finite $A$-module
$\omega_A$ such that $\omega_A[0]$ is a dualizing complex, then
$A$ is Cohen-Macaulay.
\end{lemma}

\begin{proof}
We may replace $A$ by the localization at a prime
(Lemma \ref{lemma-dualizing-localize} and
Algebra, Definition \ref{algebra-definition-ring-CM}).
In this case the result follows immediately from
Lemma \ref{lemma-apply-CM}.
\end{proof}

\begin{lemma}
\label{lemma-CM-open}
Let $A$ be a Noetherian ring with dualizing complex $\omega_A^\bullet$.
Let $M$ be a finite $A$-module. Then
$$
U = \{\mathfrak p \in \Spec(A) \mid M_\mathfrak p\text{ is Cohen-Macaulay}\}
$$
is an open subset of $\Spec(A)$ whose intersection with
$\text{Supp}(M)$ is dense.
\end{lemma}

\begin{proof}
If $\mathfrak p$ is a generic point of $\text{Supp}(M)$, then
$\text{depth}(M_\mathfrak p) = \dim(M_\mathfrak p) = 0$
and hence $\mathfrak p \in U$. This proves denseness.
If $\mathfrak p \in U$, then we see that
$$
R\Hom_A(M, \omega_A^\bullet)_\mathfrak p =
R\Hom_{A_\mathfrak p}(M_\mathfrak p, (\omega_A^\bullet)_\mathfrak p)
$$
has a unique nonzero cohomology module, say in degree $i_0$, by
Lemma \ref{lemma-local-CM}.
Since $R\Hom_A(M, \omega_A^\bullet)$
has only a finite number of nonzero cohomology modules $H^i$
and since each of these is a finite $A$-module, we can
find an $f \in A$, $f \not \in \mathfrak p$ such that
$(H^i)_f = 0$ for $i \not = i_0$. Then
$R\Hom_A(M, \omega_A^\bullet)_f$ has a unique nonzero cohomology
module and reversing the arguments just given we find
that $D(f) \subset U$.
\end{proof}

\begin{lemma}
\label{lemma-CM}
Let $A$ be a Noetherian ring. If $A$ has a dualizing complex
$\omega_A^\bullet$, then
$\{\mathfrak p \in \Spec(A) \mid A_\mathfrak p\text{ is Cohen-Macaulay}\}$
is a dense open subset of $\Spec(A)$.
\end{lemma}

\begin{proof}
Immediate consequence of Lemma \ref{lemma-CM-open} and the definitions.
\end{proof}






\section{Gorenstein rings}
\label{section-gorenstein}

\noindent
So far, the only explicit dualizing complex we've seen is $\kappa$ on $\kappa$
for a field $\kappa$, see proof of Lemma \ref{lemma-find-function}.
By Proposition \ref{proposition-dualizing-essentially-finite-type}
this means that any finite type algebra over a field has a dualizing
complex. However, it turns out that there are Noetherian (local) rings
which do not have a dualizing complex. Namely, we have seen that
a ring which has a dualizing complex is universally catenary
(Lemma \ref{lemma-universally-catenary})
but there are examples of
Noetherian local rings which are not catenary, see
Examples, Section \ref{examples-section-non-catenary-Noetherian-local}.

\medskip\noindent
Nonetheless many rings in algebraic geometry have dualizing complexes
simply because they are quotients of Gorenstein rings. This condition
is in fact both necessary and sufficient. That is: a Noetherian ring
has a dualizing complex if and only if it is a quotient of a finite
dimensional Gorenstein ring. This is Sharp's conjecture (\cite{Sharp})
which can be found as \cite[Corollary 1.4]{Kawasaki} in the literature.
Returning to our current topic, here is the definition of Gorenstein rings.

\begin{definition}
\label{definition-gorenstein}
Gorenstein rings.
\begin{enumerate}
\item Let $A$ be a Noetherian local ring. We say $A$ is {\it Gorenstein}
if $A[0]$ is a dualizing complex for $A$.
\item Let $A$ be a Noetherian ring. We say $A$ is {\it Gorenstein}
if $A_\mathfrak p$ is Gorenstein for every prime $\mathfrak p$ of $A$.
\end{enumerate}
\end{definition}

\noindent
This definition makes sense, because if $A[0]$ is a dualizing complex
for $A$, then $S^{-1}A[0]$ is a dualizing complex for $S^{-1}A$ by
Lemma \ref{lemma-dualizing-localize}.
We will see later that a finite dimensional Noetherian ring is Gorenstein
if it has finite injective dimension as a module over itself.

\begin{lemma}
\label{lemma-gorenstein-CM}
A Gorenstein ring is Cohen-Macaulay.
\end{lemma}

\begin{proof}
Follows from Lemma \ref{lemma-apply-CM}.
\end{proof}

\noindent
An example of a Gorenstein ring is a regular ring.

\begin{lemma}
\label{lemma-regular-gorenstein}
A regular local ring is Gorenstein.
A regular ring is Gorenstein.
\end{lemma}

\begin{proof}
Let $A$ be a regular ring of finite dimension $d$. Then $A$ has finite
global dimension $d$, see
Algebra, Lemma \ref{algebra-lemma-finite-gl-dim-finite-dim-regular}.
Hence $\Ext^{d + 1}_A(M, A) = 0$ for all $A$-modules $M$, see
Algebra, Lemma \ref{algebra-lemma-projective-dimension-ext}.
Thus $A$ has finite injective dimension as an $A$-module by
More on Algebra, Lemma \ref{more-algebra-lemma-injective-amplitude}.
It follows that $A[0]$ is a dualizing complex, hence $A$ is
Gorenstein by the remark following the definition.
\end{proof}

\begin{lemma}
\label{lemma-gorenstein}
Let $A$ be a Noetherian ring.
\begin{enumerate}
\item If $A$ has a dualizing complex $\omega_A^\bullet$, then
\begin{enumerate}
\item $A$ is Gorenstein $\Leftrightarrow$ $\omega_A^\bullet$ is an invertible
object of $D(A)$,
\item $A_\mathfrak p$ is Gorenstein $\Leftrightarrow$
$(\omega_A^\bullet)_\mathfrak p$ is an invertible object of
$D(A_\mathfrak p)$,
\item $\{\mathfrak p \in \Spec(A) \mid A_\mathfrak p\text{ is Gorenstein}\}$
is an open subset.
\end{enumerate}
\item If $A$ is Gorenstein, then $A$ has a dualizing complex if and
only if $A[0]$ is a dualizing complex.
\end{enumerate}
\end{lemma}

\begin{proof}
For invertible objects of $D(A)$, see
More on Algebra, Lemma \ref{more-algebra-lemma-invertible-derived}
and the discussion in Section \ref{section-dualizing}.

\medskip\noindent
By Lemma \ref{lemma-dualizing-localize} for every
$\mathfrak p$ the complex $(\omega_A^\bullet)_\mathfrak p$ is a
dualizing complex over $A_\mathfrak p$. By definition and uniqueness
of dualizing complexes (Lemma \ref{lemma-dualizing-unique})
we see that (1)(b) holds.

\medskip\noindent
To see (1)(c) assume that $A_\mathfrak p$ is Gorenstein.
Let $n_x$ be the unique integer such that
$H^{n_{x}}((\omega_A^\bullet)_\mathfrak p)$
is nonzero and isomorphic to $A_\mathfrak p$.
Since $\omega_A^\bullet$ is in $D^b_{\textit{Coh}}(A)$
there are finitely many nonzero finite $A$-modules
$H^i(\omega_A^\bullet)$. Thus there exists some
$f \in A$, $f \not \in \mathfrak p$
such that only $H^{n_x}((\omega_A^\bullet)_f)$
is nonzero and generated by $1$ element over $A_f$.
Since dualizing complexes are faithful (by definition)
we conclude that $A_f \cong H^{n_x}((\omega_A^\bullet)_f)$.
In this way we see that $A_\mathfrak q$ is Gorenstein
for every $\mathfrak q \in D(f)$. This proves that the set
in (1)(c) is open.

\medskip\noindent
Proof of (1)(a). The implication $\Leftarrow$ follows from (1)(b).
The implication $\Rightarrow$ follows from the discussion
in the previous paragraph, where we showed that if $A_\mathfrak p$
is Gorenstein, then for some $f \in A$, $f \not \in \mathfrak p$
the complex $(\omega_A^\bullet)_f$ has only one nonzero cohomology module
which is invertible.

\medskip\noindent
If $A[0]$ is a dualizing complex then $A$ is Gorenstein by
part (1). Conversely, we see that part (1) shows that
$\omega_A^\bullet$ is locally isomorphic to a shift of $A$.
Since being a dualizing complex is local
(Lemma \ref{lemma-dualizing-glue})
the result is clear.
\end{proof}

\begin{lemma}
\label{lemma-gorenstein-ext}
Let $(A, \mathfrak m, \kappa)$ be a Noetherian local ring.
Then $A$ is Gorenstein if and only if $\Ext^i_A(\kappa, A)$
is zero for $i \gg 0$.
\end{lemma}

\begin{proof}
Observe that $A[0]$ is a dualizing complex for $A$ if and only
if $A$ has finite injective dimension as an $A$-module
(follows immediately from Definition \ref{definition-dualizing}).
Thus the lemma follows from More on Algebra, Lemma
\ref{more-algebra-lemma-finite-injective-dimension-Noetherian-local}.
\end{proof}

\begin{lemma}
\label{lemma-gorenstein-divide-by-nonzerodivisor}
Let $(A, \mathfrak m, \kappa)$
be a Noetherian local ring. Let $f \in \mathfrak m$ be a
nonzerodivisor. Set $B = A/(f)$. Then $A$ is Gorenstein if and
only if $B$ is Gorenstein.
\end{lemma}

\begin{proof}
If $A$ is Gorenstein, then $B$ is Gorenstein by
Lemma \ref{lemma-divide-by-nonzerodivisor}.
Conversely, suppose that $B$ is Gorenstein. Then
$\Ext^i_B(\kappa, B)$ is zero for $i \gg 0$
(Lemma \ref{lemma-gorenstein-ext}).
Recall that $R\Hom(B, -) : D(A) \to D(B)$ is a right adjoint
to restriction (Lemma \ref{lemma-right-adjoint}).
Hence
$$
R\Hom_A(\kappa, A) = R\Hom_B(\kappa, R\Hom(B, A)) =
R\Hom_B(\kappa, B[1])
$$
The final equality by direct computation or by
Lemma \ref{lemma-compute-for-effective-Cartier-algebraic}.
Thus we see that $\Ext^i_A(\kappa, A)$ is zero for
$i \gg 0$ and $A$ is Gorenstein (Lemma \ref{lemma-gorenstein-ext}).
\end{proof}

\begin{lemma}
\label{lemma-gorenstein-lci}
If $A \to B$ is a local complete intersection homomorphism of rings and
$A$ is a Noetherian Gorenstein ring, then $B$ is a Gorenstein ring.
\end{lemma}

\begin{proof}
By More on Algebra, Definition
\ref{more-algebra-definition-local-complete-intersection}
we can write $B = A[x_1, \ldots, x_n]/I$
where $I$ is a Koszul-regular ideal. Observe that a polynomial
ring over a Gorenstein ring $A$ is Gorenstein: reduce to
$A$ local and then use Lemmas \ref{lemma-dualizing-polynomial-ring} and
\ref{lemma-gorenstein}.
A Koszul-regular ideal is by definition locally generated
by a Koszul-regular sequence, see More on Algebra, Section
\ref{more-algebra-section-ideals}.
Looking at local rings of $A[x_1, \ldots, x_n]$
we see it suffices to show: if $R$ is a Noetherian local
Gorenstein ring and $f_1, \ldots, f_c \in \mathfrak m_R$
is a Koszul regular sequence, then $R/(f_1, \ldots, f_c)$ is Gorenstein.
This follows from
Lemma \ref{lemma-gorenstein-divide-by-nonzerodivisor} and
the fact that a Koszul regular sequence in $R$ is just a
regular sequence (More on Algebra, Lemma
\ref{more-algebra-lemma-noetherian-finite-all-equivalent}).
\end{proof}

\begin{lemma}
\label{lemma-flat-under-gorenstein}
Let $A \to B$ be a flat local homomorphism of Noetherian local rings.
The following are equivalent
\begin{enumerate}
\item $B$ is Gorenstein, and
\item $A$ and $B/\mathfrak m_A B$ are Gorenstein.
\end{enumerate}
\end{lemma}

\begin{proof}
Below we will use without further mention that a local Gorenstein ring
has finite injective dimension as well as Lemma \ref{lemma-gorenstein-ext}.
By More on Algebra, Lemma
\ref{more-algebra-lemma-pseudo-coherence-and-base-change-ext}
we have
$$
\Ext^i_A(\kappa_A, A) \otimes_A B =
\Ext^i_B(B/\mathfrak m_A B, B)
$$
for all $i$.

\medskip\noindent
Assume (2). Using that
$R\Hom(B/\mathfrak m_A B, -) : D(B) \to D(B/\mathfrak m_A B)$ is a
right adjoint to restriction (Lemma \ref{lemma-right-adjoint}) we obtain
$$
R\Hom_B(\kappa_B, B) =
R\Hom_{B/\mathfrak m_A B}(\kappa_B, R\Hom(B/\mathfrak m_A B, B))
$$
The cohomology modules of $R\Hom(B/\mathfrak m_A B, B)$ are the modules
$\Ext^i_B(B/\mathfrak m_A B, B) =
\Ext^i_A(\kappa_A, A) \otimes_A B$.
Since $A$ is Gorenstein, we conclude only a finite number of these are nonzero
and each is isomorphic to a direct sum of copies of $B/\mathfrak m_A B$.
Hence since $B/\mathfrak m_A B$ is Gorenstein we conclude that
$R\Hom_B(B/\mathfrak m_B, B)$ has only a finite number of nonzero
cohomology modules. Hence $B$ is Gorenstein.

\medskip\noindent
Assume (1). Since $B$ has finite injective dimension,
$\Ext^i_B(B/\mathfrak m_A B, B)$ is $0$ for $i \gg 0$.
Since $A \to B$ is faithfully flat
we conclude that $\Ext^i_A(\kappa_A, A)$ is $0$
for $i \gg 0$. We conclude that $A$ is Gorenstein. This implies that
$\Ext^i_A(\kappa_A, A)$ is nonzero for exactly one $i$,
namely for $i = \dim(A)$, and
$\Ext^{\dim(A)}_A(\kappa_A, A) \cong \kappa_A$
(see Lemmas \ref{lemma-normalized-finite}, \ref{lemma-apply-CM}, and
\ref{lemma-gorenstein-CM}).
Thus we see that
$\Ext^i_B(B/\mathfrak m_A B, B)$ is zero except for one $i$,
namely $i = \dim(A)$ and
$\Ext^{\dim(A)}_B(B/\mathfrak m_A B, B) \cong B/\mathfrak m_A B$.
Thus $B/\mathfrak m_A B$ is Gorenstein by
Lemma \ref{lemma-normalized-finite}.
\end{proof}

\begin{lemma}
\label{lemma-tor-injective-hull}
Let $(A, \mathfrak m, \kappa)$ be a Noetherian local Gorenstein ring
of dimension $d$. Let $E$ be the injective hull of $\kappa$. Then
$\text{Tor}_i^A(E, \kappa)$ is zero for $i \not = d$
and $\text{Tor}_d^A(E, \kappa) = \kappa$.
\end{lemma}

\begin{proof}
Since $A$ is Gorenstein $\omega_A^\bullet = A[d]$ is a
normalized dualizing complex for $A$.
Also $E$ is the only nonzero cohomology module of
$R\Gamma_\mathfrak m(\omega_A^\bullet)$ sitting in degree $0$, see
Lemma \ref{lemma-local-cohomology-of-dualizing}.
By Lemma \ref{lemma-torsion-tensor-product} we have
$$
E \otimes_A^\mathbf{L} \kappa =
R\Gamma_\mathfrak m(\omega_A^\bullet) \otimes_A^\mathbf{L} \kappa =
R\Gamma_\mathfrak m(\omega_A^\bullet \otimes_A^\mathbf{L} \kappa) =
R\Gamma_\mathfrak m(\kappa[d]) = \kappa[d]
$$
and the lemma follows.
\end{proof}




\section{The ubiquity of dualizing complexes}
\label{section-ubiquity-dualizing}

\noindent
Many Noetherian rings have dualizing complexes.

\begin{lemma}
\label{lemma-flat-unramified}
Let $A \to B$ be a local homomorphism of Noetherian local rings.
Let $\omega_A^\bullet$ be a normalized dualizing complex.
If $A \to B$ is flat and $\mathfrak m_A B = \mathfrak m_B$,
then $\omega_A^\bullet \otimes_A B$ is a normalized dualizing
complex for $B$.
\end{lemma}

\begin{proof}
It is clear that $\omega_A^\bullet \otimes_A B$ is in $D^b_{\textit{Coh}}(B)$.
Let $\kappa_A$ and $\kappa_B$ be the residue fields of $A$ and $B$.
By More on Algebra, Lemma \ref{more-algebra-lemma-base-change-RHom}
we see that
$$
R\Hom_B(\kappa_B, \omega_A^\bullet \otimes_A B) =
R\Hom_A(\kappa_A, \omega_A^\bullet) \otimes_A B =
\kappa_A[0] \otimes_A B = \kappa_B[0]
$$
Thus $\omega_A^\bullet \otimes_A B$ has finite injective dimension by
More on Algebra, Lemma
\ref{more-algebra-lemma-finite-injective-dimension-Noetherian-local}.
Finally, we can use the same arguments to see that
$$
R\Hom_B(\omega_A^\bullet \otimes_A B, \omega_A^\bullet \otimes_A B) =
R\Hom_A(\omega_A^\bullet, \omega_A^\bullet) \otimes_A B = A \otimes_A B = B
$$
as desired.
\end{proof}

\begin{lemma}
\label{lemma-flat-iso-mod-I}
Let $A \to B$ be a flat map of Noetherian rings. Let
$I \subset A$ be an ideal such that $A/I = B/IB$ and
such that $IB$ is contained in the Jacobson radical of $B$.
Let $\omega_A^\bullet$ be a dualizing complex.
Then $\omega_A^\bullet \otimes_A B$ is a dualizing
complex for $B$.
\end{lemma}

\begin{proof}
It is clear that $\omega_A^\bullet \otimes_A B$ is in $D^b_{\textit{Coh}}(B)$.
By More on Algebra, Lemma \ref{more-algebra-lemma-base-change-RHom}
we see that
$$
R\Hom_B(K \otimes_A B, \omega_A^\bullet \otimes_A B) =
R\Hom_A(K, \omega_A^\bullet) \otimes_A B
$$
for any $K \in D^b_{\textit{Coh}}(A)$. For any ideal
$IB \subset J \subset B$ there is a unique ideal $I \subset J' \subset A$
such that $A/J' \otimes_A B = B/J$. Thus $\omega_A^\bullet \otimes_A B$
has finite injective dimension by
More on Algebra, Lemma
\ref{more-algebra-lemma-finite-injective-dimension-Noetherian-radical}.
Finally, we also have
$$
R\Hom_B(\omega_A^\bullet \otimes_A B, \omega_A^\bullet \otimes_A B) =
R\Hom_A(\omega_A^\bullet, \omega_A^\bullet) \otimes_A B = A \otimes_A B = B
$$
as desired.
\end{proof}

\begin{lemma}
\label{lemma-completion-henselization-dualizing}
Let $A$ be a Noetherian ring and let $I \subset A$ be an ideal.
Let $\omega_A^\bullet$ be a dualizing complex.
\begin{enumerate}
\item $\omega_A^\bullet \otimes_A A^h$ is a dualizing complex on the
henselization $(A^h, I^h)$ of the pair $(A, I)$,
\item $\omega_A^\bullet \otimes_A A^\wedge$ is a dualizing complex on
the $I$-adic completion $A^\wedge$, and
\item if $A$ is local, then $\omega_A^\bullet \otimes_A A^h$,
resp.\ $\omega_A^\bullet \otimes_A A^{sh}$ is a dualzing complex
on the henselization, resp.\ strict henselization of $A$.
\end{enumerate}
\end{lemma}

\begin{proof}
Immediate from Lemmas \ref{lemma-flat-unramified} and
\ref{lemma-flat-iso-mod-I}.
See More on Algebra, Sections \ref{more-algebra-section-henselian-pairs},
\ref{more-algebra-section-permanence-completion}, and
\ref{more-algebra-section-permanence-henselization} and
Algebra, Sections \ref{algebra-section-completion} and
\ref{algebra-section-completion-noetherian}
for information on completions and henselizations.
\end{proof}

\begin{lemma}
\label{lemma-ubiquity-dualizing}
The following types of rings have a dualizing complex:
\begin{enumerate}
\item fields,
\item Noetherian complete local rings,
\item $\mathbf{Z}$,
\item Dedekind domains,
\item any ring which is obtained from one of the rings above by
taking an algebra essentially of finite type, or by taking an
ideal-adic completion, or by taking a henselization, 
or by taking a strict henselization.
\end{enumerate}
\end{lemma}

\begin{proof}
Part (5) follows from Proposition
\ref{proposition-dualizing-essentially-finite-type}
and Lemma \ref{lemma-completion-henselization-dualizing}.
By Lemma \ref{lemma-regular-gorenstein} a regular local ring has a
dualizing complex.
A complete Noetherian local ring is the quotient of a regular
local ring by the Cohen structure theorem
(Algebra, Theorem \ref{algebra-theorem-cohen-structure-theorem}).
Let $A$ be a Dedekind domain. Then every ideal $I$ is a finite
projective $A$-module (follows from
Algebra, Lemma \ref{algebra-lemma-finite-projective}
and the fact that the local rings of $A$ are discrete valuation ring
and hence PIDs). Thus every $A$-module has finite injective dimension
at most $1$ by
More on Algebra, Lemma \ref{more-algebra-lemma-injective-amplitude}.
It follows easily that $A[0]$ is a dualizing complex.
\end{proof}





\section{Formal fibres}
\label{section-formal-fibres}

\noindent
This section is a continuation of
More on Algebra, Section \ref{more-algebra-section-properties-formal-fibres}.
There we saw there is a (fairly) good theory of Noetherian rings $A$
whose local rings have Cohen-Macaulay formal fibres. Namely, we proved
(1) it suffices to check the formal fibres of localizations at
maximal ideals are Cohen-Macaulay,
(2) the property is inherited by rings of finite type over $A$,
(3) the fibres of $A \to A^\wedge$ are Cohen-Macaulay for
any completion $A^\wedge$ of $A$, and
(4) the property is inherited by henselizations of $A$. See
More on Algebra, Lemma \ref{more-algebra-lemma-check-P-ring-maximal-ideals},
Proposition \ref{more-algebra-proposition-finite-type-over-P-ring},
Lemma \ref{more-algebra-lemma-map-P-ring-to-completion-P}, and
Lemma \ref{more-algebra-lemma-henselization-pair-P-ring}.
Similarly, for Noetherian rings whose local rings have formal fibres
which are geometrically reduced, geometrically normal, $(S_n)$, and
geometrically $(R_n)$.
In this section we will see that the same is true for Noetherian rings
whose local rings have formal fibres which are Gorenstein
or local complete intersections.
This is relevant to this chapter because a Noetherian ring which has a
dualizing complex is an example.

\begin{lemma}
\label{lemma-formal-fibres-gorenstein}
Properties (A), (B), (C), (D), and (E) of
More on Algebra, Section \ref{more-algebra-section-properties-formal-fibres}
hold for $P(k \to R) =$``$R$ is a Gorenstein ring''.
\end{lemma}

\begin{proof}
Since we already know the result holds for Cohen-Macaulay instead
of Gorenstein, we may in each step assume the ring we have is
Cohen-Macaulay. This is not particularly helpful for the proof, but
psychologically may be useful.

\medskip\noindent
Part (A). Let $k \subset K$ be a finitely generated field extension.
Let $R$ be a Gorenstein $k$-algebra.
We can find a global complete intersection
$A = k[x_1, \ldots, x_n]/(f_1, \ldots, f_c)$
over $k$ such that $K$ is isomorphic to the fraction field of $A$, see
Algebra, Lemma \ref{algebra-lemma-colimit-syntomic}.
Then $R \to R \otimes_k A$ is a relative global complete intersection.
Hence $R \otimes_k A$ is Gorenstein by Lemma \ref{lemma-gorenstein-lci}.
Thus $R \otimes_k K$ is too as a localization.

\medskip\noindent
Proof of (B). This is clear because a ring is Gorenstein
if and only if all of its local rings are Gorenstein.

\medskip\noindent
Part (C). Let $A \to B \to C$ be flat maps of Noetherian rings.
Assume the fibres of $A \to B$ are Gorenstein and $B \to C$ is regular.
We have to show the fibres of $A \to C$ are Gorenstein.
Clearly, we may assume $A = k$ is a field. Then we may assume that
$B \to C$ is a regular local homomorphism of Noetherian local rings.
Then $B$ is Gorenstein and $C/\mathfrak m_B C$ is regular, in
particular Gorenstein (Lemma \ref{lemma-regular-gorenstein}).
Then $C$ is Gorenstein by
Lemma \ref{lemma-flat-under-gorenstein}.

\medskip\noindent
Part (D). This follows from Lemma \ref{lemma-flat-under-gorenstein}.
Part (E) is immediate as the condition does not refer to the ground field.
\end{proof}

\begin{lemma}
\label{lemma-dualizing-gorenstein-formal-fibres}
Let $A$ be a Noetherian local ring. If $A$ has a dualizing complex,
then the formal fibres of $A$ are Gorenstein.
\end{lemma}

\begin{proof}
Let $\mathfrak p$ be a prime of $A$. The formal fibre of $A$ at $\mathfrak p$
is isomorphic to the formal fibre of $A/\mathfrak p$ at $(0)$. The quotient
$A/\mathfrak p$ has a dualizing complex
(Lemma \ref{lemma-dualizing-quotient}).
Thus it suffices to check the statement
when $A$ is a local domain and $\mathfrak p = (0)$.
Let $\omega_A^\bullet$ be a dualizing complex for $A$. Then
$\omega_A^\bullet \otimes_A A^\wedge$ is a dualizing complex
for the completion $A^\wedge$
(Lemma \ref{lemma-flat-unramified}).
Then $\omega_A^\bullet \otimes_A K$ is a dualizing
complex for the fraction field $K$ of $A$
(Lemma \ref{lemma-dualizing-localize}).
Hence $\omega_A^\bullet \otimes_A K$
is isomorphic ot $K[n]$ for some $n \in \mathbf{Z}$.
Similarly, we conclude a dualizing complex for the formal fibre
$A^\wedge \otimes_A K$ is
$$
\omega_A^\bullet \otimes_A A^\wedge \otimes_{A^\wedge} (A^\wedge \otimes_A K) =
(\omega_A^\bullet \otimes_A K) \otimes_K (A^\wedge \otimes_A K) \cong
(A^\wedge \otimes_A K)[n]
$$
as desired.
\end{proof}

\noindent
Here is the verification promised in
Divided Power Algebra, Remark \ref{dpa-remark-no-good-ci-map}.

\begin{lemma}
\label{lemma-formal-fibres-lci}
Properties (A), (B), (C), (D), and (E) of
More on Algebra, Section \ref{more-algebra-section-properties-formal-fibres}
hold for $P(k \to R) =$``$R$ is a local complete intersection''.
See Divided Power Algebra, Definition \ref{dpa-definition-lci}.
\end{lemma}

\begin{proof}
Part (A). Let $k \subset K$ be a finitely generated field extension.
Let $R$ be a $k$-algebra which is a local complete intersection.
We can find a global complete intersection
$A = k[x_1, \ldots, x_n]/(f_1, \ldots, f_c)$
over $k$ such that $K$ is isomorphic to the fraction field of $A$, see
Algebra, Lemma \ref{algebra-lemma-colimit-syntomic}.
Then $R \to R \otimes_k A$ is a relative global complete intersection.
It follows that $R \otimes_k A$ is a local complete intersection
by Divided Power Algebra, Lemma \ref{dpa-lemma-avramov}.

\medskip\noindent
Proof of (B). This is clear
because a ring is a local complete intersection if and only if all of its
local rings are complete intersections.

\medskip\noindent
Part (C). Let $A \to B \to C$ be flat maps of Noetherian rings.
Assume the fibres of $A \to B$ are local complete intersections
and $B \to C$ is regular. We have to show the fibres of $A \to C$
are local complete intersections. Clearly, we may assume $A = k$ is a field.
Then we may assume that $B \to C$ is a regular local homomorphism
of Noetherian local rings. Then $B$ is a complete intersection and
$C/\mathfrak m_B C$ is regular, in particular a complete intersection
(by definition). Then $C$ is a complete intersection by
Divided Power Algebra, Lemma \ref{dpa-lemma-avramov}.

\medskip\noindent
Part (D). This follows by the same arguments as in (C) from
the other implication in
Divided Power Algebra, Lemma \ref{dpa-lemma-avramov}.
Part (E) is immediate as the condition does not refer to the ground
field.
\end{proof}






\section{Upper shriek algebraically}
\label{section-relative-dualizing-complex-algebraic}

\noindent
For a finite type homomorphism $R \to A$ of Noetherian rings
we will construct a functor $\varphi^! : D(R) \to D(A)$
well defined up to nonunique isomorphism which
as we will see in Duality for Schemes, Remark
\ref{duality-remark-local-calculation-shriek}
agrees up to isomorphism with the upper shriek functors
one encounters in the duality theory for schemes.
To motivate the construction we mention two additional properties:
\begin{enumerate}
\item $\varphi^!$ sends a dualizing complex for $R$ (if it exists)
to a dualizing complex for $A$, and
\item $\omega_{A/R}^\bullet = \varphi^!(R)$ is a kind of
relative dualizing complex: it lies in $D^b_{\textit{Coh}}(A)$ and restricts
to a dualizing complex on the fibres provided $R \to A$ is flat.
\end{enumerate}
These statemens are Lemmas \ref{lemma-shriek-dualizing-algebraic} and
\ref{lemma-relative-dualizing-algebraic}.

\medskip\noindent
Let $\varphi : R \to A$ be a finite type homomorphism of Noetherian rings.
We will define a functor $\varphi^! : D(R) \to D(A)$ in the following way
\begin{enumerate}
\item If $\varphi : R \to A$ is surjective we set
$\varphi^!(K) = R\Hom(A, K)$. Here we use the functor
$R\Hom(A, -) : D(R) \to D(A)$ of
Section \ref{section-trivial}, and
\item in general we choose a surjection $\psi : P \to A$ with
$P = R[x_1, \ldots, x_n]$ and we set
$\varphi^!(K) = \psi^!(K \otimes_R^\mathbf{L} P)[n]$.
Here we use the functor
$- \otimes_R^\mathbf{L} P : D(R) \to D(P)$
of More on Algebra, Section \ref{more-algebra-section-derived-base-change}.
\end{enumerate}
Note the shift $[n]$ by the number of variables in the polynomial
ring. This construction is {\bf not} canonical and the functor
$\varphi^!$ will only be well defined up to a (nonunique) isomorphism of
functors\footnote{It is possible to make the construction canonical:
use $\Omega^n_{P/R}[n]$ instead of $P[n]$ in the
construction and use this in Lemma \ref{lemma-well-defined}.
The material in this section becomes a lot more involved
if one wants to do this.}.

\begin{lemma}
\label{lemma-well-defined}
Let $\varphi : R \to A$ be a finite type homomorphism of
Noetherian rings. The functor $\varphi^!$ is well defined
up to isomorphism.
\end{lemma}

\begin{proof}
Suppose that $\psi_1 : P_1 = R[x_1, \ldots, x_n] \to A$ and
$\psi_2 : P_2 = R[y_1, \ldots, y_m] \to A$ are two
surjections from polynomial rings onto $A$. Then we get a
commutative diagram
$$
\xymatrix{
R[x_1, \ldots, x_n, y_1, \ldots, y_m]
\ar[d]^{x_i \mapsto g_i} \ar[rr]_-{y_j \mapsto f_j} & &
R[x_1, \ldots, x_n] \ar[d] \\
R[y_1, \ldots, y_m] \ar[rr] & & A
}
$$
where $f_j$ and $g_i$ are chosen such that $\psi_1(f_j) = \psi_2(y_j)$
and $\psi_2(g_i) = \psi_1(x_i)$. By symmetry it suffices to prove
the functors defined using $P \to A$ and $P[y_1, \ldots, y_m] \to A$
are isomorphic. By induction we may assume $m = 1$. This reduces
us to the case discussed in the next paragraph.

\medskip\noindent
Here $\psi : P \to A$ is given and $\chi : P[y] \to A$ induces
$\psi$ on $P$. Write $Q = P[y]$.
Choose $g \in P$ with $\psi(g) = \chi(y)$.
Denote $\pi : Q \to P$ the $P$-algebra map
with $\pi(y) = g$. Then $\chi = \psi \circ \pi$ and hence
$\chi^! = \psi^! \circ \pi^!$ as both are
adjoint to the restriction functor $D(A) \to D(Q)$ by the material
in Section \ref{section-trivial}. Thus
$$
\chi^!\left(K \otimes_R^\mathbf{L} Q\right)[n + 1] =
\psi^!\left(\pi^!\left(K \otimes_R^\mathbf{L} Q\right)[1]\right)[n]
$$
Hence it suffices to show that
$\pi^!(K \otimes_R^\mathbf{L} Q[1]) = K \otimes_R^\mathbf{L} P$
Thus it suffices to show that the functor
$\pi^!(-) : D(Q) \to D(P)$
is isomorphic to $K \mapsto K \otimes_Q^\mathbf{L} P[-1]$.
This follows from Lemma \ref{lemma-compute-for-effective-Cartier-algebraic}.
\end{proof}

\begin{lemma}
\label{lemma-shriek-boundedness}
Let $\varphi : R \to A$ be a finite type homomorphism of Noetherian rings.
\begin{enumerate}
\item $\varphi^!$ maps $D^+(R)$ into $D^+(A)$ and
$D^+_{\textit{Coh}}(R)$ into $D^+_{\textit{Coh}}(A)$.
\item if $\varphi$ is perfect, then $\varphi^!$ maps
$D^-(R)$ into $D^-(A)$,
$D^-_{\textit{Coh}}(R)$ into $D^-_{\textit{Coh}}(A)$, and
$D^b_{\textit{Coh}}(R)$ into $D^b_{\textit{Coh}}(A)$.
\end{enumerate}
\end{lemma}

\begin{proof}
Choose a factorization $R \to P \to A$ as in the definition of $\varphi^!$.
The functor $- \otimes_R^\mathbf{L} : D(R) \to D(P)$ preserves
the subcategories
$D^+, D^+_{\textit{Coh}}, D^-, D^-_{\textit{Coh}}, D^b_{\textit{Coh}}$.
The functor $R\Hom(A, -) : D(P) \to D(A)$
preserves $D^+$ and $D^+_{\textit{Coh}}$ by
Lemma \ref{lemma-exact-support-coherent}.
If $R \to A$ is perfect, then $A$ is perfect as a $P$-module, see
More on Algebra, Lemma \ref{more-algebra-lemma-perfect-ring-map}.
Recall that the restriction of $R\Hom(A, K)$ to $D(P)$ is
$R\Hom_P(A, K)$. By More on Algebra, Lemma
\ref{more-algebra-lemma-dual-perfect-complex}
we have $R\Hom_P(A, K) = E \otimes_P^\mathbf{L} K$ for
some perfect $E \in D(P)$. Since we can represent $E$ by
a finite complex of finite projective $P$-modules
it is clear that $R\Hom_P(A, K)$ is in
$D^-(P), D^-_{\textit{Coh}}(P), D^b_{\textit{Coh}}(P)$
as soon as $K$ is. Since the restriction functor
$D(A) \to D(P)$ reflects these subcategories, the
proof is complete.
\end{proof}

\begin{lemma}
\label{lemma-shriek-dualizing-algebraic}
Let $\varphi$ be a finite type homomorphism of Noetherian rings.
If $\omega_R^\bullet$ is a dualizing complex for $R$, then
$\varphi^!(\omega_R^\bullet)$ is a dualizing complex for $A$.
\end{lemma}

\begin{proof}
Follows from Lemmas
\ref{lemma-dualizing-polynomial-ring} and
\ref{lemma-dualizing-quotient},
\end{proof}

\begin{lemma}
\label{lemma-flat-bc}
Let $R \to R'$ be a flat homomorphism of Noetherian rings.
Let $\varphi : R \to A$ be a finite type ring map.
Let $\varphi' : R' \to A' = A \otimes_R R'$ be the map induced by $\varphi$.
Then we have a functorial maps
$$
\varphi^!(K) \otimes_A^\mathbf{L} A' \longrightarrow
(\varphi')^!(K \otimes_R^\mathbf{L} R')
$$
for $K$ in $D(R)$ which are isomorphisms for $K \in D^+(R)$.
\end{lemma}

\begin{proof}
Choose a factorization $R \to P \to A$ where $P$ is a polynomial ring over $R$.
This gives a corresponding factorization $R' \to P' \to A'$ by base change.
Since we have $(K \otimes_R^\mathbf{L} P) \otimes_P^\mathbf{L} P' =
(K \otimes_R^\mathbf{L} R') \otimes_{R'}^\mathbf{L} P'$
by More on Algebra, Lemma \ref{more-algebra-lemma-double-base-change}
it suffices to construct maps
$$
R\Hom(A, K \otimes_R^\mathbf{L} P[n]) \otimes_A^\mathbf{L} A'
\longrightarrow
R\Hom(A', (K \otimes_R^\mathbf{L} P[n]) \otimes_P^\mathbf{L} P')
$$
functorial in $K$. For this we use the map (\ref{equation-base-change})
constructed in Section \ref{section-base-change-trivial-duality}
for $P, A, P', A'$.
The map is an isomorphism for $K \in D^+(R)$ by
Lemma \ref{lemma-flat-bc-surjection}.
\end{proof}

\begin{lemma}
\label{lemma-bc}
Let $R \to R'$ be a homomorphism of Noetherian rings.
Let $\varphi : R \to A$ be a perfect ring map
(More on Algebra, Definition
\ref{more-algebra-definition-pseudo-coherent-perfect})
such that $R'$ and $A$ are tor independent over $R$.
Let $\varphi' : R' \to A' = A \otimes_R R'$ be the map induced by $\varphi$.
Then we have a functorial isomorphism
$$
\varphi^!(K) \otimes_A^\mathbf{L} A' =
(\varphi')^!(K \otimes_R^\mathbf{L} R')
$$
for $K$ in $D(R)$.
\end{lemma}

\begin{proof}
We may choose a factorization $R \to P \to A$ where $P$
is a polynomial ring over $R$ such that $A$ is a perfect $P$-module, see
More on Algebra, Lemma \ref{more-algebra-lemma-perfect-ring-map}.
This gives a corresponding factorization $R' \to P' \to A'$ by base change.
Since we have $(K \otimes_R^\mathbf{L} P) \otimes_P^\mathbf{L} P' =
(K \otimes_R^\mathbf{L} R') \otimes_{R'}^\mathbf{L} P'$
by More on Algebra, Lemma \ref{more-algebra-lemma-double-base-change}
it suffices to construct maps
$$
R\Hom(A, K \otimes_R^\mathbf{L} P[n]) \otimes_A^\mathbf{L} A'
\longrightarrow
R\Hom(A', (K \otimes_R^\mathbf{L} P[n]) \otimes_P^\mathbf{L} P')
$$
functorial in $K$. We have
$$
A \otimes_P^\mathbf{L} P' = A \otimes_R^\mathbf{L} R' = A'
$$
The first equality by
More on Algebra, Lemma \ref{more-algebra-lemma-base-change-comparison}
applied to $R, R', P, P'$. The second equality because
$A$ and $R'$ are tor independent over $R$. Hence $A$ and $P'$ are
tor independent over $P$ and we can use the map (\ref{equation-base-change})
constructed in Section \ref{section-base-change-trivial-duality} for
$P, A, P', A'$
get the desired arrow. By Lemma \ref{lemma-bc-surjection}
to finish the proof it suffices to prove that $A$ is a perfect $P$-module
which we saw above.
\end{proof}

\begin{lemma}
\label{lemma-bc-flat}
Let $R \to R'$ be a homomorphism of Noetherian rings.
Let $\varphi : R \to A$ be flat of finite type.
Let $\varphi' : R' \to A' = A \otimes_R R'$ be the map induced by $\varphi$.
Then we have a functorial isomorphism
$$
\varphi^!(K) \otimes_A^\mathbf{L} A' =
(\varphi')^!(K \otimes_R^\mathbf{L} R')
$$
for $K$ in $D(R)$.
\end{lemma}

\begin{proof}
Special case of Lemma \ref{lemma-bc} by
More on Algebra, Lemma
\ref{more-algebra-lemma-flat-finite-presentation-perfect}.
\end{proof}

\begin{lemma}
\label{lemma-composition-shriek-algebraic}
Let $A \xrightarrow{a} B \xrightarrow{b} C$ be finite type homomorphisms of
Noetherian rings. Then there is a transformation of functors
$b^! \circ a^! \to (b \circ a)^!$ which is an isomorphism on $D^+(A)$.
\end{lemma}

\begin{proof}
Choose a polynomial ring $P = A[x_1, \ldots, x_n]$ over $A$
and a surjection $P \to B$. Choose elements $c_1, \ldots, c_m \in C$
generating $C$ over $B$. Set $Q = P[y_1, \ldots, y_m]$ and
denote $Q' = Q \otimes_P B = B[y_1, \ldots, y_m]$.
Let $\chi : Q' \to C$ be the surjection sending $y_j$ to $c_j$.
Picture
$$
\xymatrix{
& Q \ar[r]_{\psi'} & Q' \ar[r]_\chi & C \\
A \ar[r] & P \ar[r]^\psi \ar[u] & B \ar[u]
}
$$
By Lemma \ref{lemma-flat-bc-surjection} for $M \in D(P)$ we have an arrow
$\psi^!(M) \otimes_B^\mathbf{L} Q' \to (\psi')^!(M \otimes_P^\mathbf{L} Q)$
which is an isomorphism whenever $M$ is bounded below. Also
we have $\chi^! \circ (\psi')^! = (\chi \circ \psi')^!$ as both
functors are adjoint to the restriction functor $D(C) \to D(Q)$
by Section \ref{section-trivial}. Then we see
\begin{align*}
b^!(a^!(K))
& =
\chi^!(\psi^!(K \otimes_A^\mathbf{L} P)[n] \otimes_B^\mathbf{L} Q)[m] \\
& \to
\chi^!((\psi')^!(K \otimes_A^\mathbf{L} P \otimes_P^\mathbf{L} Q))[n + m] \\
& =
(\chi \circ \psi')^!(K\otimes_A^\mathbf{L} Q)[n + m] \\
& =
(b \circ a)^!(K)
\end{align*}
where we have used in addition to the above
More on Algebra, Lemma \ref{more-algebra-lemma-double-base-change}.
\end{proof}

\begin{lemma}
\label{lemma-upper-shriek-finite}
Let $\varphi : R \to A$ be a finite map of Noetherian rings.
Then $\varphi^!$ is isomorphic to the functor
$R\Hom(A, -) : D(R) \to D(A)$ from
Section \ref{section-trivial}.
\end{lemma}

\begin{proof}
Suppose that $A$ is generated by $n > 1$ elements over $R$.
Then can factor $R \to A$ as a composition of two finite ring maps
where in both steps the number of generators is $< n$.
Since we have Lemma \ref{lemma-composition-shriek-algebraic} and
Lemma \ref{lemma-composition-right-adjoints}
we conclude that it suffices
to prove the lemma when $A$ is generated by one element over $R$.
Since $A$ is finite over $R$, it follows that $A$ is a quotient
of $B = R[x]/(f)$ where $f$ is a monic polynomial in $x$
(Algebra, Lemma \ref{algebra-lemma-finite-is-integral}).
Again using the lemmas on composition and the fact that we
have agreement for surjections by definition, we conclude that
it suffices to prove the lemma for $R \to B = R[x]/(f)$.
In this case, the functor $\varphi^!$ is isomorphic to
$K \mapsto K \otimes_R^\mathbf{L} B$; you prove this by
using Lemma \ref{lemma-compute-for-effective-Cartier-algebraic}
for the map $R[x] \to B$ (note that the shift in the definition
of $\varphi^!$ and in the lemma add up to zero).
For the functor $R\Hom(B, -) : D(R) \to D(B)$ we can use
Lemma \ref{lemma-RHom-is-tensor-special}
to see that it suffices to show $\Hom_R(B, R) \cong B$
as $B$-modules. Suppose that $f$ has degree $d$.
Then an $R$-basis for $B$ is given by $1, x, \ldots, x^{d - 1}$.
Let $\delta_i : B \to R$, $i = 0, \ldots, d - 1$
be the $R$-linear map which picks off the coefficient
of $x^i$ with respect to the given basis. Then
$\delta_0, \ldots, \delta_{d - 1}$ is a basis for $\Hom_R(B, R)$.
Finally, for $0 \leq i \leq d - 1$ a computation shows that
$$
x^i \delta_{d - 1} =
\delta_{d - 1 - i} + b_1 \delta_{d - i} + \ldots + b_i \delta_{d - 1}
$$
for some $c_1, \ldots, c_d \in R$\footnote{If
$f = x^d + a_1 x^{d - 1} + \ldots + a_d$, then
$c_1 = -a_1$, $c_2 = a_1^2 - a_2$, $c_3 = -a_1^3 + 2a_1a_2 -a_3$, etc.}.
Hence $\Hom_R(B, R)$ is a principal $B$-module with generator
$\delta_{d - 1}$. By looking
at ranks we conclude that it is a rank $1$ free $B$-module.
\end{proof}

\begin{lemma}
\label{lemma-upper-shriek-localize}
Let $R$ be a Noetherian ring and let $f \in R$.
If $\varphi$ denotes the map $R \to R_f$, then $\varphi^!$
is isomorphic to $- \otimes_R^\mathbf{L} R_f$.
More generally, if $\varphi : R \to R'$ is a map such that
$\Spec(R') \to \Spec(R)$ is an open immersion, then
$\varphi^!$ is isomorphic to $- \otimes_R^\mathbf{L} R'$.
\end{lemma}

\begin{proof}
Choose the presentation $R \to R[x] \to R[x]/(fx - 1) = R_f$ and observe
that $fx - 1$ is a nonzerodivisor in $R[x]$. Thus we can apply
using Lemma \ref{lemma-compute-for-effective-Cartier-algebraic}
to compute the functor $\varphi^!$. Details omitted;
note that the shift in the definition
of $\varphi^!$ and in the lemma add up to zero.

\medskip\noindent
In the general case note that $R' \otimes_R R' = R'$.
Hence the result follows from the base change results
above. Either Lemma \ref{lemma-flat-bc} or
Lemma \ref{lemma-bc} will do.
\end{proof}

\begin{lemma}
\label{lemma-upper-shriek-is-tensor-functor}
Let $\varphi : R \to A$ be a perfect homomorphism of Noetherian rings
(for example $\varphi$ is flat of finite type).
Then $\varphi^!(K) = K \otimes_R^\mathbf{L} \varphi^!(R)$
for $K \in D(R)$.
\end{lemma}

\begin{proof}
(The parenthetical statement follows from
More on Algebra, Lemma
\ref{more-algebra-lemma-flat-finite-presentation-perfect}.)
We can choose a factorization $R \to P \to A$ where $P$ is a polynomial
ring in $n$ variables over $R$ and then $A$ is a perfect $P$-module, see
More on Algebra, Lemma \ref{more-algebra-lemma-perfect-ring-map}.
Recall that $\varphi^!(K) = R\Hom(A, K \otimes_R^\mathbf{L} P[n])$.
Thus the result follows from
Lemma \ref{lemma-RHom-is-tensor-special}
and More on Algebra, Lemma \ref{more-algebra-lemma-double-base-change}.
\end{proof}

\begin{lemma}
\label{lemma-relative-dualizing-if-have-omega}
Let $\varphi : A \to B$ be a finite type homomorphism of Noetherian rings.
Let $\omega_A^\bullet$ be a dualizing complex for $A$. Set
$\omega_B^\bullet = \varphi^!(\omega_A^\bullet)$. Denote
$D_A(K) = R\Hom_A(K, \omega_A^\bullet)$ for $K \in D_{\textit{Coh}}(A)$
and
$D_B(L) = R\Hom_B(L, \omega_B^\bullet)$ for $L \in D_{\textit{Coh}}(B)$.
Then there is a functorial isomorphism
$$
\varphi^!(K) = D_B(D_A(K) \otimes_A^\mathbf{L} B)
$$
for $K \in D_{\textit{Coh}}(A)$.
\end{lemma}

\begin{proof}
Observe that $\omega_B^\bullet$ is a dualizing complex for $B$ by
Lemma \ref{lemma-shriek-dualizing-algebraic}.
Let $A \to B \to C$ be finite type homomorphisms of Noetherian rings.
If the lemma holds for $A \to B$ and $B \to C$, then the lemma holds for
$A \to C$. This follows from
Lemma \ref{lemma-composition-shriek-algebraic}
and the fact that $D_B \circ D_B \cong \text{id}$ by
Lemma \ref{lemma-dualizing}.
Thus it suffices to prove the lemma in case $A \to B$ is
a surjection and in the case where $B$ is a
polynomial ring over $A$.

\medskip\noindent
Assume $B = A[x_1, \ldots, x_n]$. Since $D_A \circ D_A \cong \text{id}$,
it suffices to prove
$D_B(K \otimes_A B) \cong D_A(K) \otimes_A B[n]$ for $K$
in $D_{\textit{Coh}}(A)$.
Choose a bounded complex $I^\bullet$ of injectives representing
$\omega_A^\bullet$. Choose a quasi-isomorphism
$I^\bullet \otimes_A B \to J^\bullet$ where $J^\bullet$
is a bounded complex of $B$-modules. Given a complex
$K^\bullet$ of $A$-modules, consider the obvious
map of complexes
$$
\Hom^\bullet(K^\bullet, I^\bullet) \otimes_A B[n]
\longrightarrow
\Hom^\bullet(K^\bullet \otimes_A B, J^\bullet[n])
$$
The left hand side represents $D_A(K) \otimes_A B[n]$ and the right hand
side represents $D_B(K \otimes_A B)$. Thus it suffices to prove this
map is a quasi-isomorphism if the cohomology modules
of $K^\bullet$ are finite $A$-modules. Observe that the
cohomology of the complex in degree $r$ (on either side)
only depends on finitely many of the $K^i$. Thus we may
replace $K^\bullet$ by a truncation, i.e., we may assume
$K^\bullet$ represents an object of $D^-_{\textit{Coh}}(A)$.
Then $K^\bullet$ is quasi-isomorphic to a bounded
above complex of finite free $A$-modules.
Therefore we may assume $K^\bullet$ is a bounded
above complex of finite free $A$-modules.
In this case it is easy to that the
displayed map is an isomorphism of complexes which finishes
the proof in this case.

\medskip\noindent
Assume that $A \to B$ is surjective. Denote $i_* : D(B) \to D(A)$
the restriction functor and recall that $\varphi^!(-) = R\Hom(A, -)$
is a right adjoint to $i_*$ (Lemma \ref{lemma-right-adjoint}).
For $F \in D(B)$ we have
\begin{align*}
\Hom_B(F, D_B(D_A(K) \otimes_A^\mathbf{L} B))
& =
\Hom_B((D_A(K) \otimes_A^\mathbf{L} B) \otimes_B^\mathbf{L} F,
\omega_B^\bullet) \\
& =
\Hom_A(D_A(K) \otimes_A^\mathbf{L} i_*F, \omega_A^\bullet) \\
& =
\Hom_A(i_*F, D_A(D_A(K))) \\
& =
\Hom_A(i_*F, K) \\
& =
\Hom_B(F, \varphi^!(K))
\end{align*}
The first equality follows from More on Algebra, Lemma
\ref{more-algebra-lemma-internal-hom} and the definition
of $D_B$. The second equality by the adjointness mentioned
above and the equality
$i_*((D_A(K) \otimes_A^\mathbf{L} B) \otimes_B^\mathbf{L} F) =
D_A(K) \otimes_A^\mathbf{L} i_*F$
(More on Algebra, Lemma \ref{more-algebra-lemma-derived-base-change}).
The third equality follows from More on Algebra, Lemma
\ref{more-algebra-lemma-internal-hom}. The fourth because
$D_A \circ D_A = \text{id}$. The final equality by adjointness again.
Thus the result holds by the Yoneda lemma.
\end{proof}






\section{Relative dualizing complexes in the Noetherian case}
\label{section-relative-dualizing-complexes-Noetherian}

\noindent
Let $\varphi : R \to A$ be a finite type homomorphism of
Noetherian rings. Then we define the {\it relative dualizing
complex of $A$ over $R$} as the object
$$
\omega_{A/R}^\bullet = \varphi^!(R)
$$
of $D(A)$. Here $\varphi^!$ is as in
Section \ref{section-relative-dualizing-complex-algebraic}.
From the material in that section we see that
$\omega_{A/R}^\bullet$ is well defined up to (non-unique) isomorphism.

\begin{lemma}
\label{lemma-base-change-relative-algebraic}
Let $R \to R'$ be a homomorphism of Noetherian rings.
Let $R \to A$ be of finite type. Set $A' = A \otimes_R R'$. If
\begin{enumerate}
\item $R \to R'$ is flat, or
\item $R \to A$ is flat, or
\item $R \to A$ is perfect
and $R'$ and $A$ are tor independent over $R$,
\end{enumerate}
then there is an isomorphism
$\omega_{A/R}^\bullet \otimes_A^\mathbf{L} A' \to \omega^\bullet_{A'/R'}$
in $D(A')$.
\end{lemma}

\begin{proof}
Follows from Lemmas \ref{lemma-flat-bc}, \ref{lemma-bc-flat}, and
\ref{lemma-bc} and the definitions.
\end{proof}

\begin{lemma}
\label{lemma-relative-dualizing-algebraic}
Let $\varphi : R \to A$ be a flat finite type map of Noetherian rings.
Then
\begin{enumerate}
\item $\omega_{A/R}^\bullet$ is in $D^b_{\textit{Coh}}(A)$
and $R$-perfect (More on Algebra,
Definition \ref{more-algebra-definition-relatively-perfect}),
\item $A \to R\Hom_A(\omega_{A/R}^\bullet, \omega_{A/R}^\bullet)$
is an isomorphism, and
\item for every map $R \to k$ to a field the base change
$\omega_{A/R}^\bullet \otimes_A^\mathbf{L} (A \otimes_R k)$
is a dualizing complex for $A \otimes_R k$.
\end{enumerate}
\end{lemma}

\begin{proof}
Choose $R \to P \to A$ as in the definition of $\varphi^!$.
Recall that $R \to A$ is a perfect ring map
(More on Algebra, Lemma
\ref{more-algebra-lemma-flat-finite-presentation-perfect}) and
hence $A$ is perfect as a $P$-modue
(More on Algebra, Lemma \ref{more-algebra-lemma-perfect-ring-map}).
This shows that $\omega_{A/R}^\bullet$ is in $D^b_{\textit{Coh}}(A)$
by Lemma \ref{lemma-shriek-boundedness}.
To show $\omega_{A/R}^\bullet$ is $R$-perfect it suffices to
show it has finite tor dimension as a complex of $R$-modules.
This is true because
$\omega_{A/R}^\bullet = \varphi^!(R) = R\Hom(A, P)[n]$
maps to $R\Hom_P(A, P)[n]$ in $D(P)$, which is perfect in $D(P)$
(More on Algebra, Lemma \ref{more-algebra-lemma-dual-perfect-complex}),
hence has finite tor dimension in $D(R)$
as $R \to P$ is flat. This proves (1).

\medskip\noindent
Proof of (2). The object
$R\Hom_A(\omega_{A/R}^\bullet, \omega_{A/R}^\bullet)$
of $D(A)$ maps in $D(P)$ to
\begin{align*}
R\Hom_P(\omega_{A/R}^\bullet, R\Hom(A, P)[n])
& =
R\Hom_P(R\Hom_P(A, P)[n], P)[n] \\
& =
R\Hom_P(R\Hom_P(A, P), P)
\end{align*}
This is equal to $A$ by the already used
More on Algebra, Lemma \ref{more-algebra-lemma-dual-perfect-complex}.

\medskip\noindent
Proof of (3). By Lemma \ref{lemma-base-change-relative-algebraic}
there is an isomorphism
$$
\omega_{A/R}^\bullet \otimes_A^\mathbf{L} (A \otimes_R k) \cong
\omega^\bullet_{A \otimes_R k/k}
$$
and the right hand side is a dualizing complex by
Lemma \ref{lemma-shriek-dualizing-algebraic}.
\end{proof}

\begin{lemma}
\label{lemma-base-change-dualizing-over-field}
Let $K/k$ be an extension of fields. Let $A$ be a finite type
$k$-algebra. Let $A_K = A \otimes_k K$. If
$\omega_A^\bullet$ is a dualizing complex for $A$, then
$\omega_A^\bullet \otimes_A A_K$ is a dualizing complex for $A_K$.
\end{lemma}

\begin{proof}
By the uniqueness of dualizing complexes, it doesn't matter which
dualizing complex we pick for $A$; we omit the detailed proof.
Denote $\varphi : k \to A$ the algebra structure.
We may take $\omega_A^\bullet = \varphi^!(k[0])$ by
Lemma \ref{lemma-shriek-dualizing-algebraic}.
We conclude by
Lemma \ref{lemma-relative-dualizing-algebraic}.
\end{proof}

\begin{lemma}
\label{lemma-lci-shriek}
Let $\varphi : R \to A$ be a local complete intersection homomorphism of
Noetherian rings. Then $\omega_{A/R}^\bullet$ is an invertible object of
$D(A)$ and $\varphi^!(K) = K \otimes_R^\mathbf{L} \omega_{A/R}^\bullet$
for all $K \in D(R)$.
\end{lemma}

\begin{proof}
Recall that a local complete intersection homomorphism is a perfect
ring map by More on Algebra, Lemma \ref{more-algebra-lemma-lci-perfect}.
Hence the final statement holds by
Lemma \ref{lemma-upper-shriek-is-tensor-functor}.
By More on Algebra, Definition
\ref{more-algebra-definition-local-complete-intersection}
we can write $A = R[x_1, \ldots, x_n]/I$ where $I$ is a
Koszul-regular ideal.
The construction of $\varphi^!$ in
Section \ref{section-relative-dualizing-complex-algebraic}
shows that it suffices to show the lemma in case
$A = R/I$ where $I \subset R$ is a Koszul-regular ideal.
Checking $\omega_{A/R}^\bullet$ is invertible in $D(A)$
is local on $\Spec(A)$ by More on Algebra, Lemma
\ref{more-algebra-lemma-invertible-derived}.
Moreover, formation of $\omega_{A/R}^\bullet$ commutes with
localization on $R$ by Lemma \ref{lemma-flat-bc}.
Combining
More on Algebra, Definition \ref{more-algebra-definition-regular-ideal} and
Lemma \ref{more-algebra-lemma-noetherian-finite-all-equivalent} and
Algebra, Lemma \ref{algebra-lemma-regular-sequence-in-neighbourhood}
we can find $g_1, \ldots, g_r \in R$ generating the unit ideal in $A$
such that $I_{g_j} \subset R_{g_j}$ is generated by a regular sequence.
Thus we may assume $A = R/(f_1, \ldots, f_c)$ where $f_1, \ldots, f_c$
is a regular sequence in $R$. Then we consider the ring maps
$$
R \to R/(f_1) \to R/(f_1, f_2) \to \ldots \to R/(f_1, \ldots, f_c) = A
$$
and we use Lemma \ref{lemma-composition-shriek-algebraic}
(and the final statement already proven)
to see that it suffices to prove the lemma for each step.
Finally, in case $A = R/(f)$ for some nonzerodivisor $f$
we see that the lemma is true since $\varphi^!(R) = R\Hom(A, R)$
is invertible by Lemma \ref{lemma-compute-for-effective-Cartier-algebraic}.
\end{proof}

\begin{lemma}
\label{lemma-gorenstein-shriek}
Let $\varphi : R \to A$ be a flat finite type homomorphism of Noetherian rings.
The following are equivalent
\begin{enumerate}
\item the fibres $A \otimes_R \kappa(\mathfrak p)$ are Gorenstein
for all primes $\mathfrak p \subset R$, and
\item $\omega_{A/R}^\bullet$ is an invertible object of $D(A)$, see
More on Algebra, Lemma \ref{more-algebra-lemma-invertible-derived}.
\end{enumerate}
\end{lemma}

\begin{proof}
If (2) holds, then the fibre rings $A \otimes_R \kappa(\mathfrak p)$
have invertible dualizing complexes, and hence are Gorenstein.
See Lemmas \ref{lemma-relative-dualizing-algebraic} and \ref{lemma-gorenstein}.

\medskip\noindent
For the converse, assume (1).
Observe that $\omega_{A/R}^\bullet$ is in $D^b_{\textit{Coh}}(A)$
by Lemma \ref{lemma-shriek-boundedness} (since flat finite type homomorphisms
of Noetherian rings are perfect, see 
More on Algebra, Lemma
\ref{more-algebra-lemma-flat-finite-presentation-perfect}).
Take a prime $\mathfrak q \subset A$ lying over $\mathfrak p \subset R$.
Then
$$
\omega_{A/R}^\bullet \otimes_A^\mathbf{L} \kappa(\mathfrak q) =
\omega_{A/R}^\bullet \otimes_A^\mathbf{L}
(A \otimes_R \kappa(\mathfrak p))
\otimes_{(A \otimes_R \kappa(\mathfrak p))}^\mathbf{L}
\kappa(\mathfrak q)
$$
Applying Lemmas \ref{lemma-relative-dualizing-algebraic} and
\ref{lemma-gorenstein} and assumption (1) we find that this complex has $1$
nonzero cohomology group which is a $1$-dimensional
$\kappa(\mathfrak q)$-vector space. By
More on Algebra, Lemma
\ref{more-algebra-lemma-lift-bounded-pseudo-coherent-to-perfect}
we conclude that $(\omega_{A/R}^\bullet)_f$ is an invertible
object of $D(A_f)$ for some $f \in A$, $f \not \in \mathfrak q$.
This proves (2) holds.
\end{proof}

\noindent
The following lemma is useful to see how dimension functions change
when passing to a finite type algebra over a Noetherian ring.

\begin{lemma}
\label{lemma-shriek-normalized}
Let $\varphi : R \to A$ be a finite type homomorphism of Noetherian rings.
Assume $R$ local and let $\mathfrak m \subset A$ be a maximal
ideal lying over the maximal ideal of $R$. If $\omega_R^\bullet$
is a normalized dualizing complex for $R$, then
$\varphi^!(\omega_R^\bullet)_\mathfrak m$ is a normalized
dualizing complex for $A_\mathfrak m$.
\end{lemma}

\begin{proof}
We already know that $\varphi^!(\omega_R^\bullet)$ is a dualizing
complex for $A$, see Lemma \ref{lemma-shriek-dualizing-algebraic}.
Choose a factorization $R \to P \to A$ with $P = R[x_1, \ldots, x_n]$
as in the construction of $\varphi^!$. If we can prove the
lemma for $R \to P$ and the maximal ideal $\mathfrak m'$ of $P$ corresponding to
$\mathfrak m$, then we obtain the result for $R \to A$ by
applying Lemma \ref{lemma-normalized-quotient} to
$P_{\mathfrak m'} \to A_\mathfrak m$ or by applying
Lemma \ref{lemma-quotient-function} to $P \to A$.
In the case $A = R[x_1, \ldots, x_n]$ we see that
$\dim(A_\mathfrak m) = \dim(R) + n$ for example by
Algebra, Lemma \ref{algebra-lemma-dimension-base-fibre-equals-total}
(combined with Algebra, Lemma \ref{algebra-lemma-dim-affine-space}
to compute the dimension of the fibre).
The fact that $\omega_R^\bullet$ is normalized means
that $i = -\dim(R)$ is the smallest index such that
$H^i(\omega_R^\bullet)$ is nonzero (follows from
Lemmas \ref{lemma-sitting-in-degrees} and
\ref{lemma-nonvanishing-generically-local}).
Then $\varphi^!(\omega_R^\bullet)_\mathfrak m =
\omega_R^\bullet \otimes_R A_\mathfrak m[n]$
has its first nonzero cohomology module in degree $-\dim(R) - n$
and therefore is the normalized dualizing complex for $A_\mathfrak m$.
\end{proof}

\begin{lemma}
\label{lemma-relative-dualizing-trivial-vanishing}
Let $R \to A$ be a finite type homomorphism of Noetherian rings.
Let $\mathfrak q \subset A$ be a prime ideal lying over
$\mathfrak p \subset R$. Then
$$
H^i(\omega_{A/R}^\bullet)_\mathfrak q \not = 0
\Rightarrow - d \leq i
$$
where $d$ is the dimension of the fibre of $\Spec(A) \to \Spec(R)$
over $\mathfrak p$ at the point $\mathfrak q$.
\end{lemma}

\begin{proof}
Choose a factorization $R \to P \to A$ with $P = R[x_1, \ldots, x_n]$
as in Section \ref{section-relative-dualizing-complex-algebraic}
so that $\omega_{A/R}^\bullet = R\Hom(A, P)[n]$.
We have to show that $R\Hom(A, P)_\mathfrak q$
has vanishing cohomology in degrees $< n - d$.
By Lemma \ref{lemma-RHom-ext} this means we have to
show that $\Ext_P^i(P/I, P)_{\mathfrak r} = 0$ for $i < n - d$
where $\mathfrak r \subset P$ is the prime corresponding to $\mathfrak q$
and $I$ is the kernel of $P \to A$.
We may rewrite this as
$\Ext_{P_\mathfrak r}^i(P_\mathfrak r/IP_\mathfrak r, P_\mathfrak r)$
by More on Algebra, Lemma
\ref{more-algebra-lemma-pseudo-coherence-and-base-change-ext}.
Thus we have to show
$$
\text{depth}_{IP_\mathfrak r}(P_\mathfrak r) \geq n - d
$$
by Lemma \ref{lemma-depth}.
By Lemma \ref{lemma-depth-flat-CM} we have
$$
\text{depth}_{IP_\mathfrak r}(P_\mathfrak r) \geq
\dim((P \otimes_R \kappa(\mathfrak p))_\mathfrak r) -
\dim((P/I \otimes_R \kappa(\mathfrak p))_\mathfrak r)
$$
The two expressions on the right hand side agree by
Algebra, Lemma \ref{algebra-lemma-codimension}.
\end{proof}

\begin{lemma}
\label{lemma-relative-dualizing-flat-vanishing}
Let $R \to A$ be a flat finite type homomorphism of Noetherian rings.
Let $\mathfrak q \subset A$ be a prime ideal lying over
$\mathfrak p \subset R$. Then
$$
H^i(\omega_{A/R}^\bullet)_\mathfrak q \not = 0
\Rightarrow - d \leq i \leq 0
$$
where $d$ is the dimension of the fibre of $\Spec(A) \to \Spec(R)$
over $\mathfrak p$ at the point $\mathfrak q$. If all fibres of
$\Spec(A) \to \Spec(R)$ have dimension $\leq d$, then
$\omega_{A/R}^\bullet$ has tor amplitude in $[-d, 0]$
as a complex of $R$-modules.
\end{lemma}

\begin{proof}
The lower bound has been shown in
Lemma \ref{lemma-relative-dualizing-trivial-vanishing}.
Choose a factorization $R \to P \to A$ with $P = R[x_1, \ldots, x_n]$
as in Section \ref{section-relative-dualizing-complex-algebraic}
so that $\omega_{A/R}^\bullet = R\Hom(A, P)[n]$.
The upper bound means that $\Ext^i_P(A, P)$ is zero for $i > n$.
This follows from
More on Algebra, Lemma \ref{more-algebra-lemma-perfect-over-polynomial-ring}
which shows that $A$ is a perfect $P$-module with
tor amplitude in $[-n, 0]$.

\medskip\noindent
Proof of the final statement. Let $R \to R'$ be a ring homomorphism
of Noetherian rings. Set $A' = A \otimes_R R'$. Then
$$
\omega_{A'/R'}^\bullet =
\omega_{A/R}^\bullet \otimes_A^\mathbf{L} A' =
\omega_{A/R}^\bullet \otimes_R^\mathbf{L} R'
$$
The first isomorphism by Lemma \ref{lemma-base-change-relative-algebraic}
and the second, which takes place in $D(R')$, by
More on Algebra, Lemma \ref{more-algebra-lemma-base-change-comparison}.
By the first part of the proof
(note that the fibres of $\Spec(A') \to \Spec(R')$ have dimension $\leq d$)
we conclude that $\omega_{A/R}^\bullet \otimes_R^\mathbf{L} R'$
has cohomology only in degrees $[-d, 0]$. Taking $R' = R \oplus M$
to be the square zero thickening of $R$ by a finite $R$-module $M$,
we see that $R\Hom(A, P) \otimes_R^\mathbf{L} M$
has cohomology only in the interval $[-d, 0]$ for any finite $R$-module $M$.
Since any $R$-module is a filtered colimit of finite $R$-modules
and since tensor products commute with colimits we conclude.
\end{proof}

\begin{lemma}
\label{lemma-relative-dualizing-CM-vanishing}
Let $R \to A$ be a finite type homomorphism of Noetherian rings.
Let $\mathfrak p \subset R$ be a prime ideal. Assume
\begin{enumerate}
\item $R_\mathfrak p$ is Cohen-Macaulay, and
\item for any minimal prime $\mathfrak q \subset A$ we have
$\text{trdeg}_{\kappa(R \cap \mathfrak q)} \kappa(\mathfrak q) \leq r$.
\end{enumerate}
Then
$$
H^i(\omega_{A/R}^\bullet)_\mathfrak p \not = 0 \Rightarrow - r \leq i
$$
and $H^{-r}(\omega_{A/R}^\bullet)_\mathfrak p$ is $(S_2)$
as an $A_\mathfrak p$-module.
\end{lemma}

\begin{proof}
We may replace $R$ by $R_\mathfrak p$ by
Lemma \ref{lemma-base-change-relative-algebraic}.
Thus we may assume $R$ is a Cohen-Macaulay local ring
and we have to show the assertions of the lemma
for the $A$-modules $H^i(\omega_{A/R}^\bullet)$.

\medskip\noindent
Let $R^\wedge$ be the completion of $R$.
The map $R \to R^\wedge$ is flat and $R^\wedge$ is Cohen-Macaulay
(More on Algebra, Lemma \ref{more-algebra-lemma-completion-CM}).
Observe that the minimal primes of $A \otimes_R R^\wedge$
lie over minimal primes of $A$ by the flatness of
$A \to A \otimes_R R^\wedge$ (and going down for flatness, see
Algebra, Lemma \ref{algebra-lemma-flat-going-down}).
Thus condition (2) holds for the finite type ring map
$R^\wedge \to A \otimes_R R^\wedge$ by
Morphisms, Lemma \ref{morphisms-lemma-dimension-fibre-after-base-change}.
Appealing to Lemma \ref{lemma-base-change-relative-algebraic}
once again it suffices to prove the lemma for
$R^\wedge \to A \otimes_R R^\wedge$. In this way, using
Lemma \ref{lemma-ubiquity-dualizing},
we may assume $R$ is a Noetherian local
Cohen-Macaulay ring which has a dualizing complex $\omega_R^\bullet$.

\medskip\noindent
Let $\mathfrak m \subset A$ be a maximal ideal.
It suffices to show that the assertions of
the lemma hold for $H^i(\omega_{A/R}^\bullet)_\mathfrak m$.
If $\mathfrak m$ does not lie over the maximal ideal of $R$,
then we replace $R$ by a localization to reduce to this case
(small detail omitted). 

\medskip\noindent
We may assume $\omega_R^\bullet$ is normalized.
Setting $d = \dim(R)$ we see that $\omega_R^\bullet = \omega_R[d]$
for some $R$-module $\omega_R$, see
Lemma \ref{lemma-apply-CM}. Set
$\omega_A^\bullet = \varphi^!(\omega_R^\bullet)$.
By Lemma \ref{lemma-relative-dualizing-if-have-omega} we have
$$
\omega_{A/R}^\bullet =
R\Hom_A(\omega_R[d] \otimes_R^\mathbf{L} A, \omega_A^\bullet)
$$
By the dimension formula we have $\dim(A_\mathfrak m) \leq d + r$, see
Morphisms, Lemma \ref{morphisms-lemma-dimension-formula-general}
and use that $\kappa(\mathfrak m)$ is finite over the residue field of $R$
by the Hilbert Nullstellensatz.
By Lemma \ref{lemma-shriek-normalized}
we see that $(\omega_A^\bullet)_\mathfrak m$
is a normalized dualizing complex for $A_\mathfrak m$.
Hence $H^i((\omega_A^\bullet)_\mathfrak m)$ is nonzero
only for $-d - r \leq i \leq 0$, see
Lemma \ref{lemma-sitting-in-degrees}.
Since $\omega_R[d] \otimes_R^\mathbf{L} A$ lives in
degrees $\leq -d$ we conclude the vanishing holds.
Finally, we also see that
$$
H^{-r}(\omega_{A/R}^\bullet)_\mathfrak m =
\Hom_A(\omega_R \otimes_R A, H^{-d - r}(\omega_A^\bullet))_\mathfrak m
$$
Since $H^{-d - r}(\omega_A^\bullet)_\mathfrak m$ is $(S_2)$ by
Lemma \ref{lemma-depth-dualizing-module}
we find that the final statement is true by
More on Algebra, Lemma \ref{more-algebra-lemma-hom-into-S2}.
\end{proof}



\section{More on dualizing complexes}
\label{section-more-dualizing}

\noindent
Some lemmas which don't fit anywhere else very well.

\begin{lemma}
\label{lemma-descent}
Let $A \to B$ be a faithfully flat map of Noetherian rings.
If $K \in D(A)$ and $K \otimes_A^\mathbf{L} B$
is a dualizing complex for $B$, then $K$ is a dualizing complex
for $A$.
\end{lemma}

\begin{proof}
Since $A \to B$ is flat we have
$H^i(K) \otimes_A B = H^i(K \otimes_A^\mathbf{L} B)$.
Since $K \otimes_A^\mathbf{L} B$ is in $D^b_{\textit{Coh}}(B)$
we first find that $K$ is in $D^b(A)$ and then we see that
$H^i(K)$ is a finite $A$-module by
Algebra, Lemma \ref{algebra-lemma-descend-properties-modules}.
Let $M$ be a finite $A$-module. Then
$$
R\Hom_A(M, K) \otimes_A B = R\Hom_B(M \otimes_A B, K \otimes_A^\mathbf{L} B)
$$
by More on Algebra, Lemma \ref{more-algebra-lemma-base-change-RHom}.
Since $K \otimes_A^\mathbf{L} B$ has finite injective dimension,
say injective-amplitude in $[a, b]$, we see that the right hand side
has vanishing cohomology in degrees $> b$.
Since $A \to B$ is faithfully flat, we find
that $R\Hom_A(M, K)$ has vanishing cohomology in degrees $> b$.
Thus $K$ has finite injective dimension by
More on Algebra, Lemma \ref{more-algebra-lemma-injective-amplitude}.
To finish the proof we have to show that the map
$A \to R\Hom_A(K, K)$ is an isomorphism.
For this we again use
More on Algebra, Lemma \ref{more-algebra-lemma-base-change-RHom}
and the fact that
$B \to R\Hom_B(K \otimes_A^\mathbf{L} B, K \otimes_A^\mathbf{L} B)$
is an isomorphism.
\end{proof}

\begin{lemma}
\label{lemma-descent-ascent}
Let $\varphi : A \to B$ be a homomorphism of Noetherian rings. Assume
\begin{enumerate}
\item $A \to B$ is syntomic and induces a surjective map on spectra, or
\item $A \to B$ is a faithfully flat local complete intersection, or
\item $A \to B$ is faithfully flat of finite type with Gorenstein fibres.
\end{enumerate}
Then $K \in D(A)$ is a dualizing complex for $A$ if and only if
$K \otimes_A^\mathbf{L} B$ is a dualizing complex for $B$.
\end{lemma}

\begin{proof}
Observe that $A \to B$ satisfies (1) if and only if $A \to B$
satisfies (2) by  More on Algebra, Lemma \ref{more-algebra-lemma-syntomic-lci}.
Observe that in both (2) and (3) the relative dualzing
complex $\varphi^!(A) = \omega_{B/A}^\bullet$ is an invertible
object of $D(B)$, see
Lemmas \ref{lemma-lci-shriek} and \ref{lemma-gorenstein-shriek}.
Moreover we have
$\varphi^!(K) = K \otimes_A^\mathbf{L} \omega_{B/A}^\bullet$
in both cases, see Lemma \ref{lemma-upper-shriek-is-tensor-functor}
for case (3).
Thus $\varphi^!(K)$ is the same as $K \otimes_A^\mathbf{L} B$
up to tensoring with an invertible object of $D(B)$.
Hence $\varphi^!(K)$ is a dualizing complex for $B$
if and only if $K \otimes_A^\mathbf{L} B$ is
(as being a dualizing complex is local and invariant under shifts).
Thus we see that if $K$ is dualizing for $A$, then
$K \otimes_A^\mathbf{L} B$ is dualizing for $B$ by
Lemma \ref{lemma-shriek-dualizing-algebraic}.
To descend the property, see
Lemma \ref{lemma-descent}.
\end{proof}

\begin{lemma}
\label{lemma-injective-hull-goes-up}
Let $(A, \mathfrak m, \kappa) \to (B, \mathfrak n, l)$
be a flat local homorphism of Noetherian rings such that
$\mathfrak n = \mathfrak m B$. If $E$ is the injective
hull of $\kappa$, then $E \otimes_A B$ is the injective
hull of $l$.
\end{lemma}

\begin{proof}
Write $E = \bigcup E_n$ as in Lemma \ref{lemma-union-artinian}.
It suffices to show that $E_n \otimes_{A/\mathfrak m^n} B/\mathfrak n^n$
is the injective hull of $l$ over $B/\mathfrak n$.
This reduces us to the case where $A$ and $B$ are Artinian local.
Observe that $\text{length}_A(A) = \text{length}_B(B)$ and
$\text{length}_A(E) = \text{length}_B(E \otimes_A B)$
by Algebra, Lemma \ref{algebra-lemma-pullback-module}.
By Lemma \ref{lemma-finite} we have
$\text{length}_A(E) = \text{length}_A(A)$ and
$\text{length}_B(E') = \text{length}_B(B)$
where $E'$ is the injective hull of $l$ over $B$.
We conclude $\text{length}_B(E') = \text{length}_B(E \otimes_A B)$.
Observe that
$$
\dim_l((E \otimes_A B)[\mathfrak n]) =
\dim_l(E[\mathfrak m] \otimes_A B) =
\dim_\kappa(E[\mathfrak m]) = 1
$$
where we have used flatness of $A \to B$ and $\mathfrak n = \mathfrak mB$.
Thus there is an injective $B$-module map $E \otimes_A B \to E'$
by Lemma \ref{lemma-torsion-submodule-sum-injective-hulls}.
By equality of lengths shown above this is an isomorphism.
\end{proof}

\begin{lemma}
\label{lemma-injective-goes-up}
Let $\varphi : A \to B$ be a flat homorphism of Noetherian rings such
that for all primes $\mathfrak q \subset B$ we have
$\mathfrak p B_\mathfrak q = \mathfrak qB_\mathfrak q$
where $\mathfrak p = \varphi^{-1}(\mathfrak q)$, for example
if $\varphi$ is \'etale.
If $I$ is an injective $A$-module, then $I \otimes_A B$ is
an injective $B$-module.
\end{lemma}

\begin{proof}
\'Etale maps satisfy the assumption by
Algebra, Lemma \ref{algebra-lemma-etale-at-prime}.
By Lemma \ref{lemma-sum-injective-modules} and
Proposition \ref{proposition-structure-injectives-noetherian}
we may assume $I$ is the injective hull of $\kappa(\mathfrak p)$
for some prime $\mathfrak p \subset A$.
Then $I$ is a module over $A_\mathfrak p$.
It suffices to prove $I \otimes_A B = I \otimes_{A_\mathfrak p} B_\mathfrak p$
is injective as a $B_\mathfrak p$-module, see
Lemma \ref{lemma-injective-flat}.
Thus we may assume $(A, \mathfrak m, \kappa)$ is local Noetherian
and $I = E$ is the injective hull of the residue field $\kappa$.
Our assumption implies that the Noetherian ring $B/\mathfrak m B$
is a product of fields (details omitted).
Thus there are finitely many prime ideals
$\mathfrak m_1, \ldots, \mathfrak m_n$ in $B$
lying over $\mathfrak m$ and they are all maximal ideals.
Write $E = \bigcup E_n$ as in Lemma \ref{lemma-union-artinian}.
Then $E \otimes_A B = \bigcup E_n \otimes_A B$
and $E_n \otimes_A B$ is a finite $B$-module with support
$\{\mathfrak m_1, \ldots, \mathfrak m_n\}$ hence decomposes
as a product over the localizations at $\mathfrak m_i$.
Thus $E \otimes_A B = \prod (E \otimes_A B)_{\mathfrak m_i}$.
Since $(E \otimes_A B)_{\mathfrak m_i} = E \otimes_A B_{\mathfrak m_i}$
is the injective hull of the residue field of $\mathfrak m_i$
by Lemma \ref{lemma-injective-hull-goes-up} we conclude.
\end{proof}






\section{Relative dualizing complexes}
\label{section-relative-dualizing-complexes}

\noindent
For a finite type ring map $\varphi : R \to A$ of Noetherian rings
we have the relative dualizing complex $\omega_{A/R}^\bullet = \varphi^!(R)$
considered in Section \ref{section-relative-dualizing-complexes-Noetherian}.
If $R$ is not Noetherian, a similarly constructed complex will
in general not have good properties. In this section, we give a
definition of a relative dualizing complex for a flat and finitely presented
ring maps $R \to A$ of non-Noetherian rings. The definition is
chosen to globalize to flat and finitely presented morphisms
of schemes, see Duality for Schemes, Section
\ref{duality-section-relative-dualizing-complexes}. We will
show that relative dualizing complexes exist (when the definition
applies), are unique up to (noncanonical) isomorphism,
and that in the Noetherian case we recover the complex of
Section \ref{section-relative-dualizing-complexes-Noetherian}.

\medskip\noindent
The Noetherian reader may safely skip this section!

\begin{definition}
\label{definition-relative-dualizing-complex}
Let $R \to A$ be a flat ring map of finite presentation.
A {\it relative dualizing complex} is an object $K \in D(A)$ such that
\begin{enumerate}
\item $K$ is $R$-perfect (More on Algebra, Definition
\ref{more-algebra-definition-relatively-perfect}), and
\item $R\Hom_{A \otimes_R A}(A, K \otimes_A^\mathbf{L} (A \otimes_R A))$
is isomorphic to $A$.
\end{enumerate}
\end{definition}

\noindent
To understand this definition you may have to read and understand some
of the following lemmas. Lemmas \ref{lemma-relative-dualizing-noetherian} and
\ref{lemma-uniqueness-relative-dualizing} show this definition
does not clash with the definition in
Section \ref{section-relative-dualizing-complexes-Noetherian}.

\begin{lemma}
\label{lemma-uniqueness-relative-dualizing}
Let $R \to A$ be a flat ring map of finite presentation.
Any two relative dualizing complexes for $R \to A$ are isomorphic.
\end{lemma}

\begin{proof}
Let $K$ and $L$ be two relative dualizing complexes for $R \to A$.
Denote $K_1 = K \otimes_A^\mathbf{L} (A \otimes_R A)$
and $L_2 = (A \otimes_R A) \otimes_A^\mathbf{L} L$ the
derived base changes via the first and second coprojections
$A \to A \otimes_R A$. By symmetry the assumption on $L_2$
implies that $R\Hom_{A \otimes_R A}(A, L_2)$ is isomorphic to $A$.
By More on Algebra, Lemma
\ref{more-algebra-lemma-internal-hom-evaluate-tensor-isomorphism} part (3)
applied twice we have
$$
A \otimes_{A \otimes_R A}^\mathbf{L} L_2 \cong
R\Hom_{A \otimes_R A}(A, K_1 \otimes_{A \otimes_R A}^\mathbf{L} L_2) \cong
A \otimes_{A \otimes_R A}^\mathbf{L} K_1
$$
Applying the restriction functor $D(A \otimes_R A) \to D(A)$
for either coprojection we obtain the desired result.
\end{proof}

\begin{lemma}
\label{lemma-relative-dualizing-noetherian}
Let $\varphi : R \to A$ be a flat finite type ring map of Noetherian rings.
Then the relative dualizing complex $\omega_{A/R}^\bullet = \varphi^!(R)$
of Section \ref{section-relative-dualizing-complexes-Noetherian}
is a relative dualizing complex in the sense of
Definition \ref{definition-relative-dualizing-complex}.
\end{lemma}

\begin{proof}
From Lemma \ref{lemma-relative-dualizing-algebraic} we see that
$\varphi^!(R)$ is $R$-perfect.
Denote $\delta : A \otimes_R A \to A$ the multiplication map
and $p_1, p_2 : A \to A \otimes_R A$ the coprojections.
Then
$$
\varphi^!(R) \otimes_A^\mathbf{L} (A \otimes_R A) =
\varphi^!(R) \otimes_{A, p_1}^\mathbf{L} (A \otimes_R A) =
p_2^!(A)
$$
by Lemma \ref{lemma-flat-bc}. Recall that
$
R\Hom_{A \otimes_R A}(A, \varphi^!(R) \otimes_A^\mathbf{L} (A \otimes_R A))
$
is the image of $\delta^!(\varphi^!(R) \otimes_A^\mathbf{L} (A \otimes_R A))$
under the restriction map $\delta_* : D(A) \to D(A \otimes_R A)$.
Use the definition of $\delta^!$ from
Section \ref{section-relative-dualizing-complex-algebraic}
and Lemma \ref{lemma-RHom-ext}.
Since $\delta^!(p_2^!(A)) \cong A$ by
Lemma \ref{lemma-composition-shriek-algebraic}
we conclude.
\end{proof}

\begin{lemma}
\label{lemma-base-change-relative-dualizing}
Let $R \to A$ be a flat ring map of finite presentation. Then
\begin{enumerate}
\item there exists a relative dualizing complex $K$ in $D(A)$, and
\item for any ring map $R \to R'$ setting $A' = A \otimes_R R'$
and $K' = K \otimes_A^\mathbf{L} A'$, then $K'$ is a
relative dualizing complex for $R' \to A'$.
\end{enumerate}
Moreover, if
$$
\xi : A \longrightarrow K \otimes_A^\mathbf{L} (A \otimes_R A)
$$
is a generator for the cyclic module
$\Hom_{D(A \otimes_R A)}(A, K \otimes_A^\mathbf{L} (A \otimes_R A))$
then in (2) the derived base change of $\xi$ by
$A \otimes_R A \to A' \otimes_{R'} A'$ is a generator for
the cyclic module
$\Hom_{D(A' \otimes_{R'} A')}(A',
K' \otimes_{A'}^\mathbf{L} (A' \otimes_{R'} A'))$
\end{lemma}

\begin{proof}
We first reduce to the Noetherian case. By
Algebra, Lemma \ref{algebra-lemma-flat-finite-presentation-limit-flat}
there exists a finite type $\mathbf{Z}$ subalgebra $R_0 \subset R$
and a flat finite type ring map $R_0 \to A_0$ such that
$A = A_0 \otimes_{R_0} R$. By Lemma \ref{lemma-relative-dualizing-noetherian}
there exists a relative
dualizing complex $K_0 \in D(A_0)$.
Thus if we show (2) for $K_0$, then we find that
$K_0 \otimes_{A_0}^\mathbf{L} A$ is
a dualizing complex for $R \to A$ and that it also satisfies (2)
by transitivity of derived base change.
The uniqueness of relative dualizing complexes
(Lemma \ref{lemma-uniqueness-relative-dualizing})
then shows that this holds for
any relative dualizing complex.

\medskip\noindent
Assume $R$ Noetherian and let $K$ be a relative dualizing complex
for $R \to A$. Given a ring map $R \to R'$ set $A' = A \otimes_R R'$
and $K' = K \otimes_A^\mathbf{L} A'$. To finish the proof we have
to show that $K'$ is a relative dualizing complex for $R' \to A'$.
By More on Algebra, Lemma
\ref{more-algebra-lemma-base-change-relatively-perfect}
we see that $K'$ is $R'$-perfect in all cases.
By Lemmas \ref{lemma-base-change-relative-algebraic} and
\ref{lemma-relative-dualizing-noetherian}
if $R'$ is Noetherian, then $K'$ is a relative dualizing complex
for $R' \to A'$ (in either sense).
Transitivity of derived tensor product shows that
$K \otimes_A^\mathbf{L} (A \otimes_R A)
\otimes_{A \otimes_R A}^\mathbf{L} (A' \otimes_{R'} A') =
K' \otimes_{A'}^\mathbf{L} (A' \otimes_{R'} A')$.
Flatness of $R \to A$ guarantees that
$A \otimes_{A \otimes_R A}^\mathbf{L} (A' \otimes_{R'} A') = A'$;
namely $A \otimes_R A$ and $R'$ are tor independent over $R$
so we can apply More on Algebra, Lemma
\ref{more-algebra-lemma-base-change-comparison}.
Finally, $A$ is pseudo-coherent as an $A \otimes_R A$-module
by More on Algebra, Lemma
\ref{more-algebra-lemma-more-relative-pseudo-coherent-is-moot}. Thus
we have checked all the assumptions of
More on Algebra, Lemma
\ref{more-algebra-lemma-compute-RHom-relatively-perfect}.
We find there exists a bounded below complex
$E^\bullet$ of $R$-flat finitely presented $A \otimes_R A$-modules
such that $E^\bullet \otimes_R R'$ represents
$R\Hom_{A' \otimes_{R'} A'}(A',
K' \otimes_{A'}^\mathbf{L} (A' \otimes_{R'} A'))$
and these identifications are compatible with derived base change.
Let $n \in \mathbf{Z}$, $n \not = 0$.
Define $Q^n$ by the sequence
$$
E^{n - 1} \to E^n \to Q^n \to 0
$$
Since $\kappa(\mathfrak p)$ is a Noetherian ring, we know that
$H^n(E^\bullet \otimes_R \kappa(\mathfrak p)) = 0$, see remarks above.
Chasing diagrams this means that
$$
Q^n \otimes_R \kappa(\mathfrak p) \to E^{n + 1} \otimes_R \kappa(\mathfrak p)
$$
is injective. Hence for a prime $\mathfrak q$ of $A \otimes_R A$
lying over $\mathfrak p$ we have $Q^n_\mathfrak q$ is $R_\mathfrak p$-flat
and $Q^n_\mathfrak p \to E^{n + 1}_\mathfrak q$ is
$R_\mathfrak p$-universally injective, see
Algebra, Lemma \ref{algebra-lemma-mod-injective}.
Since this holds for all primes,
we conclude that $Q^n$ is $R$-flat
and $Q^n \to E^{n + 1}$ is $R$-universally injective. In particular
$H^n(E^\bullet \otimes_R R') = 0$ for any ring map $R \to R'$.
Let $Z^0 = \Ker(E^0 \to E^1)$. Since there is an exact sequence
$0 \to Z^0 \to E^0 \to E^1 \to Q^1 \to 0$ we see that $Z^0$
is $R$-flat and that
$Z^0 \otimes_R R' = \Ker(E^0 \otimes_R R' \to E^1 \otimes_R R')$
for all $R \to R'$. Then the short exact sequence
$0 \to Q^{-1} \to Z^0 \to H^0(E^\bullet) \to 0$
shows that
$$
H^0(E^\bullet \otimes_R R') = H^0(E^\bullet) \otimes_R R'
= A \otimes_R R' = A'
$$
as desired. This equality furthermore gives the final assertion
of the lemma.
\end{proof}

\begin{lemma}
\label{lemma-relative-dualizing-RHom}
Let $R \to A$ be a flat ring map of finite presentation.
Let $K$ be a relative dualizing complex.
Then $A \to R\Hom_A(K, K)$ is an isomorphism.
\end{lemma}

\begin{proof}
By
Algebra, Lemma \ref{algebra-lemma-flat-finite-presentation-limit-flat}
there exists a finite type $\mathbf{Z}$ subalgebra $R_0 \subset R$
and a flat finite type ring map $R_0 \to A_0$ such that
$A = A_0 \otimes_{R_0} R$. By Lemmas
\ref{lemma-uniqueness-relative-dualizing},
\ref{lemma-relative-dualizing-noetherian}, and
\ref{lemma-base-change-relative-dualizing}
there exists a relative dualizing complex $K_0 \in D(A_0)$
and its derived base change is $K$.
This reduces us to the situation discussed in the next paragraph.

\medskip\noindent
Assume $R$ Noetherian and let $K$ be a relative dualizing complex
for $R \to A$. Given a ring map $R \to R'$ set $A' = A \otimes_R R'$
and $K' = K \otimes_A^\mathbf{L} A'$. To finish the proof we show
$R\Hom_{A'}(K', K') = A'$. By Lemma \ref{lemma-relative-dualizing-algebraic}
we know this is true whenever $R'$ is Noetherian.
Since a general $R'$ is a filtered colimit of Noetherian
$R$-algebras, we find the result holds by
More on Algebra, Lemma \ref{more-algebra-lemma-colimit-relatively-perfect}.
\end{proof}

\begin{lemma}
\label{lemma-relative-dualizing-composition}
Let $R \to A \to B$ be a ring maps which are flat and of finite presentation.
Let $K_{A/R}$ and $K_{B/A}$ be relative dualizing complexes for $R \to A$
and $A \to B$. Then $K = K_{A/R} \otimes_A^\mathbf{L} K_{B/A}$
is a relative dualizing complex for $R \to B$.
\end{lemma}

\begin{proof}
We will use reduction to the Noetherian case.
Namely, by Algebra, Lemma
\ref{algebra-lemma-flat-finite-presentation-limit-flat}
there exists a finite type $\mathbf{Z}$ subalgebra $R_0 \subset R$
and a flat finite type ring map $R_0 \to A_0$ such that
$A = A_0 \otimes_{R_0} R$. After increasing $R_0$ and correspondingly
replacing $A_0$ we may assume there is a flat
finite type ring map $A_0 \to B_0$ such that $B = B_0 \otimes_{R_0} R$
(use the same lemma). If we prove the lemma for $R_0 \to A_0 \to B_0$,
then the lemma follows by Lemmas
\ref{lemma-uniqueness-relative-dualizing},
\ref{lemma-relative-dualizing-noetherian}, and
\ref{lemma-base-change-relative-dualizing}.
This reduces us to the situation discussed in the next paragraph.

\medskip\noindent
Assume $R$ is Noetherian and denote $\varphi : R \to A$ and
$\psi : A \to B$ the given ring maps. Then $K_{A/R} \cong \varphi^!(R)$ and
$K_{B/A} \cong \psi^!(A)$, see references given above.
Then
$$
K = K_{A/R} \otimes_A^\mathbf{L} K_{B/A} \cong
\varphi^!(R) \otimes_A^\mathbf{L} \psi^!(A) \cong
\psi^!(\varphi^!(R)) \cong (\psi \circ \varphi)^!(R)
$$
by Lemmas \ref{lemma-upper-shriek-is-tensor-functor} and
\ref{lemma-composition-shriek-algebraic}. Thus $K$ is a relative
dualizing complex for $R \to B$.
\end{proof}





\input{chapters}

\bibliography{my}
\bibliographystyle{amsalpha}

\end{document}
